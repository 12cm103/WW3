\section{Introduction}

The importance of wind-waves in the Great Lakes Region cannot be overestimated.
Intense storms usually start developing in fall, and occur during
most of winter, sweeping the lakes with wind speeds that can reach over
100km/h. During November, when the lakes are still predominantly ice-free,
waves generated by such storms can reach heights of several meters,
becoming navigational hazards and threatening coastal and offshore
structures.

A historic example of how extreme wave conditions can become within
the Great Lakes was ``The Great Lakes Storm''
of November 1913. Considered the deadliest and most destructive natural
disaster ever to hit the lakes (Brown, 2002), on 9 November the storm
developed hurricane-force winds and severe sea states, which allegedly destroyed
19 ships and
killed over 250 crew. Ship losses surpassed what would correspond
today to more than US\$ 100 million. In a dramatic account of the
event, the New York Times reported on 23 November 1913 that ``ship
losses equalled those of the Titanic disaster'', concluding
that ``The incident ought to impress the lesson that
the dangers of the lakes rival those of the ocean''.

The 1913 storm developed after two storm fronts collided and intensified,
fuelled by the lakes' relatively warm waters, which identifies a seasonal
process
now called the ``November gale''. Such events produce
145 km/h (90 mph) winds, waves over 11 m (35 ft) high, and whiteout
snowsqualls. Analysis of these intense storms and its impact on humans,
engineering
structures and the landscape led to better forecasting and faster
responses to storm warnings, stronger and safer constructions (especially
of marine vessels), and improved preparedness.

Since the 1913 disaster, marine forecasting in the Great Lakes region has gone
a long way, particularly after the creation of NOAA's Great Lakes
Environmental Research Laboratory (GLERL), in 1974. GLERL started issuing wave
forecasts for the Great Lakes in the early 1980's, using a parametric,
first-generation wave model. The GLELR wave model is based on statistics
derived from measured winds and waves under fetch-limited evolution conditions.
GLERL wave forecasts are used operationally by several Weather Forecast
Offices of NOAA's National Weather Service (NWS).

With the advent of third-generation wind-wave models in the late
1980's, isolated efforts have been made towards using these more advanced
models in the Great Lakes. An example was an implementation, in 2002,
of the WAVEWATCH III model \cite{tolman02} for the Lake Superior,
with wind forcing specified by workstation versions of the ETA and
RAMS atmospheric models (Thomas Hultquist, personal communication). 

Results from these early attempts loosely
established the proof of concept, which eventually motivated, in late
2004, a more formal effort within the Environmental Modelling Center
(EMC), of NOAA's National Centers for Environmental Prediction (NCEP),
for funding the development an operational Great Lakes wave forecasting
system, based on a regional implementation of the WAVEWATCH III model. The
initiative led to the development and testing of a prototype wave forecasting
system, which became operational in August 2006 within the US National
Weather Service suite of numerical weather prediction models.

(Please add a paragraph or two with a brief summary of what was done
since August 2006 until today)

For the sake of illustrating the importance of wind-waves to the dynamical
Great Lakes system, Section 2 provides an overview of the more general
properties of the wave climate within the lakes region. Section
3 gives, a brief historical summary of the development of wind-wave
modelling approaches in the Great Lakes region, for generating hindcast
databases and forecasts systems, in the last few decades.
A chronological description of the deployment, and associated development
efforts, of the current operational GLW system at NWS, is made in
Section 4. A discussion of the current limitations, associated with
a more detailed assessment of new research being undertaken to overcome
the latter, is made in Section 5. Concluding remarks are made in Section
6.

