\vssub
\subsection{~Installing files}\label{sec:install}
\vssub

\noindent
In its packaged public version (tar file distribution), \ws\ is
contained in several files:

\begin{list}{???}{\parsep 0mm \itemsep 0mm 
                  \leftmargin 20mm \rightmargin 5mm
                  \labelwidth 10mm \labelsep 5mm}
\item[{\file install\_wwatch3\_tar} \hfill]
The \ws\ install program. 

\item[{\file wwatch3.[VERTAG].model.tar} \hfill] Archive file containing 
 source codes (ftn directory), programs and scripts controlling the
compiling and linking of and code management of \ws (aux and bin
directories), and sample input files (inp directory).

\item[{\file wwatch3.[VERTAG].regtests.tar} \hfill]
Archive file containing several regression test cases.

\item[{\file wwatch3.[VERTAG].cases.tar} \hfill]
Archive file containing several large tests involving real case scenarios. 

\end{list}

\noindent
The label [VERTAG] is typically a version number for the model package, which
may be followed or preceded by alpha-numeric tags describing other
characteristics of the distribution package (e.g., v4.18.beta for the beta
version 4.18 etc).

As the first step of installing \ws, these files have to be copied to a work
directory on the machine on which \ws\ will be installed. Because this
directory will be the `home' directory of \ws, it is suggested that a new
directory is created (see also warning in previous section). Furthermore
{\file install\_wwatch3\_tar} has to be made executable by typing
\command{chmod 700 install\_wwatch3\_tar} after which the installation of the
files is started by typing \command{install\_wwatch3\_tar} at your Linux/Unix
prompt.


\begin{center}
\rule[1mm]{55mm}{1.0mm} WARNING \rule[1mm]{55mm}{1.0mm} \\ 
\vspace{\baselineskip}

\parbox{120mm}{The install program will ask for a compiler to compile some
auxiliary \fortran\ codes. Unlike the actual \ws\ source code, these programs
are still written in \fortran-77. It is therefore sufficient to point toward
the generic \fortran-77 compiler on the system. The {\file install\_ww3\_tar}
script allows the user to set pre-defined choices that will point the
\fortran-77 to a generic executable {\file f77}. This may not be available on
your system, so make sure that an appropriate choice is made during the
installation process.}
\\ \vspace{\baselineskip} \rule[1mm]{55mm}{1.0mm}
WARNING \rule[1mm]{55mm}{1.0mm}
\end{center}

When {\file install\_wwatch3\_tar} is executed for the first time, it will ask
the user to identify the directory in which \ws\ will be installed. This has
to confirm that the installation directory is the current directory. Next, the
script jumps to the most crucial option, which determines if a local or generic install 
is to be performed.

The type of install deals with where to save the the traditional {\file wwatch3.env} 
file, containing the general user-dependent directory and basic FORTRAN and C compiler 
choices. The local install will save this at the same location as the package
is being installed, which is the main \ws\ directory. This results in a standalone version
that allows multiple installations (or other branches or the trunk) to co-exist
without interference. The general install means wwatch.env will be save in the user's
home directory in the form {\file \$HOME/.wwatch3.env}, and that this will be the
main or central installation in that work area. The existence of a general install
does not preclude the existence of multiple local installs, but the user has to be 
mindful of which code is being invoked when using the general install (things can
get very confusing if not kept explicitly on track).

After a choice is made for local or generic install, the script will search for
existing config files. If none is found, it will print a message that it cannot find 
the setup file, and ask some questions. The same questions are asked if a setup file is found,
except that the intention there is to confirm the existing options have not changed. In any case,
having a pre-existing setup or not, the script will give the user an opportunity to revise
defaults/existing and change if needed. The script will echo the existing options, and the 
default/existing answers or options are shown in square brackets. 

Other than the generic or local wwatch.env files, a third alternate
setup file may be specified prior to running {\file install\_wwatch3\_tar} by
setting {\code WWATCH3\_ENV} in the user environment.  The setup can be
modified by rerunning the install program, or by manually editing the setup
file. The `home' directory of \ws\ can only be changed by editing or removing
the local or generic {\file wwatch3.env} or by changing {\code WWATCH3\_ENV} in the user
environment.

\begin{center}
\rule[1mm]{55mm}{1.0mm} WARNING \rule[1mm]{55mm}{1.0mm} \\ 
\vspace{\baselineskip}

\parbox{120mm}{In case you decide to use the generic installation, you have to make sure
that the model installation directory is either {\file \${HOME}/wwatch3} or if it has
a different name, it is linked to {\file \${HOME}/wwatch3}. If this is not the case
the generic install may fail or compromise other pre-existing installations.}
\\ \vspace{\baselineskip} \rule[1mm]{55mm}{1.0mm}
WARNING \rule[1mm]{55mm}{1.0mm}
\end{center}

After the setup file is processed, the install program asks if the user wants
to continue with the installation. If the user chooses to continue, the
program will look for the archive files. If no files are found, the
archive files do not reside in the home directory, or the home directory is
erroneously defined, the installation will exit. 
Check the location of the archive files, and the `home' directory of \ws\
(see previous paragraphs).

After files to be unpacked have been identified, the program will ask if old
files should be overwritten automatically. If the user chooses `n', the
program will ask permission to overwrite each file that already exists. Files
that contain user specific information, such as compile and link options, will
never be replaced by the install program.

As the first step of the actual installation, the install program checks if
the following directories exist in the `home' directory of \ws.
 
\begin{dlist}
\dit{arc }{Archive directory.}
\dit{aux }{Raw auxiliary programs (source codes etc.).}
\dit{bin }{Executables and shell scripts for compiling and linking.}
\dit{exe }{\ws\ executables.}
\dit{ftn }{Source code and makefile.}
\dit{inp }{Input files.}
\dit{mod }{Module files.}
\dit{obj }{Object files.}
\dit{test}{Scripts with test cases.}
\dit{work}{Auxiliary work directory.}
\end{dlist}

\noindent
All these directories are generated by the install program {\file
install\_wwatch3}, except for the archive directory, which is generated by
{\file arc\_wwatch3} (see below).

Unlike previous version, where the user could choose which parts of the
package were to be installed, the current {\file install\_ww3\_tar} script
installs the etinre updated package without prompting. 

Installation of the auxiliary programs will first process source codes of
auxiliary programs, using the compiler as defined by the user in the setup
file. Note that these codes are still in fixed format \fortran-77.

\begin{flist}
\fit{w3adc.f }{\ws\ {\fortran} preprocessor.}
\fit{w3prnt.f}{Print files (source codes) including page and line numbers.}
\fit{w3list.f}{Generate a generic source code listing.}

\fit{w3split.f}{Generate spectral bulletin identifying individual wave fields
  within a spectrum from the spectral output of the point output
  post-processor (see \para\ref{sec:ww3outp}). This is a legacy code
  superseded by generating bulletins directly from {\file ww3\_outp}. It is
  retained here for historical reasons only.}

\end{flist}

\noindent
The above source codes are stored in the directory {\dir aux} and the
executables are stored in the directory {\dir bin}. A more detailed
description of these programs (including instructions on running the
executables) can be found in the documentation included in the above source
code files. After the compilation of these programs, several \unix\ shell
scripts and auxiliary files are installed in the {\dir bin} directory.

\begin{flist}
\fit{ad3              }{Script to run the preprocessor {\file w3adc}
                        and the compile script {\file comp} for a given
                        source code file.}
\fit{ad3\_test        }{Test version of {\file ad3}, showing modifications
                        to original source file. This script does not
                        compile code.}
\fit{all\_switches    }{Generates a list of all {\code w3adc} switches 
                        present in the source code files.}
\fit{arc\_wwatch3     }{Program to archive versions of \ws\ in the 
                        directory {\dir arc}.}
\fit{comp.gen         }{Generic compiler script. The actual compiler
                        script {\file comp} will be copied from this
                        script if it does not exists.}
\fit{comp.{\it xxx}   }{The compiler script {\file comp} for a specific
                        hardware-compiler combination.}
\fit{find\_switch     }{Script to find \ws\ source code files containing 
                       compiler switches (or arbitrary strings).}
\fit{install\_ww3\_svn }{Script to install \ws\ from the svn repository.}
\fit{install\_ww3\_tar }{Script to install \ws\ from tar files.}
\fit{link.gen         }{Generic linker script. Actual script is {\file
                        link}.}
\fit{link.{\it xxx}   }{The link script {\file comp} for a specific
                        hardware-compiler combination.}
\fit{list             }{Script to print source code listing using 
                        {\file w3prnt}.}
\fit{ln3              }{Script to make symbolic link of source code file
                        to work directory.}
\fit{make\_MPI        }{Script to separately compile MPI and non-MPI programs.}
\fit{make\_OMP        }{Script to separately compile OpenMP and single threaded
                        programs.}
\fit{make\_HYB        }{Script to separately compile hybrid MPI-OpenMP and
                        single threaded programs.}
\fit{make\_makefile.sh}{Script to generate the of the makefile based
                        on selections in the file {\file switch})}. 
\fit{switch.gen       }{Generic file with preprocessor switches
                        (\para\ref{sec:switches}).}
\fit{switch.{\it xxx}       }{Examples of preprocessor switches
                        provided by users or developers.}
\fit{w3\_clean        }{Script to clean up work and scratch directories
                        by removing files generated during compilation or
                        test runs.}
\fit{w3\_make         }{Script to compile and link components of \ws\
                        using a makefile.}
\fit{w3\_new          }{Script to touch correct source code files
                        to account for changes in compiler switches in
                        combination with the makefile.} 
\fit{w3\_setup        }{Script for creating/editing the \ws\ environment setup
                        file. The default setup file is
                        {\file \$\{HOME\}/.wwatch3.env}.  An alternate setup
                        file can be specified with the {\code WWATCH3\_ENV}
                        environment variable.}
\fit{w3\_source       }{Script to generate a true \fortran\ source
                        code for any of he \ws\ program elements.}
\fit{ww3\_gspl.sh     }{Script to automate use of {\file ww3\_gspl} program 
                        (see \para\ref{sub:ww3gspl}).}
\end{flist}

\noindent
The use of these scripts is explained in \para\ref{sec:comp}.  Note that the
above scripts acquire setup information from the \ws\ environment setup file
defined by {\code WWATCH3\_ENV}, or, if that is not defined, from the generic 
setup file {\file .wwatch3.env} in the home directory of the user, or the local 
setup file {\file wwatch3.env} in the directory where the wave model package is 
being installed.

\noindent
After installation in the {\dir bin} directory, several GrADS scripts are
installed in the {\dir aux} directory.

\begin{flist}
\fit{cbarn.gs         }{Semi-standard GrADS script for displaying
                        color bars.}
\fit{colorset.gs      }{Script to define colors used in shading.}
\fit{profile.gs}      {Script to display profiling data generated by {\file
                       ww3\_multi}.} 
\fit{source.gs}       {Script for composite plot of spectra and source
                       terms (2-D polar or Cartesian plots in color or in
                       black and white).}
\fit{1source.gs}      {Script to plot single source term.}
\fit{spec.gs}         {Script to plot spectra.}
\fit{spec\_ids.gen}   {Data file used by spectral / source scripts.}
\end{flist}

\noindent
This directory also has various additional tools in and documentations, see
the actual directory for its contents. These include contributed {\it Matlab} scripts,
{\file IDL} scripts and tools, and a manual on using {\file SMG type} grids.

As the final step of {\file aux} processing, some links between directories are established.

Finally, the install program lists manual modifications required by or
suggested to the user. These messages are printed only if the compile and link
system are installed. An example of an installation session using the script
{\file install\_ww3\_tar} is provided below for a case where local install was
chosen.

\pb

%\begin{myfig}{tbp}
\begin{minipage}[c]{4.5in}
{\scriptsize \begin{verbatim}
GUIDE >> tar zxvf wwatch3.beta.v4.18.tar.gz 
install_ww3_tar
guide.beta.v4.18.pdf
manual.beta.v4.18.pdf
wwatch3.beta.v4.18.model.tar
wwatch3.beta.v4.18.regtests.tar

GUIDE >> ls -l
total 354836
-rw-------. 1 wd20ha wd2    197909 Jan 14 10:11 guide.beta.v4.18.pdf
-rwx------. 1 wd20ha wd2     38670 Jan 14 10:12 install_ww3_tar
-rw-------. 1 wd20ha wd2   3545855 Jan 14 10:12 manual.beta.v4.18.pdf
-rw-------. 1 wd20ha wd2 135690240 Jan 14 10:12 wwatch3.beta.v4.18.model.tar
-rw-------. 1 wd20ha wd2 123136000 Jan 14 10:12 wwatch3.beta.v4.18.regtests.tar
-rw-------. 1 wd20ha wd2 100731957 Mar 13 15:05 wwatch3.beta.v4.18.tar.gz

GUIDE >> ./install_ww3_tar 


                  ===================================
              ------ Installing WAVEWATCH III  v.4 ------
                  =================================== 

                  Script for installing package from tar files. 
                  Requires files in same directory as script.

 Continue? [y|n] y


                  ===================================
              ------ Installing WAVEWATCH III  v.4 ------
                  ===================================
                                     from tar source 

 This installation requires a configuration file (wwatch3.env).
 The current version allows two types of env files: 
 - A local [L] wwatch3.env (Allowing multiple independent installations).
 - A generic [G] dot-file .wwatch3.env (Old-fashioned option).
 [L] Installs new, uses existing or updates env file in current directory.
 [G] Installs new, uses existing or updates env file in home directory, 
     (home is presumably /export/emc-lw-jhalves/wd20ha}). 

 Type your choice now: G

 Installing in 
   /export/emc-lw-jhalves/wd20ha/WW3_GUIDE

   OK ? [y/n] y

\end{verbatim}}
\end{minipage}

\begin{minipage}[c]{4.5in}
{\scriptsize \begin{verbatim}

 Setting up environment variables. 


   Previous setup file not found. Variables will be set to defaults. 

     (User must check to see if these setting are appropriate.)      


 Creating wwatch3.env locally (also in home if G option chosen). 
      Printer (listings)       : printer 
      FORTRAN comp. (aux only) : f77 
      C Compiler (aux only)    : cc 
      Scratch directory        : /export/emc-lw-jhalves/wd20ha/WW3_GUIDE/tmp 
      Save source code         : yes 
      Save listings            : yes 

 Update settings ? [y/n] y

 Modifying set-up 

 Type n new settings, or press ENTER to keep [current ones]: 

      Printer for listings [printer] : 
      Compiler for aux. [f77] : gfortran
      Compiler for aux. [cc] : gcc
      Scratch space [/export/emc-lw-jhalves/wd20ha/WW3_GUIDE/tmp] : 
      Save source code files (*.f)  [yes] : 
      Save listing files  [yes] : 
 
   Modified settings:
      Printer (listings)       : printer 
      FORTRAN comp. (aux only) : gfortran 
      C Compiler (aux only)    : gcc 
      Scratch directory        : /export/emc-lw-jhalves/wd20ha/WW3_GUIDE/tmp 
      Save sources             : yes 
      Save listings            : yes 

   New settings OK ? [y/n]  y

 Continue with actual implementation ? [y/n] y


[==========================SCREEN OUTPUT OMMITTED=============================]


\end{verbatim}}
\end{minipage}

\begin{minipage}[c]{4.5in}
{\scriptsize \begin{verbatim}
=============================== 
 --- Final remarks ---
 ============================================================== 

 To run the WAVEWATCH III executables and the scripts to generate 
 and update these executables from arbitrary directories, add the
 following directories to the path of your interactive shell : 

      /export/emc-lw-jhalves/wd20ha/WW3_GUIDE/bin
      /export/emc-lw-jhalves/wd20ha/WW3_GUIDE/exe

 Note that 'comp' and 'link' and 'switch' are user/machine specific.

   Several comp and link files for known compilers are found in:
   /export/emc-lw-jhalves/wd20ha/WW3_GUIDE/bin

   If you cannot find one that suits your machine/preferences, 
   create custom scripts based on the existing ones and add to bin.


                    ===============================
                  ---       End of program        --- 
                    =============================== 
 

GUIDE >> ls -l
total 3708
drwx------.  2 wd20ha wd2    4096 Mar 13 15:45 arc
drwx------.  6 wd20ha wd2    4096 Mar 13 15:45 aux
drwx------.  2 wd20ha wd2    4096 Mar 13 15:45 bin
drwx------.  2 wd20ha wd2    4096 Mar 13 15:45 exe
drwx------.  3 wd20ha wd2    4096 Mar 13 15:45 ftn
-rw-------.  1 wd20ha wd2  197909 Jan 14 10:11 guide.beta.v4.18.pdf
drwx------.  2 wd20ha wd2    4096 Mar 13 15:45 inp
lrwxrwxrwx.  1 wd20ha wd2      21 Mar 13 15:45 install_ww3_tar -> ./bin/install_ww3_tar
-rw-------.  1 wd20ha wd2 3545855 Jan 14 10:12 manual.beta.v4.18.pdf
drwx------.  2 wd20ha wd2    4096 Mar 13 15:45 mod
drwx------.  2 wd20ha wd2    4096 Mar 13 15:45 obj
drwx------. 40 wd20ha wd2    4096 Mar 13 15:45 regtests
drwx------.  2 wd20ha wd2    4096 Mar 13 15:45 tmp
drwx------.  2 wd20ha wd2    4096 Mar 13 15:45 work
-rw-------.  1 wd20ha wd2     324 Mar 13 15:44 wwatch3.env

\end{verbatim}}
\end{minipage}

