\vssub
\subsection{~About this manual}
\vssub

This is the user manual and system documentation of version \WWver\ of the
third-generation wind-wave modeling framework \ww. This underlying model has
been developed at the Marine Modeling and Analysis Branch (MMAB) of the
Environmental Modeling Center (EMC) of the National Centers for Environmental
Prediction (NCEP). It is based on WAVEWATCH~I and WAVEWATCH~II as developed at
Delft University of Technology, and NASA Goddard Space Flight Center,
respectively. \ws\ differs from its predecessors in all major aspects; i.e.,
governing equations, program structure, numerical and physical approaches.

This manual describes the governing equations, numerical approaches,
compilation, and running of \ws. The format of a combined user manual and
system documentation has been chosen, to give users the necessary background
to include new physical and numerical approaches in the framework according to
their own specifications.  This approach became more important, as \ws\
developed into a wave modeling framework. By design, a user can apply his or
her numerical or physical approaches, and thus develop a new wave model based
on the \ws\ framework. In such an approach, optimization, parallelization,
nesting, input and output service programs from the framework can be easily
shared between actual models.  Whereas this document is intended to be
complete and self-contained, this is not the case for all elements in the
system documentation. For additional system details, reference is made to the
source code, which is fully documented. Note that a best practices guide for
code development for \ws\ is now available \citep{tol:MMAB10a}.

The governing equations and numerical approaches used in this model are
described in chapters~\ref{chapt:eq} and \ref{chapt:num}. Running the model is
described in chapter~\ref{chapt:run}. Installing \ws\ is described in
chapter~\ref{chapt:impl}. Finally, a short system documentation is given in
chapter~\ref{chapt:sys}. A thorough knowledge of \ws\ can be obtained by
following chapters~\ref{chapt:eq} through \ref{chapt:impl}. A shortcut is to
first install the model (chapter~\ref{chapt:impl}), and then successively
modify input files in example runs (chapter~\ref{chapt:run}).

\vspace{\baselineskip} 
\noindent 
The present model version (\WWver) is a developmental version based on the
last official model release (version 3.14). Since the latter release the
following modifications have been made:

\begin{list}{$\bullet$}{\rightmargin 5mm \parsep 0mm \itemsep 0mm}

\item At NCEP the \ws\ source code, scripts and auxiliary tools are now
  maintained using subversion \citep{bk:CSea06}. For co-developers, a script
  is available to install \ws\ directly from the repository at NCEP, as well
  as a best best practices guide \citep[initially published
  as][]{tol:MMAB10a}.  The latter is a living document, with the most recent
  version being version~\guidever\ \citep{\guideref}.

\item Model version 4.00 was used as a set-up for the new version. The only
  main modification of this version of the code is removing the {\code XYG}
  and {\code XYG}, and replacing the choice of grid with a keyword in the code
  and in {\file mod\_def.ww3}.

\item Adding curvilinear grids from \cite{rep:RC09} (model version 4.01).

\item Adding unstructured grids from \cite{rep:Roland2008} (model version 4.02).

\item Adding new output fields (see \para\ref{sub:outpars}, model versions
  4.03 and 4.11).

\item Adding \cite{art:Aea10} source term package (see \para\ref{sec:ST4}),
  and SHOWEX bottom friction source term (see \para\ref{sec:BT4}, model
  version 4.04)

\item Adding iceberg blocking (model version 4.05).

\item Adding NetCDF output post-processing (model version 4.06).

\item Adding formal regression testing for model developers, adopted from
  previously undistributed NRL `{\file nrltest}' directory (model version
  4.07).

\item Adding GMD and nonlinear filter (see sections \ref{sec:NL3} and
  \ref{sec:NLS}, model version 4.08).

\item Adding wave system tracking (see \para\ref{sub:num_track}, model version
  4.09).

\item Adding regular grid splitting tools (see \para\ldots, model version
  4.10).

\item Adding second order UNO schemes (sections \ref{sub:num_space_trad} and
  \ref{sub:spec}, model version 4.12).

\item Adding SMC grid and rotated grid options (sections
  \ref{sub:num_space_SMC} and \ref{sub:num_space_rotagrid}, model version
  4.13).

\item BYDRZ source term package (\para\ref{sec:ST6}, model version 4.14).

\item Mud-ice interactions (\para\ref{sec:BT8}, \ref{sec:BT9},
 \ref{sec:ICE1}, \ref{sec:ICE2}, and \ref{sec:ICE3}, model version 4.15).

\item Infra-gravity wave module (\para\ref{sec:IG1}, model version 4.16).

\item Triad interactions (\para\ref{sec:TR1}, model version 4.17).

% \vspace{\baselineskip}
% \centerline{*** IN THE PIPELINE ***}
% \vspace{\baselineskip}

% \item Implicit schemes and domain decomposition for unstructured grids (model
%  version 4.18).

\item Final preparations for distribution (model version 4.18).

\end{list}

\vspace{\baselineskip} \noindent 
Up to date information on this model can be found (including bugs and bug
fixes) on the \ws\ web page, and comments, questions and suggestions should be
directed to the NCEP E-mail address

\begin{center}
http://polar.ncep.noaa.gov/waves/wavewatch
\end{center}

\noindent
or to the general \ws\ mail group list

\begin{center}
NCEP.EMC.wavewatch@NOAA.gov
\end{center}

\noindent
NCEP will redirect questions regarding contributions from outside NCEP to the
respective authors of the codes.
