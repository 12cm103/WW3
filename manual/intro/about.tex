\vssub
\subsection{~About this manual}
\vssub

This is the user manual and system documentation of version \WWver\ of the
third-generation wind-wave modeling framework \ww. This underlying model has
been developed at the Marine Modeling and Analysis Branch (MMAB) of the
Environmental Modeling Center (EMC) of the National Centers for Environmental
Prediction (NCEP). It is based on WAVEWATCH~I and WAVEWATCH~II as developed at
Delft University of Technology, and NASA Goddard Space Flight Center,
respectively. \ws\ differs from its predecessors in all major aspects; i.e.,
governing equations, program structure, numerical and physical approaches.

This manual describes the governing equations, numerical approaches,
compilation, and running of \ws. The format of a combined user manual and
system documentation has been chosen, to give users the necessary background
to include new physical and numerical approaches in the framework according to
their own specifications.  This approach became more important as \ws\
developed into a wave modeling framework. By design, a user can apply his or
her numerical or physical approaches, and thus develop a new wave model based
on the \ws\ framework. In such an approach, optimization, parallelization,
nesting, input and output service programs from the framework can be easily
shared between actual models.  Whereas this document is intended to be
complete and self-contained, this is not the case for all elements in the
system documentation. For additional system details, reference is made to the
source code, which is fully documented. Note that a best practices guide for
code development for \ws\ is now available \citep{tol:MMAB10a, tol:MMAB14b}.

The governing equations and numerical approaches used in this model are
described in chapters~\ref{chapt:eq} and \ref{chapt:num}. Running the model is
described in chapter~\ref{chapt:run}. Installing \ws\ is described in
chapter~\ref{chapt:impl}. Finally, a short system documentation is given in
chapter~\ref{chapt:sys}. A thorough knowledge of \ws\ can be obtained by
following chapters~\ref{chapt:eq} through \ref{chapt:impl}. A shortcut is to
first install the model (chapter~\ref{chapt:impl}), and then successively
modify input files in example runs (chapter~\ref{chapt:run}).

\vspace{\baselineskip} 
\noindent 
The present model version (\WWver) is a developmental version based on the
last official model release (version 4.18). Since the latter release the
following modifications have been made:

\begin{list}{$\bullet$}{\rightmargin 5mm \parsep 0mm \itemsep 0mm}

\item Preparing for next model version, adding optional instrumentation to code
      for profiling of memory use (model version 5.00). 

\item Optimization of IC3 (ice source function). Added non-dispersive variant of "turbulence under ice" ice source function to IC2. This is simpler than the existing version and requires fewer free parameters. Method is selected by the user. Added fluxes for momentum and energy associated with ice source functions. Preliminary scheme for scattering of waves by ice (model version 5.01).

\item Revisiting OpenMP parallelisms in the model. Revising previous
      OpenMP-only approach and introducing Hybrid MPI-OpenMP approach
      initiated by Farid Parpia of IBM (model version 5.02).

\item Implementing tripole grid functionality for first order scheme, and for gradient calculations (e.g. for refraction by depth/current gradients). Adding test case for tripole grid to regtests (model version 5.03).

\item Adding capability to handle cpp macros (model version 5.04)

\item Upgrade to ST6 physics (model version 5.05)

\item Adding the NCEP coupler capability (model version 5.06)

\item Adding OASIS coupler capability (model version 5.07)

\item Series of bug fix updates (model version 5.08)

\item Updates to SMC grid type (model version 5.09) 

\item Adding sea ice scattering source terms (model version 5.10)

\item Introducing namelists formats for most input files. Traditional way of providing inputs is still possible using the inp suffix (model version 5.11)

\end{list}

\vspace{\baselineskip} \noindent 
Up to date information on this model can be found (including bugs and bug
fixes) on the \ws\ web page, and comments, questions and suggestions should be
directed to the NCEP E-mail address

\begin{center}
http://polar.ncep.noaa.gov/waves/wavewatch/wavewatch.shtml
\end{center}

\noindent
or to the general \ws\ mail group list

\begin{center}
NCEP.EMC.wavewatch@NOAA.gov
\end{center}

\noindent
NCEP will redirect questions regarding contributions from outside NCEP to the
respective authors of the codes.
