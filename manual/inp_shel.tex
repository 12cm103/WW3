\begin{verbatim}
$ -------------------------------------------------------------------- $
$ WAVEWATCH III shell input file                                       $
$ -------------------------------------------------------------------- $
$ Define input to be used with flag for use and flag for definition
$ as a homogeneous field (first three only); eight input lines.
$
   F F     Water levels
   F F     Currents
   T T     Winds
   T       Ice concentrations
   F       Assimilation data : Mean parameters
   F       Assimilation data : 1-D spectra
   F       Assimilation data : 2-D spectra.
$
$ Time frame of calculations ----------------------------------------- $
$ - Starting time in yyyymmdd hhmmss format.
$ - Ending time in yyyymmdd hhmmss format.
$
   19680606 000000
   19680606 060000
$
$ Define output data ------------------------------------------------- $
$
$ Define output server mode. This is used only in the parallel version
$ of the model. To keep the input file consistent, it is always needed.
$ IOSTYP = 1 is generally recommended. IOSTYP > 2 may be more efficient
$ for massively parallel computations. Only IOSTYP = 0 requires a true
$ parallel file system like GPFS.
$
$    IOSTYP = 0 : No data server processes, direct access output from
$                 each process (requirese true parallel file system).
$             1 : No data server process. All output for each type 
$                 performed by process that performes computations too.
$             2 : Last process is reserved for all output, and does no
$                 computing.
$             3 : Multiple dedicated output processes.
$
   1
$
$ Five output types are available (see below). All output types share
$ a similar format for the first input line:
$ - first time in yyyymmdd hhmmss format, output interval (s), and 
$   last time in yyyymmdd hhmmss format (all integers).
$ Output is disabled by setting the output interval to 0.
$
$ Type 1 : Fields of mean wave parameters
$          Standard line and line with flags to activate output fields
$          as defined in section 2.4 of the manual. The second line is
$          not supplied if no output is requested.
$                               The raw data file is out_grd.ww3, 
$                               see w3iogo.ftn for additional doc.
$          
$
   19680606 000000   3600  19680608 000000
     T T T T T  T T T T T  T T T T T  T T T T T  T T T T T
     T T T T T  T
$
$ Type 2 : Point output
$          Standard line and a number of lines identifying the 
$          longitude, latitude and name (C*10) of output points.
$          The list is closed by defining a point with the name
$          'STOPSTRING'. No point info read if no point output is
$          requested (i.e., no 'STOPSTRING' needed).
$          Example for spherical grid.
$                               The raw data file is out_pnt.ww3, 
$                               see w3iogo.ftn for additional doc.
$
$   NOTE : Spaces may be included in the name, but this is not
$          advised, because it will break the GrADS utility to 
$          plots spectra and source terms, and will make it more
$          diffucult to use point names in data files.
$
   19680606 000000    900  19680608 000000
$
    -0.25 -0.25 'Land      '
     0.0   0.0  'Point_1   '
     2.0   1.0  'Point_2   '
     1.8   2.2  'Point_3   '
     2.1   0.9  'Point_4   '
     5.0   5.0  'Outside   '
$
     0.0   0.0  'STOPSTRING'
$
$ Type 3 : Output along  track.
$          Flag for formatted input file.
$                         The data files are track_i.ww3 and
$                         track_o.ww3, see w3iotr.ftn for ad. doc.
$
   19680606 000000   1800  19680606 013000
     T
$
$ Type 4 : Restart files (no additional data required).
$                               The data file is restartN.ww3, see
$                               w3iors.ftn for additional doc.
$
   19680606 030000      1  19680606 030000
$
$ Type 5 : Boundary data (no additional data required).
$                               The data file is nestN.ww3, see
$                               w3iobp.ftn for additional doc.
$
   19680606 000000   3600  20010102 000000
$
$ Type 6 : Separated wave field data (dummy for now).
$          First, last step IX and IY, flag for formatted file
$
   19680606 000000   3600  20010102 000000
      0 999 1 0 999 1 T
$
$ Homogeneous field data --------------------------------------------- $
$ Homogeneous fields can be defined by a list of lines containing an ID
$ string 'LEV' 'CUR' 'WND', date and time information (yyyymmdd
$ hhmmss), value (S.I. units), direction (current and wind, oceanogr.
$ convention degrees)) and air-sea temparature difference (degrees C).
$ 'STP' is mandatory stop string.
$ Also defined here are the speed with which the grid is moved
$ continuously, ID string 'MOV', parameters as for 'CUR'.
$
   'LEV' 19680606 010000    1.0
   'CUR' 19680606 073125    2.0    25.
   'WND' 19680606 000000   20.    145.    2.0
   'MOV' 19680606 013000    4.0    25.
   'STP'
$
$ -------------------------------------------------------------------- $
$ End of input file                                                    $
$ -------------------------------------------------------------------- $
\end{verbatim}
