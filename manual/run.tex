
\section{~Running the wave model} \label{chapt:run}
\newcounters
\vssub
\subsection{~Program design}
\vssub

The core of \ws\ is the wave model subroutine. The wave model routine can be
called by either a stand-alone program shell or any other program that
requires dynamically updated wave data. Two such programs are provided with
the \ws\ release. Auxiliary programs include a grid preprocessor, a program to
generate artificial initial conditions, a generic program shell (and a
corresponding input pre-processor) and output post-processors.  In the
discussion of the model below, file names will be identified by the {\file
file} type font, the contents of a file by the {\code code} type font and {\sc
fortran} program elements by the {\F fortran} type font.

The main wave model routine is {\F w3wave}. Data files are identified with the
file extension {\file .ww3}, except in the multi-grid wave model {\file
ww3\_multi}, where the file extension identifies an individual grid. For
simplicity, the file extension {\file .ww3} will be used throughout this
chapter.  A relational diagram including the basic data flow is presented in
figure~\ref{fig:elements}.

The grid preprocessor writes a model definition file {\file mod\_def.ww3} with
bottom and obstruction information and parameter values defining the physical
and numerical approaches. The wave model requires initial conditions,
consisting of a restart file {\file restart.ww3}, written by either the wave
model itself, or by the initial conditions program. If this file is not
available, the wave model will be initialized automatically, depending on the
ability of the model to start from calm conditions. If linear growth or
spectral seeding is switch on, the model will start from calm conditions ($H_s
= 0$), otherwise the initial conditions will consist of a parametric
fetch-limited spectrum based on the initial wind field (see the corresponding
option in the initial conditions program).  The wave model routine ({\F
w3wave}) optionally generates up to 9 restart files {\file
restart{\em{n}}.ww3}, where {\file{\em{n}}} represents a single digit integer
number. The wave model also optionally reads boundary conditions from the file
{\file nest.ww3} and generates boundary conditions for consecutive runs in
{\file nest{\em{n}}.ww3}. The model furthermore dumps raw data to the output
files {\file out\_grd.ww3 }, {\file out\_pnt.ww3}, {\file track\_o.ww3} and
{\file partition.ww3} (gridded mean wave parameters, spectra at locations,
spectra along tracks, and partitioned wave data, respectively). The tracks
along which spectra are to be presented is defined in the file {\file
track\_i.ww3}. Note that the wave model does not write to standard output,
because this would be inconvenient if \ws\ is part of an integrated
model. Instead, it maintains its own log file {\file log.ww3} and optionally a
test output files {\file test.ww3} for a shared memory version of the model,
or {\file test{\em{nnn}}.ww3} for distributed memory versions, where {\em nnn}
is the processor number starting with 1.  Finally, six output post-processors
are available (binary post-processing of raw gridded fields, point output and
track output files; GRIB packing of gridded data; post-processing for later
GrADS graphical processing of gridded and spectral data). A more detailed
description of all program elements and there input files is given below. Note
that the source codes of each routine are fully documented. This documentation
is an additional source of information about \ws.

\setlength{\unitlength}{0.1mm}
\begin{figure}

\begin{picture}(1370,1080)(0,-1080)

\put(  50,  -80){\dashbox{5}(280,80)[c]{{\file grid data}}}
\put( 190,  -80){\vector(0,-1){60}}

\put(   0, -220){\framebox(380,80)[c]{grid preprocessor}}
\put( 190, -220){\vector(0,-1){60}}

\put(  50, -360){\dashbox{5}(280,80)[c]{{\file mod\_def.ww3}}}
\put( 190, -360){\vector(0,-1){380}}
\put( 190, -460){\vector(1,0){60}}
\put( 330, -320){\line(1,0){560}}
\put( 890, -320){\vector(1,1){100}}
\put( 890, -320){\line(1,-1){80}}
\put( 970, -400){\vector(0,-1){160}}

\put( 250, -500){\framebox(280,80)[c]{initial cond.}}
\put( 530, -460){\vector(1,0){60}}

\put( 590, -500){\dashbox{5}(280,80)[c]{{\file restart.ww3}}}
\put( 870, -500){\vector(1,-1){60}}

\put( 590, -640){\makebox(280,80)[c]{{\file restart.ww3}}}
\put( 590, -680){\makebox(280,80)[c]{{\file nest.ww3}}}
\put( 590, -680){\dashbox{5}(280,120)[c]{ }}

\put( 930, -640){\dashbox{10}(280,80)[c]{{\code wave model}}}
\put( 930, -600){\vector(-1,0){60}}
\put( 870, -600){\vector(1,0){60}}

\put( 590, -860){\makebox(280,80)[c]{{\file partition.ww3}}}
\put( 590, -780){\makebox(280,80)[c]{{\file out\_pnt.ww3}}}
\put( 590, -820){\makebox(280,80)[c]{{\file out\_grd.ww3}}}
\put( 590, -860){\dashbox{5}(280,170)[c]{ }}
\put( 970, -780){\vector(-1,0){100}}
\put( 590, -780){\vector(-1,0){60}}

\put( 970, -640){\vector(0,-1){240}}
\put( 830, -960){\makebox(280,80)[c]{{\file log.ww3}}}
\put( 830,-1000){\makebox(280,80)[c]{{\file test.ww3}}}
\put( 830,-1000){\dashbox{5}(280,120)[c]{ }}

\put(1170, -640){\vector(0,-1){100}}
\put(1170, -740){\vector(0,1){100}}
\put(1030, -820){\makebox(280,80)[c]{{\file track\_i.ww3}}}
\put(1030, -860){\makebox(280,80)[c]{{\file track\_o.ww3}}}
\put(1030, -860){\dashbox{5}(280,120)[c]{ }}

\put( 150, -820){\makebox(380,80)[c]{output}}
\put( 150, -860){\makebox(380,80)[c]{postprocessing}}
\put( 150, -860){\framebox(380,120)[c]{ }}

\put(1020, -560){\line(0,1){280}}
\put(1020, -280){\line(1,0){320}}
\put(1210, -620){\line(1,0){130}}
\put(1340, -620){\line(0,1){340}}
\put(1020, -370){\makebox(320,80)[c]{program}}
\put(1020, -410){\makebox(320,80)[c]{shell}}
\put(1020, -450){\makebox(320,80)[c]{or}}
\put(1020, -490){\makebox(320,80)[c]{integrated}}
\put(1020, -540){\makebox(320,80)[c]{program}}

\put(1040,  -80){\dashbox{5}(280,80)[0,0]{{\file input files}}}
\put(1170,  -80){\vector(0,-1){60}}

\put( 990, -220){\framebox(380,80)[c]{input preprocessor}}
\put(1170, -220){\vector(0,-1){60}}

{\scriptsize
\put(  50,-1010){\dashbox{5}(150,40)[0,0]{{\file file}}}
\put( 240,-1010){\dashbox{10}(150,40)[c]{{\code subrout.}}}
\put( 430,-1010){\framebox(150,40)[c]{program}}
\put( 140,-1050){\vector(1,0){60}}
\put( 220,-1070){\makebox(400,40)[l]{data transfer by file}}
\put(   0,-1090){\framebox(630,150){ }}                 }

\end{picture}

\caption{Basic program elements and data flow.}
\label{fig:run_elements}

\botline

\end{figure}


Files specific to \ws\ are opened by name within the program. The unit
numbers, however, have to be defined by the user\footnote{~Except for {\file
ww3\_multi}.}, guaranteeing the largest possible flexibility for
implementation in integrated models.

Next to the wave model subroutine, an initialization routine and an interface
routine for data assimilation are provided. The latter routine is intended to
be run side by side with the wave model routine. The routine includes a
generic interface that provides all necessary model components to perform full
spectral data assimilation. This routine is integrated into the generic wave
model shell, which is set up to perform time step managements for a wave model
with or without data assimilation. The shell also provides a simple yet
flexible way to provide the data assimilation scheme with various types of
data. Data assimilation has not yet been included in the multi-grid wave model
shell.


% -------------------------------------------------------------
\vssub
\subsection{~The wave model routines} \label{sec:core}
\vssub

As discussed above, the actual wave model is a subroutine. To run the model, a
program shell is needed. \ws\ is provided with a simple stand-alone shell as
will be discussed in \para\ref{sec:shell}, and with a more complex multi-grid
model shell as will be discussed in \para\ref{sec:multi}. The present section
concentrates on the wave model subroutines.

The wave model initialization routine {\F w3init} performs model
initialization. This includes setting up part of the I/O system by defining
unit numbers, initializing internal time management, processing the model
definition file ({\file mod\_def.ww3}), processing initial conditions ({\file
restart.ww3}), preparing model output, and calculating grid-dependent
parameters. If the model is compiled for an \mpi\ environment, all necessary
communication for both calculations and output are determined and initialized
(the model uses persistent \mpi\ communication throughout).

The wave model routine {\F w3wave} can be called any number of times to
propagate the wave field in time after the initialization has taken
place. After some initial checks, the subroutine interpolates winds and
currents, updates ice concentrations and water levels, propagates the wave
field, and applies the selected source terms for a number of time steps. The
internal time step is defined by the interval for which the calculations are
to be performed, and by the requested output times. At the end of the
calculations, the routine provides the calling program with the requested
fields of wave data. A documentation of the interface of {\F w3wave} can be
found in the source code ({\file w3wavemd.ftn}).

\begin{figure}
{\small \begin{verbatim}
                                    |    input    |     output    |
                                    |-------------|---------------|
  step | pass |    date      time   | b w l c i d | g p t r b f c |
-------|------|---------------------|-------------|---------------|
    0  |   1  | 1968/06/06 00:00:00 |   F         | X X           |
    8  |   1  |            02:00:00 |             |   X           |
   12  |   1  |            03:00:00 |             | X             |
   16  |   1  |            04:00:00 |             |   X           |
   24  |   1  |            06:00:00 |   X         | X X           |
   32  |   2  |            08:00:00 |             |   X           |
   36  |   2  |            09:00:00 |             | L             |
   40  |   2  |            10:00:00 |             |   X           |
   48  |   2  |            12:00:00 |   X       X |   L   L       |
-------+------+---------------------+-------------+---------------+ 
\end{verbatim} }
\caption{Example action table from file {\file log.ww3}.} \label{fig:log}
\botline
\end{figure}
 

Apart from the raw data files as described above, the program maintains a log
file {\file log.ww3}. This file is opened by {\F w3init} (contained in {\F
w3wave} in {\file w3wavemd.ftn}), which writes some self-explanatory header
information to this file. Each consecutive call to {\F w3wave} adds several
lines to an `action table' in this log file as is shown in
Fig.~\ref{fig:log}. The column identified as `step' shows the discrete time
step considered. The column identified as `pass' identifies the sequence
number of the call to {\F w3wave}; i.e., 3 identifies that this action took
place in the third call to {\F w3wave}. The third column shows the ending time
of the time step. In the input and output columns the corresponding actions of
the model are shown. An {\tt X} identifies that the input has been updated, or
that the output has been performed. An {\tt F} indicates a first field read,
and an {\tt L} identifies the last output. The seven input columns identify
boundary conditions ({\tt b}), wind fields ({\tt w}), water levels ({\tt l}),
current fields ({\tt c}), ice concentrations ({\tt i}), and data for
assimilation ({\tt d}), respectively. Note that data assimilation takes place
at the end of the time step after the wave routine call. The six output
columns identify gridded output ({\tt g}), point output ({\tt p}), output
along tracks ({\tt t}), restart files ({\tt r}), boundary data ({\tt b}), and
partitioned spectral data {\tt f}, respectively. 


% -------------------------------------------------------------
\vssub
\subsection{~The data assimilation interface} \label{sec:das}
\vssub

As discussed above, the wave model subroutine is supplemented with a data
assimilation interface routine ({\F w3wdas} in {\file w3wdasmd.ftn}). This
routine is integrated in the stand-alone shell (see \para\ref{sec:shell}) to
provide time step management of a combined wave model / data assimilation
scheme. In this a fairly simple approach is assumed where data assimilation is
performed at selected times, while the wave model marches forward in time. In
the setup of the shell, the data assimilation is performed after the model has
reached the target time, but has not yet produced output. After the data
assimilation is performed, the wave model routine is called again only to
generate output as requested. Thus, the wave model output for a given time
will include the effects of data assimilation for that specific target time.

The generic program shell also processes several types of data to be
assimilated, and passes it on to the data assimilation interface routine. All
data needs to be preprocessed using the wave model input preprocessor (see
\para\ref{sec:prep}), and will be recognized by the generic shell by file
name. Presently, up to three different data files can be used. Tentatively,
these could be mean wave parameters, one dimensional spectral data, and two
dimensional spectral data, respectively. This is, however, not hardwired to
the model and in fact needs to be defined by the user.

Presently, no data assimilation packages are available. User supplied data
assimilation schemes can be included in the wave model using the interface
routine ({\F w3wdas} in {\file w3wdasmd.ftn}), the documentation of which
should be sufficient for the necessary programming. Details on how to add user
supplied software to the \ws\ compilation system can be found in the following
chapter. NCEP is presently working on wave data assimilation techniques, but
presently has no plans to distribute wave data assimilation software.

\pb

% -------------------------------------------------------------
\vssub
\subsection{~Auxiliary programs} \label{sec:auxprog}
\vsssub
\subsubsection{General concepts}
\vsssub

All auxiliary programs presented here with the exception of the track output
post-processor read input from a pre-defined input file. The first character
on the first line of the input file will be considered to be the comment
character, identifying comment lines in the input file. This comment character
has to appear on the first position of input lines to be effective. In all
examples in the following sections lines starting with '{\tt \$}' therefore
only contain comment. The programs furthermore all write formatted output to
the standard output unit.

In the following sections, all available auxiliary programs are described
using an example input file with all options included (partially as
comment). These files are identical to the distributed example input
files. The sections furthermore show the name of the executable program, the
program name (as appears in the program statement), the source code file and
input and output files and there unit numbers (in brackets behind the file
name). Input and output files marked with \opt are optional. The intermediate
files mentioned below are all {\F unformatted}, and are not described in
detail here. Each file is written and read by a single routine, to which
reference is made for additional documentation.

\begin{list}{}{\itemsep 0mm \parsep 0mm \leftmargin 40mm \labelwidth 30mm}
\item[{mod\_def.ww3} \hfill] Subroutine {\F w3iogr} ({\file w3iogrmd.ftn}).
\item[{out\_grd.ww3} \hfill] Subroutine {\F w3iogo} ({\file w3iogomd.ftn}).
\item[{out\_pnt.ww3} \hfill] Subroutine {\F w3iopo} ({\file w3iopomd.ftn}).
\item[{track\_o.ww3} \hfill] Subroutine {\F w3iotr} ({\file w3iotrmd.ftn}).
\item[{restart.ww3}  \hfill] Subroutine {\F w3iors} ({\file w3iorsmd.ftn}).
\item[{nest.ww3}     \hfill] Subroutine {\F w3iobc} ({\file w3iobcmd.ftn}).
\item[{partition.ww3}\hfill] Subroutine {\F w3iosf} ({\file w3iosfmd.ftn}).
\end{list}

\noindent
Preprocessing and compilation of the programs is discussed in the following
two chapters. Examples of test runs of the model are provided with the source
code.


\pb

% -------------------------------------------------------------
\vsssub
\subsubsection{The grid preprocessor} \label{sub:ww3grid}
\vsssub

\proddefH{ww3\_grid}{w3grid}{ww3\_grid.ftn}
\proddeff{Input}{ww3\_grid.inp}{Formatted input file for program.}{10}
\proddefa{'grid file' \opt}{File with bottom depths.}{user}
\proddefa{'obstr. file' \opt}{File with sub-grid obstructions. }{user}
\proddefa{'mask file' \opt}{File with grid mask. }{user}
\proddeff{Output}{standard out}{Formatted output of program.}{6}
\proddefa{mod\_def.ww3}{Model definition file in \ws\ format.}{20}
\proddefa{mask.ww3 \opt}{Land-sea mask file (switch {\F o2}a).}{20}
\proddeff{Scratch}{ww3\_grid.scratch}{Formatted scratch file.}{90}

\vspace{\baselineskip} \noindent
Note that bottom and obstruction data may be in same file.

\inpfile{ww3_grid.tex}


\pb
% -------------------------------------------------------------
\vsssub
\subsubsection{The initial conditions program} 
\vsssub

\proddefH{ww3\_strt}{w3strt}{ww3\_strt.ftn}
\proddeff{Input}{ww3\_strt.inp}{Formatted input file for program.}{10}
\proddefa{mod\_def.ww3}{Model definition file.}{20}
\proddeff{Output}{standard out}{Formatted output of program.}{6}
\proddefa{restart.ww3}{Restart file in \ws\ format.}{20}

\inpfile{ww3_strt.tex}


\pb
% -------------------------------------------------------------
\vsssub
\subsubsection{The field preprocessor for the generic shell}
\label{sec:prep}
\vsssub

\proddefH{ww3\_prep}{w3prep}{ww3\_prep.ftn}
\proddeff{Input}{ww3\_prep.inp}{Formatted input file for program.}{10}
\proddefa{mod\_def.ww3}{Model definition file.}{11}
\proddefa{'user input'\opt}{See example below.}{user}
\proddeff{Output}{standard out}{Formatted output of program.}{6}
\proddefa{level.ww3\opt}{Water levels file.}{12}
\proddefa{current.ww3\opt}{Current fields file.}{12}
\proddefa{wind.ww3\opt}{Wind fields file.}{12}
\proddefa{ice.ww3\opt}{Ice fields file.}{12}
\proddefa{data0.ww3\opt}{Assimilation data (`mean').}{12}
\proddefa{data1.ww3\opt}{Assimilation data (`1-D spectra').}{12}
\proddefa{data2.ww3\opt}{Assimilation data (`2-D spectra').}{12}

\inpfile{ww3_prep.tex}

\vspace{\baselineskip} 
\noindent 
Note that the optional output files are specific to {\file ww3\_shel} and
{\file ww3\_multi}, but are not processed by the actual wave model
routines. These files are consequently not needed if the wave model routines
are used in a different shell or in an integrated program. However, the
routines reading and writing these files are system-independent and could
therefore be used in customized applications of the basic wave model. The
reading and writing of these files is performed by the subroutine {\F w3fldg}
({\file w3fldsmd.ftn}). For additional documentation and file formats
reference if made to this routine.


\pb
% -------------------------------------------------------------
\vsssub
\subsubsection{The generic shell} \label{sec:shell}
\vsssub

\proddefH{ww3\_shel}{w3shel}{ww3\_shel.ftn}
\proddeff{Input}{ww3\_shel.inp}{Formatted input file for program.}{10}
\proddefa{mod\_def.ww3}{Model definition file.}{30}
\proddefa{restart.ww3}{Restart file.}{30}
\proddefa{nest.ww3\opt}{Boundary conditions file.}{33}
\proddefa{level.ww3\opt}{Water levels file.}{11}
\proddefa{current.ww3\opt}{Current fields file.}{12}
\proddefa{wind.ww3\opt}{Wind fields file.}{13}
\proddefa{ice.ww3\opt}{Ice fields file.}{14}
\proddefa{data0.ww3\opt}{Assimilation data.}{15}
\proddefa{data1.ww3\opt}{Assimilation data.}{16}
\proddefa{data2.ww3\opt}{Assimilation data.}{17}
\proddefa{track\_i.ww3\opt}{Output track information.}{22}
\proddeff{Output}{standard out}{Formatted output of program.}{6}
\proddefa{log.ww3}{Output log of wave model (see \para\ref{sec:core}).}{20}
\proddefa{test.ww3\opt}{Test output of wave model.}{6/21}
\proddefa{restart{\sl{n}}.ww3\opt}{Restart file(s).}{30}
\proddefa{nest{\sl{n}}.ww3\opt}{Nesting file(s).}{34-42}
\proddefa{out\_grd.ww3\opt}{Raw output of gridded fields.}{31}
\proddefa{out\_pnt.ww3\opt}{Raw output of spectra.}{32}
\proddefa{track\_o.ww3\opt}{Raw output of spectra along tracks.}{23}
\proddeff{Scratch}{ww3\_shel.scratch}{Formatted scratch file.}{90}

\inpfile{ww3_shel.tex}


\pb
% -------------------------------------------------------------
\vsssub
\subsubsection{The multi-grid shell} \label{sec:multi}
\vsssub

\proddefH{ww3\_multi}{w3mlti}{ww3\_multi.ftn}
\proddeff{Input}{ww3\_multi.inp}{Input file for multi-grid wave model shel.}{8}
\proddeff{Output}{standard out}{Formatted output of program.}{6}
\proddefa{log.mww3}{Output log of wave model driver.}{9}
\proddefa{test.mww3\opt}{Test output of wave model.}{auto}

\vspace{\baselineskip}
\noindent
This wave model program requires and produces a plethora of input and output
files consistent with those of {\file ww3\_shel} in \para\ref{sec:shell},
where file extensions {\file .ww3} are replaced by an identifier for a
specific grid. Note that all files are opened by name, and that the unit
number assignment is dynamic and automatic.

\inpfile{ww3_multi.tex}

\pb
% -------------------------------------------------------------
\vsssub
\subsubsection{Gridded output post-processor} \label{sec:post_f}
\vsssub

\proddefH{ww3\_outf}{w3outf}{ww3\_outf.ftn}
\proddeff{Input}{ww3\_outf.inp}{Input file for gridded output
post-processor.}{10}
\proddefa{mod\_def.ww3}{Model definition file.}{20}
\proddefa{out\_grd.ww3}{Raw gridded output data.}{20}
\proddeff{Output}{standard out}{Formatted output of program.}{6}
\proddefa{\ldots \opt}{Transfer file.}{50}

\inpfile{ww3_outf.tex}

\vspace{\baselineskip} 
\noindent
The extension of the file name of transfer files for {\F itype = 3} identifies
the content of the file. The file extension for each data type is given in
Table~\ref{tab:fields} on page~\pageref{tab:fields}.


\pb
% -------------------------------------------------------------
\vsssub
\subsubsection{Point output post-processor} \label{sec:post_p}
\vsssub

\proddefH{ww3\_outp}{w3outp}{ww3\_outp.ftn}
\proddeff{Input}{ww3\_outp.inp}{Input file for point output
post-processor.}{10}
\proddefa{mod\_def.ww3}{Model definition file.}{20}
\proddefa{out\_pnt.ww3}{Raw point output data.}{20}
\proddeff{Output}{standard out}{Formatted output of program.}{6}
\proddefa{tab{\sl{nn}}.ww3 \opt}{Table of mean parameters where
{\file{\sl{nn}}} is a two-digit integer.}{\sl nn}
\proddefa{\ldots \opt}{Transfer file.}{user}

\inpfile{ww3_outp.tex}

\vspace{\baselineskip} 
\noindent 
The spectral data transfer file generated with {\F itype = 1} and {\F otype =
3} can be converted into a spectral bulletin using {\file w3split} (see
section~\ref{sec:install}). This program reads the following five records from
standard input (no comment lines allowed) :

\begin{list}{$\bullet$}{\itemsep 0mm \parsep 0mm}
\item Name of output location.
\item Identifier for run to be used in table.
\item Name of input file.
\item Logical identifying UNFORMATTED input file.
\item Name of output file.
\end{list}

\noindent
All above strings are read as characters using free format, and therefore need
to be enclosed in quotes.


\pb
% -------------------------------------------------------------
\vsssub
\subsubsection{Track output post-processor} \label{sec:post_t}
\vsssub

\proddefH{ww3\_trck}{w3trck}{ww3\_trck.ftn}
\proddeff{Input}{track\_o.ww3}{Raw track output data.}{11}
\proddeff{Output}{standard out}{Formatted output of program.}{6}
\proddefa{track.ww3}{Formatted data file.}{51}

\vspace{\baselineskip} 
\noindent
This post-processor does not require a formatted input file with program
commands. It will simply convert the entire unformatted file to an integer
compressed formatted file. The file contains the following header records :

\begin{list}{$\bullet$}{\itemsep 0mm \parsep 0mm}
\item File identifier (character string of length 34).
\item Number of frequencies and directions, first direction and directional
      increment (radians, oceanographic convention).
\item Radian frequencies of each frequency bin.
\item Corresponding directional bin size times frequency bin size to obtain
      discrete energy per bin.
\end{list}

\noindent
For each output point the following records are printed :
\begin{list}{$\bullet$}{\itemsep 0mm \parsep 0mm}
\item Date and time in {\tt yyyymmdd hhmmss} format, longitude and latitude in
      degrees, and a status identifier `{\F ice}', `{\F lnd}' or `{\F
      sea}'. The following two records are written only for sea points.
\item Water depth in meters, current and wind u and v components in meters per
      second, friction velocity in meters per second, air-sea temperature
      difference in degrees centigrade and scale factor for spectrum.
\item The entire spectrum in integer packed format (can be read using free
      format).
\end{list}


\pb
% -------------------------------------------------------------
\vsssub
\subsubsection{GRIB output post-processor} \label{sec:post_g}
\vsssub

\proddefH{ww3\_grib}{w3grib}{ww3\_grib.ftn}
\proddeff{Input}{ww3\_grib.inp}{Input file for gridded output
post-processor.}{10}
\proddefa{mod\_def.ww3}{Model definition file.}{20}
\proddefa{out\_grd.ww3}{Raw gridded output data.}{20}
\proddeff{Output}{standard out}{Formatted output of program.}{6}
\proddefa{gribfile}{GRIB file.}{50}

\inpfile{ww3_grib.tex}

\vspace{\baselineskip} 
\noindent
This post-processor packs fields of mean wave parameters in GRIB format, using
GRIB version II and \ncep's w3 and bacio library routines, or in GRIB2, using
NCEPS's operational package. Additional packing data can be found in
Table~\ref{tab:fields} on page \pageref{tab:fields}.

The GRIB packing is performed using the \ncep's GRIB tables as described in
\cite{rep:GRIB}. Because the w3 and bacio routine are not fully portable, they
are not supplied with the code. The user will have to provide corresponding
routines. It is suggested that such routines are activated with additional
\ws\ switches in the mandatory switch group containing the `{\F nogrb}'
switch, as if presently the case with the \ncep\ routines.  The GRIB2 packing
is performed according to \cite{rep:GRIB2}, and is performed with NCEP's
standard operational packages.

Table~\ref{tab:fields} shows the {\F kpds(5)} data values for GRIB
packing. For the partitioned data, the first number identifies the wind sea,
the second number identifies swell. Most data are packed as surface data ({\F
kpds(6) = 0}). For the partitioned swell fields, however, consecutive fields
are packed at consecutive levels, with the level type indicator set to ({\F
kpds(6) = 241}). {\F kpds(7)} identifies the actual level or swell field
number.

Table~\ref{tab:fields} shows several {\F kpds} data values for GRIB2
packing. The first number in the table represents {\F listsec0(2)}, which
identifies the discipline type (e.g., oceanography, meteorology, etc.)  The
second number represents {\F kpds(1)}, which identifies the parameter category
(e.g., waves, circulation, ice, etc.) within the discipline type.  The third
number represents {\F kpds(2)}, which identifies the actual parameter.  For
the partitioned data, A/B means A for wind sea and B for swell.  Additionally
{\F kpds(10) = 0} for surface data, and {\F kpds(10) = 241 } to pack
consecutive swell fields at consecutive levels. {\F kpds(12)} identifies the
actual level or swell field number.

Although the above input file contains flags for all 31 output fields of \ws,
not all fields can be packed in GRIB. If a parameter is chosen for which GRIB
packing is not available, a message will be printed to standard
output. Table~\ref{tab:fields} shows which parameter can be packed in GRIB.
Note that at \ncep\ the conversions from GRIB to GRIB2 coincided with the
introduction of partitioned wave model output. This required some duplicate
definitions in GRIB and some apparent inconsistencies between GRIB and GRIB2
packing.


\pb
% -------------------------------------------------------------
\vsssub
\subsubsection{Gridded output post-processor for GrADS} \label{sec:post_gf}
\vsssub

\proddefH{gx\_outf}{gxoutf}{gx\_outf.ftn}
\proddeff{Input}{gx\_outf.inp}{Input file for gridded output
post-processor.}{10}
\proddefa{mod\_def.ww3}{Model definition file.}{20}
\proddefa{out\_grd.ww3}{Raw gridded output data.}{20}
\proddeff{Output}{standard out}{Formatted output of program.}{6}
\proddefa{ww3.grads}{GrADS data file.}{50}
\proddefa{ww3.ctl}{GrADS control file.}{51}

\inpfile{gx_outf.tex}

\vspace{\baselineskip} 
\noindent 
This post-processor generates input files with gridded model parameters for
the Grid Analysis and Display System \citep[GrADS,][]{man:GrADS}. This
graphical software can be obtained from http://www.iges.org/grads. Although
GrADS can also work with GRIB files, the present preprocessor is preferable,
as the data file also gives access to a land-sea-ice map.


\pb
% -------------------------------------------------------------
\vsssub
\subsubsection{Point output post-processor for GrADS} \label{sec:post_gp}
\vsssub

\proddefH{gx\_outp}{gxoutp}{gx\_outp.ftn}
\proddeff{Input}{gx\_outp.inp}{Input file for point output
post-processor.}{10}
\proddefa{mod\_def.ww3}{Model definition file.}{20}
\proddefa{out\_pnt.ww3}{Raw point output data.}{20}
\proddeff{Output}{standard out}{Formatted output of program.}{6}
\proddefa{ww3.spec.grads}{GrADS data file with spectra and source terms.}{30}
\proddefa{ww3.mean.grads}{File with mean wave parameters.}{31}
\proddefa{ww3.spec.ctl}{GrADS control file.}{32}

\inpfile{gx_outp.tex}

\vspace{\baselineskip} 
\noindent
This post-processor is intended to generate data files with which GrADS (see
previous section) can plot polar plots of spectra and source terms. To achieve
this, spectra and source terms are store as "longitude-latitude" grids. For
each output point a different name is generated for the data, typically {\F
loc{\it nnn}}. When the data file is loaded in GrADS, the variable {\F loc001}
will contain a spectral grid for the first requested output point at level 1,
the input source term at level 2, etc. For the second output point the data is
stored in {\F loc002} etc. The actual output point names are passed to GrADS
through the control file {\file ww3.spec.ctl}. Wave heights and environmental
data are obtained from {\file ww3.mean.grads} The user, however, need not be
aware of the details of the GrADS data files and data storage. The GrADS
scripts {\file spec.gs}, {\file source.gs} and {\file 1source.gs} are provided
to automatically generate spectral plots from the output files of this
post-processor.

Note: for the GrADS scripts to work properly, the names of the output points
should not contain spaces.

\pb
\label{pg:tab_fields}

% tab:fields

\begin{table} \begin{center}
\begin{tabular}{|c|c|c|c|c|} \hline
field & description &  file     & GRIB1 & GRIB2  \\
     &             &  extension & data  & data   \\ \hline \hline
 1 & depth                           & {\file .dpt} &  --  &    --    \\
 2 & mean current components         & {\file .cur} &  --  &    --    \\
 3 & wind speed                      & {\file .wnd} &  32  &  0,2,1   \\
   &  wind direction                 &              &  31  &  0,2,0   \\
   &  wind $u$                       &              &  33  &  0,2,2   \\
   &  wind $v$                       &              &  34  &  0,2,3   \\
 4 & air-sea temp. dif.              & {\file .dt}  &  --  &    --    \\
 5 & friction velocity comp.         & {\file .ust} &  --  &    --    \\
 6 & wave height $H_s$               & {\file .hs}  & 100  &  10,0,3  \\
 7 & mean wave length                & {\file .l}   &  --  &    --    \\
 8 & mean wave period $T_m$          & {\file .t}   & 103  &    --    \\
 9 & mean wave direction $\theta_m$  & {\file .dir} & 101  &    --    \\
10 & directional spread $\sigma$     & {\file .spr} &  --  &    --    \\
11 & peak period $T_p$               & {\file .fp}  & 108  &  10,0,11 \\
12 & peak direction $\theta_p$       & {\file .dp}  & 107  &  10,0,10 \\
13 & wind sea period $T_w$           & {\file .fpl} & 110  &    --    \\
14 & wind sea direction $\theta_w$   & {\file .dpl} & 109  &    --    \\
15 & $H_s$ of partition              & {\file .phs} & 102,105 & 10,0,5/8 \\
16 & $T_p$ of partition              & {\file .ptp} & 103,106 & 10,0,6/9\\
17 & $L_p$ of partition              & {\file .plp} &  --  &    --    \\
18 & $\theta_m$ of partition         & {\file .pth} & 101,104 & 10,0,4/7 \\
19 & $\sigma$ of partition           & {\file .psi} &  --  &    --    \\
20 & wind sea fraction of part.      & {\file .pws} &  --  &    --    \\
21 & total wind sea fraction         & {\file .wsf} &  --  &    --    \\
22 & number of partitions            & {\file .pnr} &  --  &    --    \\
23 & average time step               & {\file .dtd} &  --  &    --    \\
24 & cut-off frequency $f_c$         & {\file .fc}  &  --  &    --    \\
25 & ice coverage                    & {\file .ice} &  91  &  10,2,0  \\
26 & water level                     & {\file .wlv} &  --  &  10,3,1  \\
27 & near-bottom amplitude           & {\file .abr} &  --  &    --    \\
28 & near-bottom velocity            & {\file .ubr} &  --  &    --    \\
29 & radiation stress                & {\file .Sxy} &  --  &    --    \\
30 & user defined \#1                & {\file .us1} &  --  &    --    \\
31 & user defined \#2                & {\file .us2} &  --  &    --    \\ \hline

\end{tabular} \end{center}
\caption{~Field output post processors ancillary data.} \label{tab:fields}
\vspace{0.5in}
\end{table}

\clearpage

\bpage

