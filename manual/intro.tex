\section{~Introduction}
\vssub
\subsection{~About this manual}
\vssub

This is the user manual and system documentation of version \WWver\ of the
 third-generation wind-wave modeling framework \ww. This underlying model has
 been developed at the Marine Modeling and Analysis Branch (MMAB) of the
 Environmental Modeling Center (EMC) of the National Centers for Environmental
 Prediction (NCEP). It is based on WAVEWATCH~I and WAVEWATCH~II as developed
 at Delft University of Technology, and NASA Goddard Space Flight Center,
 respectively. \ws\ differs from its predecessors in all major aspects; i.e.,
 governing equations, program structure, numerical and physical
 approaches. %\wt\ is a trademark of the United States government.

This manual describes the governing equations, numerical approaches,
compilation, and running of \ws. The format of a combined user manual and
system documentation has been chosen, to give users the necessary background
to include new physical and numerical approaches in the framework according to
their own specifications.  This approach became more important, as \ws\
developed into a wave modeling framework. By design, a user can apply his or
her numerical or physical approaches, and thus develop a new wave model based
on the \ws\ framework. In such an approach, optimization, parallelization,
nesting, input and output service programs from the framework can be easily
shared between actual models.  Whereas this document is intended to be
complete and self-contained, this is not the case for all elements in the
system documentation. For additional system details, reference is made to the
source code, which is fully documented. Note that a best practices guide for
code development for \ws\ is now available \citep{tol:MMAB09b}.

The governing equations and numerical approaches used in this model are
described in chapters~\ref{chapt:eq} and \ref{chapt:num}. Running the model is
described in chapter~\ref{chapt:run}. Installing \ws\ is described in
chapter~\ref{chapt:impl}. Finally, a short system documentation is given in
chapter~\ref{chapt:sys}. A thorough knowledge of \ws\ can be obtained by
following chapters~\ref{chapt:eq} through \ref{chapt:impl}. A shortcut is to
first install the model (chapter~\ref{chapt:impl}), and then successively
modify input files in example runs (chapter~\ref{chapt:run}).

\vspace{\baselineskip} 
\pb
\noindent 
The present model version (\WWver) is a developmental version based on the
last official model release (version 3.14). Since the latter release the
following modifications have been made:

\begin{list}{$\bullet$}{\rightmargin 5mm \parsep 0mm \itemsep 0mm}

\item
The code is now maintained using subversion \citep{bk:CSea06}. For
co-developers, a script is available to install \ws\ directly from the
repository at NCEP \citep[see][]{tol:MMAB09b} (model version 3.14/4.00).

\item
Slot for new grids (model version 4.01)

\item
Slot for new grids (model version 4.02)

\item
\ldots (model version 4.03)

\end{list}

\vspace{\baselineskip} \noindent 
A note of caution. \ww\ includes wave-current interactions. The implementation
of these interactions has only been tested in idealized test cases. Only
limited tests in realistic conditions have been performed.

\vspace{\baselineskip} \noindent 
Up to date information on this model can be found (including bugs and bug
fixes) on the \ww\ web page, and comments, questions and suggestions should be
directed to the corresponding E-mail address

\begin{center}
http://polar.ncep.noaa.gov/waves/wavewatch \\
NCEP.EMC.wavewatch@NOAA.gov
\end{center}

\noindent
We will redirect questions regarding contributions from outside NCEP to the
respective authors of the codes.


\vssub
\subsection{~Licensing terms}
\vssub

Starting with model version 3.14, \ww\ is distributed under the following
licensing terms:

\vspace \baselineskip \noindent
\centerline{
\rule[1mm]{47mm}{.5mm} {\rm start of licensing terms}
\rule[1mm]{47mm}{.5mm}}

\vspace \baselineskip \noindent 
Software, as understood herein, shall be broadly interpreted as being
inclusive of algorithms, source code, object code, data bases and related
documentation, all of which shall be furnished free of charge to the Licensee.

Corrections, upgrades or enhancements may be furnished and, if furnished,
shall also be furnished to the Licensee without charge. NOAA, however, is not
required to develop or furnish such corrections, upgrades or enhancements.

NOAA's software, whether that initially furnished or corrections or upgrades,
are furnished “as is.” NOAA furnishes its software without any warranty
whatsoever and is not responsible for any direct, indirect or consequential
damages that may be incurred by the Licensee.  Warranties of merchantability,
fitness for any particular purpose, title, and non-infringement, are
specifically negated.

The Licensee is not required to develop any software related to the licensed
software.  However, in the event that the Licensee does so, the Licensee is
required to offer same to NOAA for inclusion under the instant licensing terms
with NOAA's licensed software along with documentation regarding its
principles, use and its advantages.  This includes changes to the wave model
proper including numerical and physical approaches to wave modeling, and
boundary layer parameterizations embedded in the wave model The Licensee is
encouraged but not obligated to provide pre-and post processing tools for
model input and output.  The software required to be offered shall not include
additional models to which the wave model may be coupled, such as oceanic or
atmospheric circulation models. The software provided by the Licensee shall be
consistent with the latest model version available to the Licensee, and
interface routines to the software provided shall conform to programming
standards as outlined in the model documentation. The software offered to NOAA
shall be offered as is, without any warranties whatsoever and without any
liability for damages whatsoever.  NOAA shall not be required to include a
Licensee's software as part of its software.  Licensee's offered software
shall not include software developed by others.

A Licensee may reproduce sufficient software to satisfy its needs.  All copies
shall bear the name of the software with any version number as well as
replicas of any applied copyright notice, trademark notice, other notices and
credit lines.  Additionally, if the copies have been modified, e.g. with
deletions or additions, this shall be so stated and identified.

All of Licensee's employees who have a need to use the software may have
access to the software but only after reading the instant license and stating,
in writing, that they have read and understood the license and have agreed to
its terms.  Licensee is responsible for employing reasonable efforts to assure
that only those of its employees that should have access to the software, in
fact, have access.

The Licensee may use the software for any purpose relating to sea state
prediction.

No disclosure of any portion of the software, whether by means of a media or
verbally, may be made to any third party by the Licensee or the Licensee's
employees

The Licensee is responsible for compliance with any applicable export or
import control laws of the United States.

\vspace \baselineskip \noindent
\centerline{
\rule[1mm]{48mm}{.5mm} {\rm end of licensing terms}
\rule[1mm]{48mm}{.5mm}}

\vspace \baselineskip \noindent
The software will be distributed through our web site after the Licensee has
agreed to the license terms.


\vssub
\subsection{~Copyrights and trademarks}
\vssub

\ww\ \copyright\ 2009 National Weather Service, National Oceanic and
Atmospheric Administration.  All rights reserved. \ww\ is a trademark of the
National Weather Service. No unauthorized use without permission.

\vssub
\subsection{~The development group}
\vssub

Even in its original development, when I was working on the code mostly on my
own on the development of \ww, many have contributed to the success of the
model. With the expansion of physical and numerical parameterizations
available, the list of contributors to this model is growing. I would like to
recognize the following contributors as the development group (in alphabetic
order) :

\begin{list}{\ldots}{ }
\item [Henrique Alves] (Metocean Engineers, Australia) \\
Support of code development while at NCEP, shallow water physics packages.
\item [Fabrice Ardhuin] (Service Hydrographique et Oc\'{e}anographique de la
Marine, France) \\
Various  physics packages.
\item [Nico Booij] (Delft University of Technology, The Netherlands) \\
Original design of source code pre-processor ({\code w3adc}), basic method of
documentation and other programming habits. Spatially varying wavenumber grid.
\item [Dmitry V. Chalikov] (ESSIC, Univ. of Maryland, USA) \\ Co-author of the
\cite{tol:JPO96} input and dissipation parameterizations and source code.
\item [Arun Chawla](Science Applications International Corporation, USA) \\
Support of code development at NCEP, GRIB packing, automated grid generation
software \citep{tol:MMAB07a, tol:OMOD08a}.
%\item [Vladimir Krasnopolsky] (Science Applications International Corporation,
%  USA) \\
%Neural Network nonlinear interaction approaches.
\item [Barbara Tracy] (US Army Corps of Engineers, ERDC-CHL, USA) \\
Spectral partitioning
\item [Gerbrant Ph. van Vledder] (Alkyon Hydraulic Consultancy \& Research,
NL) \\ 
Webb-Resio-Tracy exact nonlinear interaction routines, as well as some of the
original service routines.
\end{list}


\vssub
\subsection{~Acknowledgments}
\vssub

The development of \ww\ has been an ongoing process for well over a
decade. The development of WAVEWATCH I was entirely funded through my
Ph.D. work at Delft University. The development of WAVEWATCH II has been
funded entirely through my position as a National Research Council Resident
Research Associate at NASA, Goddard Space Flight Center. The initial
development of WAVEWATCH III version 1.18 was entirely funded by NOAA/NCEP,
with most funding provided by the NOAA High Performance Computing and
Communication (HPCC) office. Developments of version 2.22 have been funded
similarly through NOAA/NCEP. Since then, funding is provided by NCEP, but also
by many partners outside NCEP. Much of the work at NCEP has been performed
under contract by Science Applications International Corporation (SAIC).
Special thanks are due to to The Office of Naval Research (ONR), for the
funding of many upcoming model upgrades.

I would finally like to thank all users who have reported errors and glitches,
or have made suggestions for improvements, particularly those who have
beta-tested this and previous model release.

%\bpage
