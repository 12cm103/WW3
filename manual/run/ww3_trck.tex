\vsssub
\subsubsection{Track output post-processor} \label{sec:ww3trck}
\vsssub

\proddefH{ww3\_trck}{w3trck}{ww3\_trck.ftn}
\proddeff{Input}{track\_o.ww3}{Raw track output data.}{11}
\proddeff{Output}{standard out}{Formatted output of program.}{6}
\proddefa{track.ww3}{Formatted data file.}{51}

\vspace{\baselineskip} 
\noindent
This post-processor does not require a formatted input file with program
commands. It will simply convert the entire unformatted file to an integer
compressed formatted file. The file contains the following header records :

\begin{list}{$\bullet$}{\itemsep 0mm \parsep 0mm}
\item File identifier (character string of length 34).
\item Number of frequencies and directions, first direction and directional
      increment (radians, oceanographic convention).
\item Radian frequencies of each frequency bin.
\item Corresponding directional bin size times frequency bin size to obtain
      discrete energy per bin.
\end{list}

\noindent
For each output point the following records are printed :
\begin{list}{$\bullet$}{\itemsep 0mm \parsep 0mm}
\item Date and time in {\tt yyyymmdd hhmmss} format, longitude and latitude in
      degrees, and a status identifier `{\F ice}', `{\F lnd}' or `{\F
      sea}'. The following two records are written only for sea points.
\item Water depth in meters, current and wind u and v components in meters per
      second, friction velocity in meters per second, air-sea temperature
      difference in degrees centigrade and scale factor for spectrum.
\item The entire spectrum in integer packed format (can be read using free
      format).
\end{list}

\pb