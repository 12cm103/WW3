\vssub
\subsection{~The data assimilation interface} \label{sec:das}
\vssub

As discussed above, the wave model subroutine is supplemented with a data
assimilation interface routine ({\F w3wdas} in {\file w3wdasmd.ftn}). This
routine is integrated in the stand-alone shell (see \para\ref{sec:ww3shel}) to
provide time step management of a combined wave model / data assimilation
scheme. It has not yet been integrated in the multi-grid model driver,
although it is accounted for in the multi-grid model management
algorithm. In this a fairly simple approach is assumed where data assimilation
is performed at selected times, while the wave model marches forward in
time. In the setup of the shell, the data assimilation is performed after the
model has reached the target time, but has not yet produced output. After the
data assimilation is performed, the wave model routine is called again only to
generate output as requested. Thus, the wave model output for a given time
will include the effects of data assimilation for that specific target time.

The generic program shell also processes several types of data to be
assimilated, and passes it on to the data assimilation interface routine. All
data needs to be preprocessed using the wave model input preprocessor (see
\para\ref{sec:ww3prep}), and will be recognized by the generic shell by file
name. Presently, up to three different data files can be used. Tentatively,
these could be mean wave parameters, one dimensional spectral data, and two
dimensional spectral data, respectively. This is, however, not hardwired to
the model and in fact needs to be defined by the user.

Presently, no data assimilation packages are available. User supplied data
assimilation schemes can be included in the wave model using the interface
routine ({\F w3wdas} in {\file w3wdasmd.ftn}), the documentation of which
should be sufficient for the necessary programming. Details on how to add user
supplied software to the \ws\ compilation system can be found in the following
chapter. NCEP is presently working on wave data assimilation techniques, but
presently has no plans to distribute wave data assimilation software.
