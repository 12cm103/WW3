\begin{verbatim}
$ -------------------------------------------------------------------- $
$ WAVEWATCH III Field preprocessor input file                          $
$ -------------------------------------------------------------------- $
$ Mayor types of field and time flag
$   Field types  :  ICE   Ice concentrations.
$                   LEV   Water levels.
$                   WND   Winds.
$                   WNS   Winds (including air-sea temp. dif.)
$                   CUR   Currents.
$                   DAT   Data for assimilation.
$   Format types :  AI    Transfer field 'as is'.
$                   LL    Field defined on regular longitude-latitude
$                         or Cartesian grid.
$                   F1    Arbitrary grid, longitude and latitude of
$                         each grid point given in separate file.
$                   F2    Like F1, composite of 2 fields.
$
$   NOTE : Options F1 and F2 have not yet been set up or tested for the
$          Cartesian version of the model.
$        - Format type not used for field type 'DAT'.
$
$   Time flag    : If true, time is included in file.
$   Header flag  : If true, header is added to file.
$                  (necessary for reading, FALSE is used only for
$                   incremental generation of a data file.)
$
   'ICE' 'LL' F T
$
$ Additional time input ---------------------------------------------- $
$ If time flag is .FALSE., give time of field in yyyymmdd hhmmss format.
$
   19680606 053000
$
$ Additional input format type 'LL' ---------------------------------- $
$ Grid range (degr. or m) and number of points for axes, respectively.
$ Example for longitude-latitude grid.
$
    -0.25 2.5  15  -0.25  2.5  4
$
$ Additional input format type 'F1' or 'F2' -------------------------- $
$ Three or four additional input lines, to define the file(s) with
$ the grid information :
$ 1) Discrete size of input grid (NXI,NYI).
$ 2) Define type of file using the parameters FROM, IDLA, IDFM (see
$    input for grid preprocessor), and a format
$ 3) Unit number and (dummy) name of first file.
$ 4) Unit number and (dummy) name of second file (F2 only).
$
$ 15  3                                              
$ 'UNIT' 3 1 '(.L.L.)'
$ 10 'll_file.1'
$ 10 'll_file.2'      
$
$ Additional input for data ------------------------------------------ $
$ Dimension of data (0,1,2 for mean pars, 1D or 2D spectra), "record
$ length" for data, data value for missing data
$
$  0  4  -999.
$
$ Define data files -------------------------------------------------- $
$ The first input line identifies the file format with FROM, IDLA and
$ IDFM, the second (third) lines give the file unit number and name.
$
  'UNIT' 3 1 '(..T..)' '(..F..)' 
  10 'data_file.1'
$ 10 'data_file.2'
$
$ If the above unit numbers are 10, data is read from this file
$ (no intermediate comment lines allowed),
$ This example is an ice concentration field.
$
   1. 1. 1. 1. 1. 1. 0. 0. 0. 0. 0. 0. 0. 0. 0.
   1. 1. .5 .5 .5 0. 0. 0. 0. 0. 0. 0. 0. 0. 0.
   0. 0. 0. 0. 0. 0. 0. 0. 0. 0. 0. 0. 0. 0. 0.
   0. 0. 0. 0. 0. 0. 0. 0. 0. 0. 0. 0. 0. 0. 0.
$
$ This example is mean parameter assimilation data
$ First record gives number of data records, data are read as as
$ individual records of reals with recored length as given above
$
$  3
$ 1.5  1.6 0.70 10.3
$ 1.7  1.5 0.75  9.8
$ 1.9  1.4 0.77 11.1
$
$ -------------------------------------------------------------------- $
$ End of input file                                                    $
$ -------------------------------------------------------------------- $
\end{verbatim}
