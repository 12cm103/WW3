\addcontentsline{toc}{subsubsection}{\strut \hspace{24mm} Third order scheme (UQ)}

\vspace{\baselineskip}
\vspace{\baselineskip} 
\noindent {\bf Third order scheme (UQ)}

\opthead{UQ}{\ws}{H. L. Tolman}

\noindent 
The third order accurate scheme available in \ws\ is the \qck\ scheme
\citep{art:Leo79,art:DM82} combined with the \ult\ \tvd\ (total variance
diminishing) limiter \citep{art:Leo91}. This is the default propagation scheme
for \ws. This scheme is third-order accurate in both space and time, and has
been selected based on the extensive intercomparison of higher order finite
difference schemes for water quality models performed by
\citep[see][]{PhD:Cah,pro:FC93,tol:OMB95}. This scheme is applied to
propagation in longitudinal and latitudinal directions separately, alternating
the direction to be treated first.

In the \qck\ scheme the flux between grid points with counters $i$ and $i-1$
in $\phi$-space $(\cF_{i,-})$ is calculated as\footnote{~Fluxes $(\cF_{i,+})$
between grid points with counters $i+1$ and $i$ again are obtained by
substituting the appropriate indices.}

% ------ QUICKEST scheme ---------- %
% eq:quick_1         Basic flux
% eq:quick_2         Boundary value
% eq:quick_3         Divergence
% eq:quick_4         CFL number

\begin{equation}
\cF_{i,-} = \left [ \dot{\phi}_b \: \cN_b \: \right ]^n_{j,l,m} \: , \label{eq:quick_1}
\end{equation} \begin{equation}
\dot{\phi}_b = 0.5 \: \left ( \dot{\phi}_{i-1} + \dot{\phi}_i 
\: \right ) \: , \label{eq:quick_1a}
\end{equation}  \begin{equation}
\cN_b = \frac{1}{2} \left [ \rule[0mm]{0mm}{\baselineskip} \: 
(1+C)\cN_{i-1} + (1-C)\cN_i \: \right ] - \:
\left ( \frac{1-C^2}{6} \right ) \: {\cal CU} \: \Delta\phi^2, \label{eq:quick_2} \end{equation} \begin{equation}
{\cal CU} =  \left \{ \begin{array}{ccc}
(\: \cN_{i-2} - 2\cN_{i-1} + \cN_{i} \:)\: \: \Delta \phi^{-2}
               & \mbox{for} & \dot{\phi}_b \geq 0 \\
(\: \cN_{i-1} - 2\cN_{i} + \cN_{i+1} \:)\: \Delta \phi^{-2}
               & \mbox{for} & \dot{\phi}_b   <  0
\end{array} \right . \: , \label{eq:quick_3}
\end{equation} \begin{equation}
C = \frac{\dot{\phi}_b \: \Delta t}{\Delta \phi} 
\: , \label{eq:quick_4} \end{equation}

\noindent
where $\cal CU$ is the (upstream) curvature of the action density
distribution, and where $C$ is a \cfl\ number including a sign to identify the
propagation direction. Like the first order scheme, this scheme gives stable
solutions for $|C| \leq 1$. To assure that this scheme does not generate
aphysical extrema, it is used in combination with the \ult\ limiter. This
limiter uses the central, upstream and downstream action density (suffices
$c$, $u$ and $d$, respectively), which are defined as

% ------ ULTIMATE limiter --------- %
% eq:ult_1           Cell definition
% eq:ult_2           Nondimensional action
% eq:ult_3           Monotonic limiter
% eq:ult_4           Non-monotonic action

\begin{equation} \begin{array}{ccccc}
\cN_c = \cN_{i-1} \: , \: & \cN_u = \cN_{ i-2 } \: , \: &
                            \cN_d = \cN_{i} &
                            \mbox{for} & \dot{\phi}_b \geq 0 \\
\cN_c = \cN_{ i } \: , \: & \cN_u = \cN_{i+1} \: , \: &
                            \cN_d = \cN_{i-1} &
                            \mbox{for} & \dot{\phi}_b   <  0
\end{array} \: . \label{eq:ult_1} \end{equation}

\noindent
To assess if the initial state and the solution show similar monotonic or
non-monotonic behavior, the normalized action $\tilde\cN$ is defined

\begin{equation}
\tilde\cN = \frac{\cN-\cN_u}{\cN_d - \cN_u} 
\: . \label{eq:ult_2} \end{equation}

\noindent
If the initial state is monotonic (i.e., $0 \leq \tilde\cN_c \leq 1$), the
(normalized) action at the cell boundary $\cN_b$ is limited to

\begin{equation}
\tilde\cN_c \leq \tilde\cN_b \leq 1 \:\:\: , \:\:\:
\tilde\cN_b \leq \tilde\cN_c \: C^{-1} \:\:\: . 
\label{eq:ult_3} \end{equation}
\noindent otherwise \begin{equation}
\tilde\cN_b = \tilde\cN_c 
\: . \label{eq:ult_4} \end{equation}

\noindent
An alternative scheme is necessary if one of the two grid points adjacent to
the cell boundary is on land or represents an active boundary point. In such
cases, Eqs.~(\ref{eq:1up_xy_2}) and (\ref{eq:quick_2}) are replaced by

% eq:1_vel           Boundary velocity
% eq:1_bval          Boundary value

\begin{equation}
\dot{\phi}_b = \dot{\phi}_s \: ,
\label{eq:1_vel} \end{equation} \begin{equation}
\cN_b = \cN_u \: ,
\label{eq:1_bval} \end{equation}

\noindent
where the suffix $s$ indicates the (average of) the sea point(s). This
boundary condition represents a simple first order upwind scheme, which does
not require the limiter (\ref{eq:ult_1}) through (\ref{eq:ult_4}).

The final propagation scheme, similar to Eq.~(\ref{eq:1up_xy_tot}),  becomes

% eq:uq_xy_tot

\begin{equation}
\cN_{i,j,l,m}^{n+1} = \cN_{i,j,l,m}^n +
\frac{\Delta t}{\Delta \phi} \left [ \cF_{i,-} - \cF_{i,+} \right ]
\: . \label{eq:uq_xy_tot} \end{equation}

\noindent
The scheme for propagation in $\lambda$-space is simply obtained by rotating
indices and increments in the above equations\footnote{~The `soft' boundary
treatment as described on page 31 of \cite{tol:OMB02c} is no longer available,
because it is incompatible with the advanced nesting techniques introduced in
model version 3.14.}

Note that the \uq\ scheme is implemented as alternate one-dimensional schemes,
for which the maxima of component advection speeds need to be used in
Eq.~(\ref{eq:dtpl}). For consistency, the same time steps are always used for
$\lambda$ and $\phi$ propagation for a given component.
