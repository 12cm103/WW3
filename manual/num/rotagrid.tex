\vssub
\subsubsection{~Rotated grids} \label{sub:num_space_rotagrid}
\opthead{RTD}{\ws\ (MetOffice)}{J.-G. Li}

\noindent
Rotated grid is a latitude-longitude (lat-lon) grid and is obtained by
rotating the North Pole to a new pole position at latitude $\phi_{p}$ and
longitude $\lambda_{p}$ in the standard latitude-longitude system.  The new
pole position is chosen so that an interested model domain may be placed
around the rotated equatorial area for a evenly spaced lat-lon mesh. For this
reason the rotated grid is also known as \emph{Equatorial grid}. For instance,
the North Atlantic and European wave (NAEW) model used in the UK Met Office
uses a rotated pole at 37.5N, 177.5E so that London, UK
(\textasciitilde{}51.5N 0.0E) is almost on the rotated equator. This rotated
grid allows a much more evenly spaced lat-lon mesh in the NAE domain than the
standard lat-lon grid in the same area. In \ws\, the rotated grid is
implemented with minimum changes to the original lat-lon grid. In fact, the
rotated grid is treated just like the standard lat-lon grid inside the
model. Only input and output files are modified for the rotated grid. Users
should choose the regular lat-lon grid along with the RTD switch to use the
rotated grid. Model input files, like wind, current and ice files should be
mapped on to the rotated grid. For convenience of nesting in standard lat-lon
grid, boundary conditions for the rotated grid use standard lat-lon grid
points, which are converted into rotated grid lat-lon inside \ws\. List of 2D
spectral output locations in ww3\_shel.inp are also specified in standard
lat-lon. All directional output such as wind direction, peak direction, 2D
spectra, etc. are converted into standard lat-lon orientation. Full grid
output are still on rotated grid but 2D spectra locations have been converted
into standard lat-lon.

Four subroutines are provided in module {\bf w3servmd.ftn} for rotated grid
conversion:
\begin{vlist}
\vit{w3spectn}{}{Turns wave spectrum anti-clockwise by AnglD}
\vit{w3acturn}{}{Turns wave action(k,nth) anti-clockwise by AnglD}
\vit{w3lltoeq}{}{Convert standard into rotated lat/lon plus AnglD}
\vit{w3eqtoll}{}{Reverse of w3lltoeq, but AnglD unchanged}
\end{vlist}
These subroutines are self-contained and can be extracted outside the model
for pre- or post-processing of rotated grid files. Some conversion tools have
been developed based on these subroutines but have not been included in \ws\
yet. Refer to the regression test \emph{regtests/ww3\_tp2.11} for an example
of a rotated grid model (NAEW) and the document \emph{RotatedGrid.pdf} for
rotated grid formulations. For more information, users may contact
\url{Jian-Guo.Li@metoffice.gov.uk}.
