\vssub
\subsection{~Nesting} \label{sub:num_nest}
\conthead{\ws}{H. L. Tolman}

\noindent
Traditionally, wave models only consider one-way nesting, with boundary data
from low resolution grids being provided to high resolution grids. This
approach has always been available in \ws, and is discussed in
\para\ref{sub:one_way}. In model version 3.14, a multi-grid wave model
driver was introduced, considering full two-way nesting between grids. This
approach is discussed in \para\ref{sub:two_way}. The illustrations below
consider regular grids, but the principles discussed are applicable to
curvilinear and triangular grids too.


% -------------------------------------------------------------
\vssub
\subsubsection{~Traditional one-way nesting} \label{sub:one_way}
\vssub

\begin{figure}
\begin{center}

\setlength{\unitlength}{0.001in}

\begin{picture}(3000,1800)(0,-1550)

\put(0200,-1200){\vector(1,0){2100}}
\put(2400,-1250){\makebox(400,100){\small space}}
\put(0300,-1300){\vector(0,1){1300}}
\put(0100, 0100){\makebox(400,100)[c]{\small time}}

\multiput(0300,-1200)(0,300){4}{\circle {50}}
\multiput(0600,-1200)(0,300){4}{\circle {50}}
\multiput(0900,-1200)(0,300){4}{\circle {50}}
\multiput(1200,-1200)(0,300){4}{\circle*{50}}
\multiput(1500,-1200)(0,300){4}{\circle {50}}
\multiput(1800,-1200)(0,300){4}{\circle {50}}
\multiput(2100,-1200)(0,300){4}{\circle {50}}

\put(0300,-0150){\makebox(600,100)[c]{`land'}}
\put(0900,-0200){\vector(-1,0){575}}

\put(1500,-0150){\makebox(600,100)[c]{sea}}
\put(1500,-0200){\vector(1,0){875}}
				       
\put(0900, 0050){\makebox(600,100)[c]{bound. data}}
\put(1100,-1200){\dashbox{50}(0200,1150){}}

\put(1050,-1550){\makebox(600,100)[c]{bound. scheme}}
\put(1200,-1400){\dashbox{25}(0300,1100){}}

\put(1550,-0750){\makebox(600,100)[l]{internal scheme}}
\put(1500,-0800){\vector(1,0){875}}

\end{picture}
\end{center}

\caption{Traditional one-way nesting approach as used in {\file ww3\_shel}.
         One-dimensional representation in space and time, symbols represent
         grid points.} \label{fig:nest1}
\botline
\end{figure}


The conventional wave model program {\file ww3\_shel} considers a single wave
model grid. This program includes options to transfer boundary conditions from
large-scale runs to small-scale runs. Each run can simultaneous accept one
data set with boundary conditions, and generate up to 9 data sets with
boundary conditions. To assure conservation of wave energy with incompatible
depths and currents, the boundary data consists of energy spectra
$F(\sigma,\theta)$. The data file consists of spectra at grid points of the
generating run, and information needed to interpolate spectra at the requested
boundary points. The size of the transfer files is thus minimized if the input
points for a small-scale run are located on grid lines in the large-scale
run. When used as input, the spectra are interpolated in space and time for
every global time step $\Delta t_g$, using a linear interpolation of spectral
components. 

The numerical approach for including boundary data in a wave model is
illustrated in Fig.~\ref{fig:nest1}. Active boundary points are assigned in
the grid to separate sea points from land points or from otherwise deactivated
grid points. Between the active boundary points and sea points, a local
boundary scheme is applied (typically first order). In the internal sea points
of the model, the selected propagation scheme is used.

Practical aspect of the conventional one-way nesting approach are
discussed in more detail in Appendix~\ref{app:nest}.


% -------------------------------------------------------------
\vssub
\subsubsection{~Two-way nesting} \label{sub:two_way}
\vssub

Model version 3.14 includes an option to use the multi-grid or mosaic approach
to wave modeling with the the program {\file ww3\_multi} \citep{tol:Vict06a,
  tol:MMAB07b, tol:OMOD08b}. In this program, an arbitrary number of grids
with arbitrary resolutions is considered, with data exchange between grids at
each relevant model time step. The grids are given a rank number, where lower
rank corresponds to lower resolution, and equal rank corresponds to similar
resolution (but not necessarily equal resolution).  Three types of data
transfer between grids are considered. These are

\begin{itemize}
\item Transfer of data from lower to higher rank grids.
\item Transfer of data from higher to lower rank grids.
\item Transfer of data between grids with equal rank.
\end{itemize}

Data transfer from lower to higher ranked grids is accomplished by providing
boundary data to the higher ranked grid, as in the traditional one-way nesting
approach described in the previous section and in Fig.~\ref{fig:nest1}.

\begin{figure}
\begin{center}

\setlength{\unitlength}{0.001in}

\begin{picture}(3000,1900)(0,-1600)

\multiput(0300,-1400)(0,300){6}{\circle {50}}
\multiput(0600,-1400)(0,300){6}{\circle {50}}
\multiput(0900,-1400)(0,300){6}{\circle {50}}
\multiput(1200,-1400)(0,300){6}{\circle {50}}
\multiput(1500,-1400)(0,300){6}{\circle {50}}
\multiput(1800,-1400)(0,300){6}{\circle {50}}
\multiput(2100,-1400)(0,300){6}{\circle {50}}
\multiput(2400,-1400)(0,300){6}{\circle {50}}
\multiput(2700,-1400)(0,300){6}{\circle {50}}

\multiput(0450,-1550)(300,0){8}{\dashbox{50}(0000,1800){}}
\multiput(0150,-1250)(0,300){5}{\dashbox{50}(2700,0000){}}

\multiput(0400,-1450)(0,800){3}{\circle*{100}}
\multiput(1400,-1450)(0,800){3}{\circle*{100}}
\multiput(2400,-1450)(0,800){3}{\circle*{100}}

\put(895,-1055){\framebox(1010,810){}}
\put(900,-1050){\framebox(1000,800){}}
\put(905,-1045){\framebox(990,790){}}

\end{picture}
\end{center}

\caption{Concept for reconciling lower ranked grid with higher ranked grid in
         two-way nesting approach. $\circ$ and hashed lines represent the
         higher ranked grid points and grid boxes, respectively, $\bullet$ and
         solid lines represent lower ranked grid and central grid box.}
\label{fig:nest2}
\botline
\end{figure}


When this approach is combined with data transfer from higher to lower rank, a
full two way nesting approach is established. In {\file ww3\_multi} the data
at the lower ranked grids is reconciled with the data at the higher ranked
grids after the higher ranked grids have `caught up' in time with the lower
ranked grids.  Considering that the resolution of the lower ranked grid by
definition is lower that the resolution of the higher ranked grid, a natural
way to estimate the wave energy in the lower ranked grid $E_{l,i}$ from energy
in the higher ranked grid $E_{h,j}$ is

\begin{equation}
E_{l,i} = \sum w_{i,j} W_{h,j} \:\:\: , \label{eq:nest_hg1}
\end{equation}

\noindent
where $i$ and $j$ are grid counters in the two grids, and where $w_{i,j}$ are
averaging weights. The weights can be defined consistent with conservation of
wave energy as the surface of the grid box $j$ in the higher ranked grid that
covers the grid box $i$ in the lower ranked grid, normalized with the surface
of the lower ranked grid box $i$. This is illustrated in Fig.~\ref{fig:nest2}.
To avoid circular reconciliation, grid points in the lower ranked grid that
contribute to the boundary data in the higher ranked grid are not updated in
this manner.

\begin{figure}
\begin{center}

\setlength{\unitlength}{0.001in}

\begin{picture}(3000,1900)(0,-1600)

\multiput(0300,-1400)(0,300){5}{\circle {50}}
\multiput(0600,-1400)(0,300){5}{\circle {50}}
\multiput(0900,-1400)(0,300){5}{\circle {50}}
\multiput(1200,-1400)(0,300){5}{\circle {50}}
\multiput(1500,-1400)(0,300){5}{\circle {50}}
\multiput(1800,-1400)(0,300){5}{\circle {50}}
\multiput(2100,-1400)(0,300){5}{\circle {50}}

\multiput(0950,-1275)(0,300){5}{\circle*{50}}
\multiput(1250,-1275)(0,300){5}{\circle*{50}}
\multiput(1550,-1275)(0,300){5}{\circle*{50}}
\multiput(1850,-1275)(0,300){5}{\circle*{50}}
\multiput(2150,-1275)(0,300){5}{\circle*{50}}
\multiput(2450,-1275)(0,300){5}{\circle*{50}}
\multiput(2750,-1275)(0,300){5}{\circle*{50}}

\put(1325,-1500){\dashbox{25}(0400,1600){}}
\put(1300,  100){\vector(-1,0){1100}}
\put(1725,  100){\vector(1,0){1100}}
\put( 725,  150){\makebox(400,100)[c]{A}}
\put(1325,  150){\makebox(400,100)[c]{B}}
\put(1925,  150){\makebox(400,100)[c]{C}}

\end{picture}
\end{center}

\caption{Concept for reconciling grids with identical rank and therefore
         similar resolution. $\circ$ represents points of grid 1, $\bullet$
         represents grid 2.}
\label{fig:nest3}
\botline
\end{figure}


Overlapping grids with similar rank cannot use the above two-way nesting
technique to consistently exchanger data. Instead, all such grids are
propagated one time step, after which the grids are reconciled as is
illustrated in Fig.~\ref{fig:nest3} For grid 1 ($\circ$ in
Fig.~\ref{fig:nest3}) two areas can be distinguished. In area C, the influence
of the boundary has propagated into the grid since the last
reconciliation. The actual depth of penetration depends on the stencil width
of the numerical scheme, and the number of propagation time steps.  In areas A
and B, information from the boundary has not yet penetrated, and this area can
be considered as the `interior' of grid 1.  Similarly, area A represents the
boundary penetration depth for grid 2 ($\bullet$ in Fig.~\ref{fig:nest3})
whereas B and C represent the interior of grid 2.  A simple and consistent
reconciliation between grid 1 and 2 uses data from grid 1 exclusively in area
A (interpolating data from grid 1 to grid points in grid 2 as necessary), and
uses data from grid 2 exclusively in area C. In area B, where interior parts
of both grids overlap, a consistent solution can be found by using weighted
averages from both grids. Note that this approach is easily extended to
multiple overlapping grids.

Note that for explicit numerical propagation schemes and overlapping grids with
identical resolution and coinciding grid points, solutions for overlapping
grids and the compatible single grid can be identical, as long as the overlap
areas are sufficiently wide.

\vspace{\baselineskip} 
\noindent 
The two-way nesting techniques in {\file ww3\_multi} are largely automated.
Each grid is prepared individually, with its own preferred time stepping
information. Locations where each grid expects to get boundary data are marked
as in the one-way nesting approach. All other bookkeeping needed to implement
the two-way nesting techniques are automated, although some iterations may be
needed to assure that all input boundary points defined in each grid can be
provided with boundary data from other grids in the multi-grid application.
Alternatively, each grid can obtain data from an external data file as in the
traditional nesting approach. In the present implementation, each grid has to
obtain all boundary data from a single file, of from other grids in the
multi-grid application, but cannot receive data from file and grids
simultaneously. Details on the management algorithm developed to run all grid
simultaneously can be found in \citet[section 3.4]{tol:MMAB07b} or
\cite{tol:OMOD08b}, and will not be reproduced here.

Note that the grids used in {\file ww3\_multi} do not need to have the same
spectral discretization. Spectra are converted on the fly in {\file
ww3\_multi}. Details on the numerical techniques used for this approach can be
found in \citet[section 3.5.5]{tol:MMAB07b}.

Grid generation for multiple grids in such an approach can be cumbersome, and
consistency between grids is required for consistent model results. For this
reason automated grid generation utilities have been developed by
\cite{tol:MMAB07a, tol:OMOD08a}.