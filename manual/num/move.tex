\vsssub
\subsubsection{~Continuously moving grids} \label{sub:num_move}
\opthead{MGx}{\ws}{H. L. Tolman}

\addcontentsline{toc}{subsubsection}{\strut \hspace{24mm} General concepts}

\noindent {\bf General concepts}
\vspace{\baselineskip} 
\noindent
In order to address wave growth issues in rapidly changing, small scale
conditions such as hurricanes, an option to add a given continuous advection
speed to the grid has been added to the model in model version 3.02. This
model version is described in detail in \cite{tol:OMOD05b}. Here, only a
cursory description is given.

\begin{center}
\rule[1mm]{55mm}{1.0mm} WARNING \rule[1mm]{55mm}{1.0mm} \\
 \vspace{\baselineskip}
\parbox{120mm}{The continuously moving grid version of \ws\ is only intended 
for testing wave model properties in highly idealized conditions. This model
version should only be used for deep water without mean currents and land
masses. Furthermore, to avoid complications with great circle propagation,
only Cartesian grid should be used. The option is furthermore implemented only
for propagation options {\F pr1} and {\F pr3}. Note that this is not checked
in the scripts or programs at either the compile or run time level.  This
option is not described in the body of the manual but in this appendix only,
because it is not considered to be a general application.} \\
\vspace{\baselineskip}
\rule[1mm]{55mm}{1.0mm} WARNING \rule[1mm]{55mm}{1.0mm}
\end{center}

\noindent
For the above described application Eq.~(\ref{eq:bal_plane}) can be written as

\begin{equation}
\frac{\partial N}{\partial t} + 
\left ( {\bf\dot{x}}-{\bf{v}}_g \right ) \cdot \nabla_x N  = 
\frac{S}{\sigma} \: , \label{eq:bal_move}
\end{equation}

\noindent
where ${\bf{v}}_g$ represents the advection velocity of the grid. This option
is selected when compiling the model (see below). A second compile level
option allows for adding the grid advection velocity ${\bf{v}}_g$ to the wind
field. This allows for a simple method to assure mass conservation of a wind
field independent of the actual and instantaneous grid advection velocity. The
advection velocity ${\bf{v}}_g$ can vary in time and is provided by the user
at the run time of the model (see below).

\vspace{\baselineskip} 
\vspace{\baselineskip} 
\addcontentsline{toc}{subsubsection}{\strut \hspace{24mm} Numerical implementation}
\noindent {\bf Numerical implementation}
\vspace{\baselineskip} 

\noindent
For the simplified conditions for which Eq.~(\ref{eq:bal_move}) is valid, the
implementation of the moving grids is trivial if it is considered that this
equation is equivalent to

\begin{equation}
\frac{\partial N}{\partial t} + 
\nabla_x \cdot \left ( {\bf\dot{x}}-{\bf{v}}_g \right ) N  = 
\frac{S}{\sigma} \: , \label{eq:bal_move2}
\end{equation}

\noindent
which in turn implies that the advection velocity ${\bf{v}}_g$ can be added
directly to ${\bf\dot{x}}$ for arbitrary numerical schemes solving
Eq.~(\ref{eq:bal_plane}). Because this influences the net advection velocity,
it also influences stability characteristics. This impact has been accounted
for automatically by including the moving grid velocity in the calculation of
the actual propagation time step in Eq~(\ref{eq:dtpl}). Hence, the user need
to provide a proper maximum propagation time step representative for $\bf{v}_g
= 0$ only.

The motion of the grid has an apparent influence on the Garden Sprinkler
Effect (GSE), due to the different retention time in the grid of spectral
components with identical frequency but different propagation direction.
Current GSE alleviation methods tend to be more efficient for younger swells
than for older swells. Hence, swells with longer retention time in the moving
grid tend to show a more pronounced GSE \citep[see][]{tol:OMOD05b}. To
mitigate this apparent imbalance in GSE alleviation,  Eq.~(\ref{eq:GSE_avg}) is
replaced with

\begin{equation}
\pm \gamma_a \gamma_{a,s} \: \Delta c_g \: \Delta t \: \bs \:\:\: ,
\pm \gamma_a \gamma_{a,n} \: c_g \Delta \theta \: \Delta t \: \bn \:\:\: ,
\label{eq:move_GSE_avg1}
\end{equation}

\begin{equation}
\gamma_a = \left ( \frac{|\dot{\bf{x}}|}{|\dot{\bf{x}}-\bf{v}_g|}
\right ) ^{p}
\label{eq:move_GSE_avg2}
\end{equation}

\noindent
where $\gamma_a$ is a correction factor accounting for the grid movement, and
where the power $p$ is a parameter allows for some tuning. With this
modification, the effects of the GSE can be distributed more evenly over the
grid by rescaling the amount of smoothing applied with the expected residence
time of corresponding spectral component in the moving grid
\citep[see][]{tol:OMOD05b}.

\vspace{\baselineskip} 
\vspace{\baselineskip} 
\addcontentsline{toc}{subsubsection}{\strut \hspace{24mm} Running with moving grids}
\noindent {\bf Running with moving grids}
\vspace{\baselineskip} 

\noindent
To switch on the moving of the grid, or the correction of the wind field, two
optional switches are added to the \ws\ source code:

\begin{slist}
\sit{mgp} {Apply advection correction for continuous moving grid.}
\sit{mgw} {Apply wind correction for continuous moving grid.}
\sit{mgg} {Apply correction to averaging strength in GSE correction for
           continuous moving grid.}
\end{slist}

\noindent
The advection velocity and direction is input to the shell similar to the
input of homogeneous currents (see bottom of file {\file ww3\_shel.inp} in
section~\ref{sec:ww3shel}), exchanging the keyword '{\code CUR}' with '{\code
  MOV}'. The advection velocity can be changed in time like all homogeneous
input fields. An example of running with a moving grid model is given in test
case {\file ww3\_ts3}. A similar capability exist in {\file ww3\_multi.inp} in
section \ref{sec:ww3multi}, and is tested in test case {\file mww3\_test\_05}.
