\vssub
\subsection{~Non-ice source terms integration} \label{sub:source}
\vssub

Finally, the source terms are accounted for by solving

%---------------------%
% Step : Source terms %
%---------------------%
% eq:step_source

\begin{equation}
\frac{\p N}{\p t} = \cS_{no~ice} \: . \label{eq:step_source}
\end{equation}

\noindent 
As in \wam, a semi-implicit integration scheme is used. In this scheme the
discrete change of action density $\Delta N$ becomes \citep{art:WAM88}

% eq:implicit_st

\begin{equation}
\Delta N(k,\theta) = \frac{\cS(k,\theta)}{1- \epsilon D(k,\theta)\Delta t}
\: , \label{eq:implicit_st} \end{equation}

\noindent 
where $D$ represents the diagonal terms of the derivative of $\cS$ with
respect to $N$ \citep[Eqs. 4.1 through 4.10]{art:WAM88}, and where $\epsilon$
defines the offset of the scheme. Originally, $\epsilon = 0.5$ was implemented
to obtain a second order accurate scheme. Presently, $\epsilon = 1$ is used as
it is more appropriate for the large time steps in the equilibrium range of
the spectrum \citep{pro:HA98,art:HA01}, and as it result in much smoother
integration of the spectrum. The change of $\epsilon$ has little impact on
mean wave parameters, but makes the dynamical time stepping as described below
more economical.

The semi-implicit scheme is applied in the framework of a dynamic
time-stepping scheme \citep{tol:JPO92}. In this scheme, integration over the
global time step $\Delta t_g$ can be performed in several dynamic time steps
$\Delta t_d$, depending on the net source term $\cS$, a maximum change of
action density $\Delta N_m$ and the remaining time in the interval $\Delta
t_g$. For the $n^{\rm th}$ dynamic time step in the integration over the
interval $\Delta t_g$, $\Delta t_d^n$ is calculated in three steps as

% ------ Dynamic s.t. int. scheme ------- %
% eq:st_d_1
% eq:st_d_2a
% eq:st_d_2b
% eq:st_d_3

\begin{equation}
\Delta t_d^n = 
\min_{f<f_{hf}} \left [ \frac{\Delta N_m}{|\cS|}
\left ( 1 + \epsilon D \frac{\Delta N_m}{|\cS|} \right ) ^{-1}
\right ] \: , \label{eq:st_d_1}
\end{equation} \begin{equation}
\Delta t_d^n = \max \: \left [ \: \Delta t_d^n \: , \: 
\Delta t_{d,\min} \right ] \: , \label{eq:st_d_2a}
\end{equation} \begin{equation}
\Delta t_d^n = \min \: \left [ \: \Delta t_d^n \: , \: 
\Delta t_g - \sum_{i=1}^{n-1} \Delta t_d^i
 \: \right ] \: , \label{eq:st_d_2b}
\end{equation}

\noindent
where $\Delta t_{\min}$ is a user-defined minimum time step, which is added to
avoid excessively small time steps. The corresponding new spectrum $N^n$
becomes

\begin{equation}
N^n = \max\: \left [ \: 0 \: , \: N^{n-1} + 
\left ( \frac{\cS \Delta t_d}{1 - \epsilon D \Delta t_d} \right )
\: \right ] \: . \label{eq:st_d_3}
\end{equation}

\noindent 
The maximum change of action density $\Delta N_m$ is determined from a
parametric change of action density $\Delta N_p$ and a filtered relative
change $\Delta N_r$

% eq:st_d_4
% eq:st_d_5
% eq:st_d_6
% eq:st_d_7

\begin{equation}
\Delta N_m (k,\theta) = \min \: \left [ \:
\Delta N_p (k,\theta) \: , \: \Delta N_r (k,\theta) 
\: \right ] \: , \label{eq:st_d_4}
\end{equation} \begin{equation}
\Delta N_p (k,\theta) = X_p \: \frac{\alpha}{\pi} \:
\frac{(2\pi)^4}{g^2} \: \frac{1}{\sigma k^3}
\: , \label{eq:st_d_5}
\end{equation} \begin{equation}
\Delta N_r (k,\theta) = X_r \; \max \: \left [ \: 
N(k,\theta) \: , \: N_f \: \right ] \: , \label{eq:st_d_6}
\end{equation} \begin{equation}
N_f = \max \: \left [ \: \Delta N_p (k_{\max},\theta) \: , 
\: X_f \: \max_{\forall k,\theta} \left \{ N(k,\theta) \right \}
\: \right ] \: , \label{eq:st_d_7} \end{equation}

\noindent 
where $X_p$, $X_r$ and $X_f$ are user-defined constants (see
Table~\ref{tab:st_d_p}), $\alpha$ is a {\sc pm} energy level (set to $\alpha =
0.62\,10^{-4}$) and $k_{\max}$ is the maximum discrete wavenumber. The
parametric spectral shape in (\ref{eq:st_d_5}) corresponds in deep water to
the well-known high-frequency shape of the one-dimensional frequency spectrum
$F(f) \propto f^{-5}$. The link between the filter level and the maximum
parametric change in (\ref{eq:st_d_7}) is used to assure that the dynamic time
step remains reasonably large in cases with extremely small wave energies. A
final safeguard for stability of integration is provided by limiting the
discrete change of action density to the maximum parametric change
(\ref{eq:st_d_5}) in conditions where Eq.~(\ref{eq:st_d_2a}) dictates $\Delta
t_d^n$. In this case Eq.~(\ref{eq:st_d_2a}) becomes a limiter as in the WAM
model. Impacts of limiters are discussed in detail in for instance
\cite{art:HJ99,art:HJ01}, \cite{art:HA01} and \cite{tol:GAOS02}.

% tab:st_d_p

\begin{table} 
\begin{center} \begin{tabular}{|l|c|c|c|c|} \hline \hline
                 & $X_p$     & $X_r$             & $X_f$ &
$\Delta t_{d,\min}$      \\ \hline
\wam\ equivalent & $\frac{\pi}{24}10^{-3}\Delta t$
 & $\infty (\geq 1)$ & --    & $\Delta t_g$  \\ 
 suggested       & 0.1-0.2  & 0.1-0.2 & 0.05 & $\approx 0.1 \Delta t_g$ \\  
default setting  &  0.15    &   0.10  & 0.05 & -- \\ \hline \hline
\end{tabular} \end{center}
\caption{User-defined parameters in the source term integration
 scheme}
\label{tab:st_d_p} \botline \end{table}

The dynamic time step is calculated for each grid point separately, adding
additional computational effort only for grid points in which the spectrum is
subject to rapid change. The source terms are re-calculated for every dynamic
time step.

It is possible to compile \ws\ without using a linear growth term. In such a
case, waves can only grow if some energy is present in the spectrum. In
small-scale applications with persistent low wind speeds, wave energy might
disappear completely from part of the model. To assure that wave growth can
occur when the wind increases, a so-called seeding option is available in \ws\
(selected during compilation). If the seeding option is selected, the energy
level at the seeding frequency $\sigma_{\rm seed} = \min(\sigma_{\max}, 2\pi
f_{hf})$ is required to at least contain a minimum action density

% ------ Spectral seeding ------- %
% eq:seed

\begin{eqnarray}
N_{\min}(k_{seed},\theta) & = & 
        6.25 \: 10^{-4} \frac{1}{k_{\rm seed}^3 \: \sigma_{\rm seed}}
        \max \left [ \: 0. \: , \: \cos^2 ( \theta - \theta_w ) \right ]
                             \nonumber \\ & & \hspace{5mm}
        \min \left [ \: 1 \: , \: \max \left ( \: 0 \: , \: 
        \frac{|u_{10}|}{X_{\rm seed} g \sigma_{\rm seed}^{-1}}-1 
\: \right ) \: \right ] \: , \label{eq:seed} \end{eqnarray}

\noindent
where $g \sigma_{\rm seed}^{-1}$ approximates the equilibrium wind speed for
the highest discrete spectral frequency. This minimum action distribution is
aligned with the wind direction, goes to zero for low wind speeds, and is
proportional to the integration limiter (\ref{eq:st_d_5}) for large wind
speeds. $X_{\rm seed} \geq 1$ is a user-defined parameter to shift seeding to
higher frequencies. Seeding starts if the wind speed reaches $X_{\rm seed}$
times the equilibrium wind speed for the highest discrete frequency, and
reaches its full strength for twice as high wind speeds. The default model
settings include the seeding algorithm, with $X_{\rm seed} = 1$.

In model version 3.11, surf-zone physics parameterizations have been
introduced. Such physics, particularly depth-induced breaking, operate on much
smaller time scales than deep water and limited depth physics outside the surf
zone. To assure reasonable behavior for larger time steps, an additional
optional limiter has been adopted from the SWAN model, similar to the Miche
style maximum wave height in the depth limited wave breaking source term of
Eq.~(\ref{eq:BJ78_Miche}). In this limiter, the maximum wave energy $E_m$ is
computed as

% ------ Surf zone limiter ------ %
% eq:MLIM

\begin{equation}
E_m = \frac{1}{16} [ \gamma_{lim}  \tanh ( \bar{k} d ) / \bar{k} ] ^2
\:\:\: , \label{eq:MLIM} 
\end{equation}

\noindent
where $\gamma_{lim}$ is a factor comparable to $\gamma_M$ in
Eq.~(\ref{eq:BJ78_Miche}), with the caveat that $\gamma_M$ is representative
for an individual wave, whereas $\gamma_{lim}$ is representative for the
significant wave height. For monochromatic waves, the original expression by
\cite{art:Miche44} would correspond to $\gamma_{lim} = 0.94$ and replacing
$H_s$ by the height $H$ of the waves. Here this idea, is applied to random
waves.  In shallow water, this limit $H_s$ to be less than $\gamma_{lim} d$.

If the total spectral energy $E$ is larger than the maximum energy $E_m$, the
limiter is applied by simply rescaling the spectrum by the factor $E/E_m$,
loosely following the argumentation from \cite{art:EB96} and used
in \para\ref{sec:DB1}.  This limiter can be switched on or off in the
compilation of the model, and $\gamma_{lim}$ can be adjusted by the user. The
default is set to $\gamma_{lim} = 1.6$ because $H_{rms}$ values close to $d$
have indeed been recorded and thus taking a ration $H_s/H_{rms}$ of 1.4, using
1.6 allows this large steepness to be exceeded by some margin.  Note that this
limiter should be used as a `safety valve' only, and hence that it should be
less strict than the breaking criterion in the surf-breaking or whitecapping
source terms, if these source terms are modeled explicitly.

Also, this limiter does not guarantee that all parts of the spectrum are
realistic. Indeed, the use of a mean wavenumber, as in the Komen et
al. dissipation, makes it possible to have unrealistically steep short waves
in the presence of swell. A future extension of this limiter could be to limit
the steepness with a partial spectral integration in frequencies, to make sure
that waves of all scales are indeed not too steep.

\vssub
\subsection{Ice source terms integration} \label{sub:source}
\vssub

Because the attenuation and scattering in the ice can be very strong although they are linear, it is convenient 
to performe a separate integration of the ice terms $S_{\mathrm{ice}}=S_{\mathrm{id}}+S_{\mathrm{is}}$. This combines 
a dissipation term 
\begin{equation}
S_{\mathrm{id}}/\sigma  = \beta_{\mathrm{id}} N 
\end{equation}
 and a scattering term 
which is of the form 
\begin{equation}
 \frac{S_{\mathrm{is}}(k,\theta)}{\sigma} =  \int_0^{2\pi}\beta_{\mathrm{is}} [N(k,\theta')-N(k,\theta)] d\theta'  
 \end{equation}
 in which the scattering coefficient $\beta_{\mathrm{is}}$ is a priori a function of the difference in direction between 
 incident $\theta'$ and scattered $\theta$ directions, as well as the shape of ice floes. In general 
  the directional spectrum $N(k,.)$ is a vector with {\F NTH} components, 
and the source term is a vector of the same size  given by the matric product $S/\sigma = M N(k,.)$ where  
$M$ is a positive symmetric square {\F NTH} by {\F NTH} matrix with components given from the $\beta_{\mathrm{id}}$ values.
The matrix $M$ is easily diagonalized as 
\begin{equation}
 M  =  V D V^T  
 \end{equation}
 where $D$ is a diagonal matrix containing all eigenvalues and $V$ is the array of eigenvectors, 
 and $V^T$ is its transpose. As a result the split wave action equation for ice source terms 
 \begin{equation}
\frac{\p}{\p t} \frac{N}{c_g}   = \frac{S_{\mathrm{id}}}{\sigma c_g} 
 \end{equation}
 can be rewritten for the action $N_i$ of each eigenvector $V_i$ with eigenvalue $\lambda_i$ as 
  \begin{equation}
\frac{\p}{\p t} \frac{N_i}{c_g}   = \frac{\beta_{\mathrm{id}} + \lambda_i}{\sigma c_g} N_i 
 \end{equation}
 which has the following exact solution 
   \begin{equation}
 {N_i}(t+\Delta t_g)   = N_i(t) \exp \left[ (\beta_{\mathrm{id}} + \lambda_i) +\Delta t_g\right].
 \end{equation}
 
 In all cases the eigenvector corresponding to an isotropic spectrum has an eigenvalue $\lambda=\beta_{\mathrm{id}}$. 
 In the case of an isotropic back-scatter, the other eigenvalues are all equal to $(\beta_{\mathrm{id}} + \beta_{\mathrm{is}})$. 
 This decomposition over the two eigenspaces simplifies the solution to 
%---------------------%
% Step : Source terms %
%---------------------%
% eq:exact_st_ice

\begin{equation}
 N(t+\Delta t_g) = \exp(\beta_{\mathrm{id}} \Delta t_g) \overline{N}(t)  + \exp \left[\left(\beta_{\mathrm{id}} + \beta_{\mathrm{is}}\right) \Delta t_g\right] \left[N(t)-\overline{N}(t)\right]
\: , \label{eq:exact_st_ice} \end{equation}
where $\overline{N}$ is the average over all directions. As a result, for a spatially homogeneous field, 
the spectrum exponentially tends to isotropy over a time scale $1/(\beta_{\mathrm{id}})$.
