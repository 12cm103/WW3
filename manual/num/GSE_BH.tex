\addcontentsline{toc}{subsubsection}{\strut \hspace{24mm} Booij and Holthuijsen 1987}

\vspace{\baselineskip} \noindent {\bf Booij and Holthuijsen 1987}

\opthead{PR2}{\ws}{H. L. Tolman}

\noindent
The classical GSE alleviation method is from \cite{art:BH87}, who derived an
alternative propagation equation for the discrete spectrum, including a
diffusive correction to account for continuous dispersion in spite of the
discrete spectral description. This correction influences spatial propagation
only, which for general spatial coordinates $(x,y)$ becomes

% ------ Correction linear dispersion --------- %
% eq:d_bal_0         Booij and Holthuijsen, 1987
% eq:Dxx
% eq:Dyy
% eq:Dxy
% eq:Dss
% eq:Dnn

\begin{equation}
\frac{\p \cN}{\p t} +
\frac{\p}{\p x} \left [ \dot{x} \cN 
           - D_{xx} \frac{\p \cN}{\p x} \right ] +
\frac{\p}{\p y} \left [ \dot{y} \cN 
           - D_{yy} \frac{\p \cN}{\p y} \right ] -
2 D_{xy} \frac{\p^2 \cN}{\p x \p y} = 0
\: , \label{eq:d_bal_0}\end{equation}  \begin{equation}
D_{xx} = D_{ss} \, \cos^2 \theta + D_{nn} \, \sin^2 \theta
\: , \label{eq:Dxx} \end{equation} \begin{equation}
D_{yy} = D_{ss} \, \sin^2 \theta + D_{nn} \, \cos^2 \theta
\: , \label{eq:Dyy} \end{equation}  \begin{equation}
D_{xy} = ( D_{ss} - D_{nn} ) \, \cos \theta \, \sin \theta
\: , \label{eq:Dxy} \end{equation} \begin{equation}
D_{ss} = (\Delta c_g )^2 \: T_s / 12
\: , \label{eq:Dss} \end{equation} \begin{equation}
D_{nn} =  ( c_g \Delta \theta )^2 \: T_s / 12
\: , \label{eq:Dnn} \end{equation}

\noindent
where $D_{ss}$ is the diffusion coefficient in the propagation direction of
the discrete wave component, $D_{nn}$ is the diffusion coefficient along the
crest of the discrete wave component and $T_s$ is the time elapsed since the
generation of the swell. In the present fractional step method the diffusion
can be added as a separate step

% eq:d_bal_1         Separated diffusion equation

\begin{equation}
\frac{\p \cN}{\p t} = 
\frac{\p}{\p x} \left [ D_{xx} \frac{\p \cN}{\p x} \right ] +
\frac{\p}{\p y} \left [ D_{yy} \frac{\p \cN}{\p y} \right ] +
   2 D_{xy} \frac{\p^2 \cN}{\p x \p y}
\: . \label{eq:d_bal_1} \end{equation}

\noindent
This equation is incorporated with two simplifications, the justification of
which is discussed in \cite{tol:OMB95}. First, the swell `age' $T_s$ is kept
constant throughout the model (defined by the user, no default value
available). Secondly, the diffusion coefficients $D_{ss}$ and $D_{nn}$ are
calculated assuming deep water

% eq:Dss_d
% eq:Dnn_d

\begin{equation}
D_{ss} = \left ( \: (X_\sigma-1) \: \frac{\sigma_m}{2 k_m} \right )^2
\: \frac{T_s}{12} \: , \label{eq:Dss_d} \end{equation} \begin{equation}
D_{nn} =  \left ( \frac{\sigma_m}{2 k_m} \Delta \theta \right )^2
\: \frac{T_s}{12} \: , \label{eq:Dnn_d} \end{equation}

\noindent
where $X_\sigma$ is defined as in Eq.~(\ref{eq:sigma_grid}). With these two
assumptions, the diffusion tensor becomes constant throughout the spatial
domain for each separate spectral component.

Equation~(\ref{eq:d_bal_1}) is solved using a forward-time central-space
scheme. At the cell interface between points $i$ and $i-1$ in $\phi$ ($x$)
space, the term in brackets in the first term on the right side of
Eq.~(\ref{eq:d_bal_1}) (denoted as $\cD_{i,-})$ is estimated as

% eq:ddif_1          Discrete diffusion 1

\begin{equation}
D_{xx} \frac{\p \cN}{\p x} \approx \cD_{i,-} =
\left . D_{xx} \; \left ( \frac{\cN_i - \cN_{i-1}}{\Delta x} \right ) \: 
\right |_{j,l,m} \: . \label{eq:ddif_1} \end{equation}

\noindent
Corresponding values for counters $i$ and $i+1$, and for gradients in
$\lambda$ ($y$) space again are obtained by rotating indices and
increments. If one of the two grid points in located on land,
Eq.~(\ref{eq:ddif_1}) is set to zero. The mixed derivative at the right side
of Eq.~(\ref{eq:d_bal_1}) (denoted as $\cD_{ij,--}$) is estimated for the grid
point $i$ and $i-1$ in $x$-space and $j$ and $j-1$ in $y$-space as

% eq:ddif_2          Discrete diffusion 2

\begin{equation}
\cD_{ij,--} = \left . \: D_{xy} 
\left ( \frac{- \cN_{i,j} + \cN_{i-1,j} +
              \cN_{i,j-1} - \cN_{i-1,j-1}}
           {0.5 ( \Delta x_j + \Delta x_{j-1} ) \: \Delta y}
              \right ) \: \right |_{l,m} 
\: . \label{eq:ddif_2} \end{equation}

\noindent
Note that the increment $\Delta x$ is a function of $y$ due to the use of the
spherical grid. This term is evaluated only if all four grid points considered
are sea points, otherwise it is set to zero. Using a forward in time
discretization of the first term in Eq.~(\ref{eq:d_bal_1}), and central in
space discretizations for the remainder of the first and second term on the
right side, the final algorithm becomes

% eq:ddif_3          Discrete diffusion 3

\begin{eqnarray}
\cN_{i,j,l,m}^{n+1} = \cN_{i,j,l,m}^n + 
\frac{\Delta t}{\Delta x} 
   \left ( \cD_{i,+} - \cD_{i,-} \right ) +
\frac{\Delta t}{\Delta y}
   \left ( \cD_{j,+} - \cD_{j,-} \right ) \nonumber \\
+ \frac{\Delta t}{4} \left ( \cD_{ij,--} + 
   \cD_{ij,-+} + \cD_{ij,+-} + \cD_{ij,++} \right )
\: . \label{eq:ddif_3} \end{eqnarray}

\noindent
Stable solutions are obtained for \citep[e.g.,][Part I section
7.1.1]{bk:Fle88}

% eq:ddif_4          Stability

\begin{equation}
\frac{D_{\max} \: \Delta t}{\min(\Delta x , \Delta y)^2}
\leq 0.5 \: , \label{eq:ddif_4} \end{equation}

\noindent
where $D_{\max}$ is the maximum value of the diffusion coefficient (typically
$D_{\max} = D_{nn}$). Because this stability criterion is a quadratic function
of the grid increment, stability can become a serious problem at high
latitudes for large scale applications. To avoid that this puts undue
constraints on the time step of a model, a corrected swell age $T_{s,c}$ is
used

% eq:Ts_cor

\begin{equation}
T_{s,c} = T_s \min \left \{ \: 1 \: , \: 
\left ( \frac{\cos(\phi)}{\cos(\phi_c)} \right )^2 \: \right \}
\: , \label{eq:Ts_cor} \end{equation}

\noindent
where $\phi_c$ is a cut-off latitude defined by the user.

\vspace{\baselineskip} \noindent 
The above diffusion is needed for swell propagation only, but is not realistic
for growing wind seas. In the latter conditions, the \uq\ scheme without the
dispersion correction is sufficiently smooth to render stable fetch-limited
growth curves \citep{tol:OMB95}. To remove minor oscillations, a small
isotropic diffusion is used for growing wave components. To assure that this
diffusion is small and equivalent for all spectral components, it is
calculated from a preset cell Reynolds (or cell Peclet) number $\cR = c_g
\Delta x D_g^{-1} = 10$, where $D_g$ is the isotropic diffusion for growing
components

% eq:Dg              Diffusion growing components

\begin{equation}
D_g = \frac{c_g \: \min(\Delta x , \Delta y)}{\cR} \: , \label{eq:Dg}
 \end{equation}

\noindent
The diffusion for swell and for wind seas are combined using a linear
combination depending on the nondimensional wind speed or inverse wave age
$u_{10} c^{-1} = u_{10} k \sigma^{-1}$ as

% eq:Xg              Linear combination for diffusion
% eq:Dss_f           Final Dss
% eq:Dnn_f           Final Dnn

\begin{equation}
X_g = \min \: \left \{ \: 1 \: , \: \max \: \left [ \: 0 \: , \:
3.3 \left ( \frac{k \, u_{10}}{\sigma} \right ) - 2.3
\: \right ] \: \right \}
 \: , \label{eq:Xg} \end{equation} \begin{equation}
D_{ss} = X_g D_g + (1-X_g) D_{ss,p}
\: , \label{eq:Dss_f} \end{equation} \begin{equation}
D_{nn} = X_g D_g + (1-X_g) D_{nn,p}
\: , \label{eq:Dnn_f} \end{equation}

\noindent
where the suffix $p$ denotes propagation diffusion as defined in
Eqs.~(\ref{eq:Dss_d}) and (\ref{eq:Dnn_d}). The constants in Eqs~(\ref{eq:Dg})
and (\ref{eq:Xg}) are preset in the model.

