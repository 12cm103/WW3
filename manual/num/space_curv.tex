\vssub
\subsubsection{~Curvilinear grids} \label{sub:num_space_curv}
\conthead{\ws (NRL Stennis)}{W. E. Rogers, T. J. Campbell}

\noindent 
As an extension to traditional grids, computations may be made on curvilinear
grids within \ws\ . This makes it possible to run the model on alternate grid
projections (e.g. Lambert conformal conic), rotated grids, or
shoreline-following grids with higher resolution near shore, though the
restrictions on time step from the conditionally stable schemes still
apply. The same propagation schemes are utilized for irregular grids as for
regular grids (\para\ref{sub:num_space_trad}).

The implementation is described in detail in \citep{rep:RC09}, summarized
here: a Jacobian is used to convert the entire domain between the normal,
curving space, and a straightened space. This conversion is performed only
within the propagation routine, rather than integrating the entire model in
straightened space. A simple, three step process is used every time the
propagation subroutine is call (i.e. every time step and every spectral
component): first, the dependent variable (wave action density) is converted
to straightened space using a Jacobian; second, the wave action density is
propagated via subroutine calls for each (of two) grid axes; third, the wave
action density is converted back to normal, curved space. The actual flux
computation is not significantly modified from its original, regular grid
form. The same process occurs, regardless of grid type (regular or irregular);
for regular grids, the Jacobian is unity.

Regarding the user interface: in {\file ww3\_grid.inp}, a string is used to
indicate the grid type. In cases where this grid string is `{\code RECT}', the
model processes input for a regular grid. In case where this grid string is
`{\code CURV}' , the model processes input for an irregular grid. [Note that
with \ws\ version 4.00, the coordinate system (i.e. degrees vs. meters) and
the closure type (e.g. global/wrapping grid) are also specified in {\file
ww3\_grid.inp} ; the switches {\code LLG} and {\code XYG} are deprecated.]
