\vssub
\subsection{~Source terms}
\vsssub
\subsubsection{~General concepts}
\vsssub

In deep water, the net source term $S$ is generally considered to consist of
three parts, a wind-wave interaction term $S_{in}$, a nonlinear wave-wave
interactions term $S_{nl}$ and a dissipation (`whitecapping') term
$S_{ds}$. More recently, explicit swell disipation has been added to this
source term in some model.  The input term $S_{in}$ is dominated by the
exponential growth term, and this source term generally describes this
dominant process only. For model initialization, and to provide more realistic
initial wave growth, and linear input term $S_{ln}$ can also be considered in
\ws.

In shallow water additional processes have to be considered, most notably
wave-bottom interactions $S_{bot}$ \cite[e.g.,][]{pro:Sea78}. In extremely
shallow water, depth-induced breaking ($S_{db}$) and triad wave-wave
interactions ($S_{tr}$) become important. Also available in \ws\ are source
terms for scattering of waves by bottom features ($S_{sc}$), wave-ice
interactions ($S_{ice}$), reflection off shorelines or floating objects such
as icebergs ($S_{ref}$) and a general purpose slot for additional, user
defined source terms ($S_{xx}$).

This defines the general source terms used in \ws\ as

%---------------------------------%
% General source term composition %
%---------------------------------%
% eq:general_st

\begin{equation}
S = S_{ln} + S_{in} + S_{nl} + S_{ds} + S_{bot} + S_{db} + S_{tr} +
    S_{sc} + S_{ice} + S_{ref} + S_{xx}\: .
\label{eq:general_st}
\end{equation}

\noindent
Other source terms are easily added. These source terms are defined for the
{\em energy} spectra. In the model, however, most source terms are directly
calculated for the action spectrum. The latter source terms are denoted as
$\cS \equiv S/\sigma$.

The treatment of the nonlinear interactions defines a third-generation wave
model. Therefore, the options for the calculation of $S_{nl}$ will be
discussed first, starting in section \ref{sec:NL1}. $S_{in}$ and $S_{ds}$
represent separate processes, but should be considered as interrelated,
because the balance of these two source terms governs the integral growth
characteristics of the wave model. Several combinations of these basic source
terms are available, and are described in section~\ref{sec:ST1} and following.
The description of linear input starts in section~\ref{sec:LN1}, and
section~\ref{sec:BT1} and following describe available additional (mostly
shallow water) processes.

\vspace{\baselineskip} \noindent
A third-generation wave model effectively integrates the spectrum only up to a
cut-off frequency $f_{hf}$ (or wavenumber $k_{hf}$). Above this frequency a
parametric tail is applied \citep[e.g.,][]{art:WAM88}

%--------------------------------------%
% General tail parameterization E(f,t) %
%--------------------------------------%
% eq:tail_E_f

\begin{equation}
F(f_r,\theta) = F(f_{r,hf},\theta) \left ( \frac{f_r}{f_{r,hf}}
\right ) ^{-m} \label{eq:tail_E_f}
\end{equation}

\noindent
which is easily transformed to any other spectrum using the Jacobian
transformations as discussed above. For instance, for the present action
spectrum, the parametric tail can be expressed as (assuming deep water for the
wave components in the tail)

%--------------------------------------%
% General tail parameterization N(k,t) %
%--------------------------------------%
% eq:tail_N_k

\begin{equation}
N(k,\theta) = N(k_{hf},\theta) \left ( \frac{f_r}{f_{r,hf}}
\right ) ^{-m-2} \label{eq:tail_N_k}
\end{equation}

\noindent
The actual values of $m$ and the expressions for $f_{r,hf}$ depend on the
source term parameterization used, and will be given below.

\vspace{\baselineskip} \noindent
Before actual source term parameterizations are described, the definition of
the wind requires some attention. In cases with currents, one can either
consider the wind to be defined in a fixed frame of reference, or in a frame
of reference moving with the current. Both definitions are available in \ws,
and can be selected during compilation. The output of the program, however,
will always be the wind speed which is not in any way corrected for the
current.

\vspace{\baselineskip} \noindent
The treatment of partial ice coverage (ice concentration) in the source term
follows the concept of a limited air-sea interface. This means that the momentum
transferred from the atmosphere to the waves is limited. Therefore, input and
dissipation terms are scaled by the fraction of ice concentration. The
nonlinear wave-wave interaction term can be used in areas of open water
and ice \citep{art:PL07}. The scaling is implemented so that it is independent
of the source term selected.

