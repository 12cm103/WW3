\vsssub
\subsubsection{~$S_{ice}$: Damping by sea ice (Shen et al.)} \label{sec:ICE3}
\vsssub

\opthead{IC3}{Clarkson U. Fortran-77 code}{E. Rogers and S. Zieger}

\noindent
The third method for representing wave-ice interaction is taken from\linebreak
\cite{art:WS10}. This model treats the ice as a visco-elastic
layer. ${C_{ice,1}}$ is used for ice thickness (m); ${C_{ice,2}}$ is used for
the viscosity ($\mathrm{m^2\,s^{-1}}$); ${C_{ice,3}}$ is used for density
($\mathrm{kg\,m^{-3}}$); ${C_{ice,4}}$ is used for effective shear modulus
(Pa); ${C_{ice,5}}$ is not used. An example setting is 
${C_{ice,1...4}}=[0.1, 1.0, 917.0, 0.0]$.

The ${k_r}$ modified by ice is incorporated into the governing
equation~(\ref{eq:bal_plane}) via the $C_g$ and $C$ calculations on the
left-hand side; e.g. \citet[][and subsequent unpublished work]{art:RH09}.  The
modified wavenumber ${k_r}$ produces effects analogous to shoaling and
refraction by bathymetry. No special action is required to activate the
shoaling effect. To activate the refraction effect, the model should be
compiled with switch {\code REFRX}. With this switch, the model computes
refraction based on spatial gradients in phase velocity, rather than the
simpler, original approach of computing refraction based on spatial gradients
in water depth. These effects are demonstrated in the regression test {\file
ww3\_tic1.3} which is provided with the code.

This method of $S_{ice}$ ({\code IC3}) is much more expensive than
 {\code IC1} or {\code IC2}.  Relative efficiency improves with larger 
numbers of MPI processes (due to  better scaling), but a factor 3 increase 
in overall computation time is not unusual. 
