\vssub
\subsection{~Source terms for wave-ice interactions}
\vsssub


Wave-ice interaction processes have been the topic of many investigations. In general, wave-ice interactions require 
a description of the ice properties that usually include at least the ice concentration (fraction of ocean surface covered by ice), 
mean ice thickness, and maximum floe diameter. Indeed, the ice is often broken into pieces (the floes) that can have a wide variety of sizes, 
and these sizes strongly modify the dispersion and wave-ice interaction processes. 


In the present version of \ww, the different options 
for treating the ice are the result of ongoing research effort and are not completely self-consistent. In particular, the forcing fields may take 
different meanings for different source terms. There are now 5 different 
version of dissipation processes activated with the switches {\code IC1}, {\code IC2},  {\code IC3}, {\code IC4} and {\code IC5} that can be combined with 2 
different versions of scattering effects  {\code IS1} and {\code IS2}. The second scatering routine, because it was the only routine 
to use a maximum floe diameter, also contains an estimation of 
ice break-up and resulting maximum floe diameter and some dissipation due to creep. 


At present it is not possible to combine dissipation parametrizations designed for frazil or pancake ice (  {\code IC3} or {\code IC4}) with 
a parametrization designed for the ice pack, such  as {\code IC2}. Further, all parameterizations are not yet completely consistent: 
for example the floe size is not yet taken into account in some modified dispersion relations that take into account the ice, and 
the spatial variability of the ice properties, in particular the thickness, is generally not taken into account. As a result, 
the various ice effects have only been tested in very  few real conditions \citep[e.g.][]{art:Aea16}.
We expect to have a more streamlined way of combining various processes in future versions of \ww, possibly using 
a maximum floe diameter to call one or the other routines. 


In several source terms, a modified dispersion relation can be used. In particular {\code IC2} and {\code IS2} share 
the optional use of the \cite{art:LMC88} dispersion relation for unbroken ice, 
\begin{equation}
\sigma^2 =  \left(gk_{ice} + B k_{ice}^5\right)  / \left(1/\tanh ( k_{ice}H) +\frac{\rho_{ice}  h k_{ice}} {\rho_{w} }\right),
\end{equation}
\begin{equation}
c_g =  (g+(5 + 4 k_{ice} M)Bk_{ice}^5)/(2\sigma(1+k_{ice}M)^2).
\end{equation}

\noindent
$B$ and $M$ quantify the effects of, respectively, ice bending due to waves and ice inertia. 
The group velocity under the ice, derived from the same relation, is used in the module {\code W3SIS2MD} and computed in  
{\code W3DISPMD}.  See \citet{art:LMC88} for details.\\


This equation is only solved when ICEDISP=TRUE in the {\F MISC} namelist. Otherwise, $k_{ice}=k$, just like in open water. 
Note that the effect of $k_{ice}$ is limited to wave breaking and dissipation, and is not passed back to the main program. \\

