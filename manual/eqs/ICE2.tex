\vsssub
\subsubsection{~$S_{ice}$: Damping by sea ice (Liu et al.)} \label{sec:ICE2}
\vsssub

\opthead{ICE2}{\ws/NRL}{E. Rogers and S. Zieger}

\noindent
The second method for representing wave-ice interaction is based on
the papers by \cite{art:LMC88} and \cite{art:LHV91}. This is a model
for  attenuation by a sea ice cover, derived on the assumption that
dissipation is caused by turbulence in the boundary layer between the
ice floes and the water layer, with the ice modeled as a continuous
thin elastic plate. Input ice parameters are ice thickness (in meters)
and an eddy viscosity in the turbulent boundary layer beneath the ice,
${\nu}$. Here, ${C_{ice,1}}$ represents the former and ${C_{ice,2}}$
represents the latter. This source function (IC2) is non-uniform in
frequency space. The parameters ${C_{ice,3}}$,...,${C_{ice,5}}$ are
not used.

\vspace{\baselineskip} \noindent
With IC2 and IC3, the sea ice effects requires solution of a new
dispersion relation. For IC2, the key equations are:
\begin{equation}\label{eq:ice1}
  {\sigma}^2 = ({gk_r} + {Bk_r^5})/(\coth({k_r}{h_w}) + {k_r}{M})
\end{equation}
\begin{equation}\label{eq:ice2}
  {C_g} = (g + (5 + 4{k_r}{M}){B}{k_r^5})/(2{\sigma}(1+{k_r}{M})^2)
\end{equation}
\begin{equation}\label{eq:ice3}
  {\alpha} = (\sqrt{{\nu\sigma}}{k_r)}/({C_g}\sqrt{2}(1+{k_r}{M}))
\end{equation}
In our notation, $h_w$ is water depth and $h_i$ is ice thickness.
The variables $B$ and $M$ quantify the effects of the bending of
the ice and inertia of the ice, respectively. Both of these
variables depend on $h_i$ (for these equations, see
\citeauthor{art:LMC88}, \citeyear{art:LMC88} and \citeauthor{art:LHV91}
\citeyear{art:LHV91}).

\vspace{\baselineskip} \noindent
In the case of IC2, though the ${k_r}$ is calculated, its effect
is not passed back to the main program. The only effect is via
${k_i}$ (dissipation).
