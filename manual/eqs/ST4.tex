\vsssub
\subsubsection{~$S_{in} + S_{ds}$: Ardhuin et al. 2010} \label{sec:ST4}
\vsssub

\opthead{ST4}{\ws}{F. Ardhuin}

\noindent 
This parameterization uses a positive part of the wind input that is takem
from WAM cycle 4, with an ad hoc reduction of $u_\star$ is implemented in
order to allow a balance with a saturation-based dissipation.  This correction
also reduces the drag coefficient at high winds. This is done by reducing the
wind input for high frequencies and high winds. For this, $u_\star$ in
eq. (\ref{eq:SinWAM4}) is replaced by $u_\star '(k)$ defined for each
frequency as

\begin{equation}
\left(u_\star '\right)^2=\left|u_\star^2 \left(\cos \theta_u, \sin
\theta_u \right) - \left|s_u\right| \int_0^k \int_0^{2 \pi}
\frac{S_{in}\left(k',\theta \right)}{C}  \left(\cos \theta, \sin
\theta \right)  {\mathrm d} k' \mathrm d
\theta,\label{ustarp}\right|
\end{equation}

\noindent 
where the sheltering coefficient $\left|s_u\right|\sim 1$ can be used to tune
the stresses at high winds, which would be largely overestimated for
$s_u=0$. For $s_u > 0$ this sheltering is also applied within the diagnostic
tail in eq. (\ref{eq:tauhfint}), which requires the estimation of a
3-dimensional look-up table for the high frequency stress, the third parameter
being the energy level of the tail.

The swell dissipation parameterization of \cite{art:ACC09} is activated by
setting $s_0$ to a non-zero integer value, and is given by a combination of
the viscous boundary layer value,

\begin{equation}
\cS_{out,vis}\left(k,\theta\right) = - s_5 \frac{\rho_a}{\rho_w}\left\{ 2 k \sqrt{2
\nu \sigma}\right\}  N \left(k,\theta\right) \label{Sds_turb},\label{eq:Dore}
\end{equation}

\noindent
with the turbulent boundary layer expression 
\begin{equation}
\cS_{out,tur} \left(k,\theta\right) = - \frac{\rho_a}{\rho_w}\left\{  16 f_e
\sigma^2 u_{\mathrm{orb},s} / g \right\}
 N\left(k,\theta\right) \label{Sds_visc}.\label{eq:swell_turb}
\end{equation}

\noindent
giving the full term 
\begin{equation}
\cS_{out} \left(k,\theta\right) = r_{vis} \cS_{out,vis}\left(k,\theta\right)  + 
 r_{tur} \cS_{out,tur}\left(k,\theta\right)  
 \label{eq:swell_comb}
\end{equation}

\noindent
where the two weights $ r_{vis} $ and $r_{tur}$ are defined from 
a modified  air-sea boundary layer significant Reynolds number $\mathrm{Re} = 2
u_{\mathrm{orb},s} H_s / \nu_{a}$ 

\begin{eqnarray}
 r_{vis} &=& 0.5 (1- \tanh((\mathrm{Re}-\mathrm{Re}_{c})/s_7) \\
 r_{tur}&=& 0.5 (1+ \tanh((\mathrm{Re}-\mathrm{Re}_{c})/s_7) 
\end{eqnarray}
The significant surface orbital velocity is defined by
\begin{equation} u_{\mathrm{orb},s} = 2 \left [  \int \!\!\!\! \int
      \sigma^3 \: N(k,\theta) \: dk d\theta \right ] ^{1/2}
      \: . \label{eq:ub_orbs} \end{equation}

\noindent 
The first equation (\ref{eq:Dore}) is the linear viscous decay by
\cite{art:Dore78}, with $\nu_a$ the air viscosity and $s_5$ is an $O(1)$
tuning parameter. A few tests have indicated that a threshold Re$_{c}=2 \times
10^5 \times (4~\mathrm{m}/H_s)^{(1-s_6)}$ provides reasonable result with
$s_6=0$, although it may also be a function of the wind speed, and we have no
explanation for the dependence on $H_s$.  With $s_6=1$, a constant threshold
close to $2 \times 10^5$ provides similar -- but less accurate -- results.


%\begin{landscape}
\begin{table} \begin{center} {\tiny
\begin{tabular}{|l|c|c|c|c|c|c|} \hline \hline
Par.         &  WWATCH var.       & namelist & TEST451 & TEST451f & TEST405  & TEST500 \\
\hline
  $z_u$ &  ZWND                       & SIN4 & 10.0    & 10.0    & 10.0      & 10.0 \\
  $\alpha_0$ &  ALPHA0                & SIN4 & 0.0095  & 0.0095   & 0.0095    &  0.0095 \\
  $\beta_{\mathrm{max}}$ & BETAMAX    & SIN4 & 1.52    &\textbf{1.33}& \textbf{1.55}    &  1.52   \\
  $p_{\mathrm{in}}$ &  SINTHP         & SIN4 & 2       & 2       & 2         &  2      \\
  $z_\alpha$ &  ZALP                  & SIN4 & 0.006   & 0.006   & 0.006     &  0.006 \\
  $s_u$ &  TAUWSHELTER                & SIN4 & 1.0     & 1.0     & \textbf{0.0} & 1.0   \\
  $s_1$ &  SWELLF                     & SIN4 & 0.8     & 0.8     & 0.8       &  0.8 \\
  $s_2$ &  SWELLF2                    & SIN4 & -0.018  & -0.018  & -0.018    &  -0.018 \\
  $s_3$ &  SWELLF3                    & SIN4 &  0.015  &  0.015  &  0.015    &  0.015 \\
  $\mathrm{Re}_c$ &  SWELLF4                    & SIN4 & $10^5$  & $10^5$  & $10^5$    & $10^5$  \\
  $s_5$ &  SWELLF5                    & SIN4 & 1.2     & 1.2     & 1.2       &  1.2 \\
  $s_6$ &  SWELLF6                    & SIN4 & 0.      & 0.      & 0.        & 0.   \\
  $s_7$ &  SWELLF7                    & SIN4 & 2.3$\times 10^5$ & 2.3$\times 10^5$  & 0.0       &  0.0 \\
  $z_r$ &  Z0RAT                      & SIN4 & 0.04    & 0.04    & 0.04      &  0.04 \\
  $z_{0,\max}$ &  Z0MAX               & SIN4 & 1.002   & 1.002   &\textbf{0.002}&  1.002 \\
\hline
\end{tabular} }
 \end{center}
\caption{Parameter values for TEST451, TEST451f, TEST405 and TEST500 source term
parameterizations that can be reset via the SIN4 namelist. TEST451 generally
provides the best results at global scale when using ECMWF winds, with the
only serious problem being a low bias for $H_s > 8$~m.  TEST451f corresponds
to a retuning for CSFR wind reanalysis from NCEP/NCAR \citep{art:CFSRR10}, and
has almost no bias all the way to $H_s = 15$~m. Simumations and papers
prepared before March 2012, used a slightly different TEST441 and TEST441f
which can be recovered by setting SWELLF7 to 0.  TEST405 is slightly superior
for short fetches, and TEST500 is intermediate in terms of quality but it also
includes depth-induced breaking in the same formulation, and thus may be more
appropriate for depth-limited conditions.  Please note that the name of the
variables only apply to the namelists. In the source term module the names are
slightly different, with a doubled first letter, in order to differentiate the
variables from the pointers to these variables, and the SWELLFx are combined
in one array SSWELLF. Bold values are different from the default values set by
ww3\_grid.} \label{tab:ST4_parSIN}
\end{table}
%\end{landscape}


Eq. (\ref{eq:swell_turb}) is a parameterization for the
nonlinear turbulent decay. When comparing model results to observations, it
was found that the model tended to underestimate large swells and overestimate
small swells, with regional biases. This defect is likely due, in part, to
errors in the generation or non-linear evolution of theses swells. However, it
was chosen to adjust $f_e$ as a function of the wind speed and direction,

\begin{equation}
f_e = s_1 f_{e,GM} + \left[\left|s_3\right| + s_2 \cos
(\theta-\theta_u)\right]u_\star / u_{\mathrm{orb}},\label{fevar}
\end{equation}

\noindent 
where $f_{e,GM}$ is the friction factor given by Grant and Madsen's
(1979)\nocite{art:GM79} theory for rough oscillatory boundary layers without a
mean flow, using a roughness length adjusted to $r_z$ times the roughness for
the wind $z_1$. The coefficients $s_1$ is an $O(1)$ tuning parameter, and the
coefficients $s_2$ and $s_3$ are two other adjustable parameters for the
effect of the wind on the oscillatory air-sea boundary layer. When $s_2 < 0$,
wind opposing swells are more dissipated than following swells. Further, if
$s_3 > 0$, $\cS_{out}$ is applied to the entire spectrum and not just the
swell.

The dissipation term is parameterized from the wave spectrum
saturation. Because the directional wave spectrum were too narrow when using a
saturation spectrum integrated over the full circle \citep{art:AL06}, we
restricted the integration over a sector of half-width $\Delta_\theta$,
\begin{equation}
B'\left(k,\theta\right)=
\int_{\theta-\Delta_\theta}^{\theta+\Delta_\theta} \sigma k^3 cos^{\mathrm{sB}}\left(\theta-
\theta^{\prime}\right) N(k,\theta^{\prime}) \mathrm d
\theta^{\prime} \label{defBofkprime},
\end{equation}
As a result, a sea state with two systems of same energy but opposite
direction will typically produce less dissipation than a sea state with all
the energy radiated in the same direction.

We finally define our dissipation term as the sum of the saturation-based term
and a cumulative breaking term $S_{\mathrm{bk,cu}}$,
\begin{eqnarray}
\cS_{ds}(k,\theta)& =&  \sigma
 \frac{C_{\mathrm{ds}}^{\mathrm{sat}}}{B^2_r} \left[ \delta_d
\max\left\{ B\left(k\right) -
B_r,0\right\}^2 \right.
\nonumber \\
  & & +  \left(1-\delta_d \right) \left. \max\left\{B'\left(k,\theta \right)- B_r
 ,0\right\}^2\right]N(k,\theta)  \nonumber \\
 & & + \cS_{\mathrm{bk,cu}}(k,\theta) + \cS_{\mathrm{turb}}(k,\theta) \label{Sds_all}.
\end{eqnarray}
where
\begin{equation}
B\left(k \right)=\max\left\{B'(k,\theta), \theta \in [0,2
\pi[\right\} \label{defBof}.
\end{equation}
The combination of an isotropic part (the term that multiplies $ \delta_d$)
and a direction-dependent part (the term with $1-\delta_d$) was intended to
allow some control of the directional spread in resulting spectra.

The cumulative breaking term $\cS_{\mathrm{bk,cu}}$ represents the smoothing
of the surface by big breakers with celerity $C'$ that wipe out smaller waves
of phase speed $C$. Due to uncertainties in the estimation of this effect in
various observations, we use the theoretical model of
\cite{art:Aea09}. Briefly, the relative velocity of the crests is the norm of
the vector difference, $\Delta_C =\left|\mathbf{C}-\mathbf{C}'\right|$, and
the dissipation rate of short wave is simply the rate of passage of the large
breaker over short waves, i.e. the integral of $\Delta_C \Lambda(\mathbf{C})
d\mathbf{C}$, where $\Lambda (\mathbf{C}) d\mathbf{C}$ is the length of
breaking crests per unit surface that have velocity components between $C_x$
and $C_x+dC_x$, and between $C_y$ and $C_y+dC_y$ \citep{art:Phi85}.  Here
$\Lambda$ is inferred from breaking probabilities. Based on Banner et
al. (2000, figure 6, $b_T=22
\left(\varepsilon-0.055\right)^2$)\nocite{art:BBY00}, and taking their
saturation parameter $\varepsilon$ to be of the order of $1.6
\sqrt{B'(k,\theta)}$, the breaking probability of dominant waves waves is
approximately
\begin{equation}
P=56.8\left(\max\{\sqrt{B'(k,\theta)}-\sqrt{B'_r},0\}\right)^2.\label{PBanner}
\end{equation}
However, because they used a zero-crossing analysis, for a given wave scale,
there are many times when waves are not counted because the record is
dominated by another scale: in their analysis there is only one wave at any
given time.  This tends to overestimate the breaking probability by a factor
of 2 \citep{art:FAB10}, compared to the present approach in which we consider
that several waves (of different scales) may be present at the same place and
time. We correct for this effect simply, dividing $P$ by 2.

With this approach the spectral density of crest length (breaking or not) per
unit surface $l(\mathbf{k})$ such that $\int l(\mathbf{k}) \mathrm{d}k_x
\mathrm{d}k_y$, we take
\begin{equation}
l(\mathbf{k})= 1/(2\pi^2 k),
\end{equation}
and the spectral density of breaking crest length per unit surface is
$\Lambda(\mathbf{k})=l(\mathbf{k})P(\mathbf{k})$.  Assuming that any breaking
wave instantly dissipates all the energy of all waves with frequencies higher
by a factor $r_{\mathrm{cu}}$ or more, the cumulative dissipation rate is
simply given by the rate at which these shorter waves are taken over by larger
breaking waves, times the spectral density, namely
\begin{equation}
\cS_{\mathrm{bk,cu}}(k,\theta) = -C_{\mathrm{cu}}  N \left(k,\theta\right) \int_{f' < r_{\mathrm{cu}} f } \Delta_C \Lambda(\mathbf{k'}) \mathrm{d\mathbf{k'}},
\label{Sds_cu1}
\end{equation}
where $r_{\mathrm{cu}}$ defines the maximum ratio of the frequencies of long
waves that will wipe out short waves.  This gives the source term,
\begin{eqnarray}
\cS_{\mathrm{bk,cu}}(k,\theta) &=&  \frac{-14.2 C_{\mathrm{cu}}}{\pi^2}  N \left(k,\theta\right)
 \nonumber \\
& &\int_0^{ r^2_{\mathrm{cu}} k }\int_0^{2\pi}
\max \left\{\sqrt{B(f',\theta')}-\sqrt{B_r},0\right\}^2
\mathrm{d}\theta' \mathrm{d}k'.
\label{Sds_sat_isotropic}
\end{eqnarray}
We shall take $r_{\mathrm{cu}}=0.5$, and $C_{\mathrm{cu}}$ is a tuning
coefficient expected to be of order 1, which also corrects for errors in the
estimation of $l$.

\begin{table} \begin{center} {\small
\begin{tabular}{|l|c|c|c|c|c|} \hline \hline
Par.         &  WWATCH var. & namelist & TEST451 & TEST405  & TEST500 \\
\hline
  $p$                               &  WNMEANP        & SDS4 & 0.5                   & 0.5 &  0.5    \\
  $p_{\mathrm{tail}}$               &  WNMEANPTAIL    & SDS4 & 0.5                   & 0.5 &  0.5 \\
  $f_{\mathrm{FM}}$                 &  FXFM3          & SDS4 & 9.9                   & \textbf{2.5} &  9.9  \\
                                    & SDSC1           & SDS4 & 0   & 0                              & \textbf{1.0} \\
  $C_{\mathrm{ds}}^{\mathrm{sat}}$  & SDSC2           & SDS4 & $-2.4\times 10^{-5}$  & $-2.2\times 10^{-5}$  & \textbf{0.0} \\
  $C_{\mathrm{ds}}^{\mathrm{BCK}}$  & SDSBCK          & SDS4 & 0                     & 0  & \textbf{0.185} \\
  $C_{\mathrm{ds}}^{\mathrm{HCK}}$  & SDSHCK           & SDS4 & 0                    & 0  & \textbf{1.5} \\
  $\Delta_\theta$                   &  SDSDTH         & SDS4 & 80                    & 80 &  80 \\
  $B_r$                             &  SDSBR          & SDS4 & 0.0009                & \textbf{0.00085} & 0.0009    \\
  $C_{\mathrm{cu}}$                 & SDSCUM         & SDS4 & -0.40344               & \textbf{0.0}    & -0.40344     \\
  ${\mathrm{s_B}}$                  & SDSCOS           &SDS4 & 2.0                   &  0.0 & 2.0 \\
   $B_0$                             &  SDSC4          & SDS4 & 1.0                  & 1.0 & 1.0   \\
  $p^{\mathrm{sat}}$                &  SDSP           & SDS4 & 2.0  & 2.0 &  2.0   \\
  $C_{\mathrm{turb}}$               & SDSC5           & SDS4  & 0.0 & 0.0 &  0.0   \\
  $\delta_d$                        & SDSC6           & SDS4  & 0.3 & 0.3 &  0.3   \\
  $C$                               & NLPROP          & SNL1  & $2.5\times 10^7$ & $\mathbf{2.7\times 10^7}$  & $2.5\times 10^7$   \\
 \hline \hline
\end{tabular} } \end{center}

\caption{Same as table \ref{tab:ST4_parSIN}, for the SDS4 and SNL1
namelists. Bold values are different from the default values set by 
 ww3\_grid.} \label{tab:ST4_parSDS}
\botline
\end{table}


Finally, the wave-turbulence interaction term of \cite{art:TB02} and \cite{art:AJ06}),
is given by

\begin{equation}
\cS_{\mathrm{ds}}^{\mathrm{TURB}}\left(k,\theta\right) = - 2
C_{\mathrm{turb}} \sigma \cos(\theta_u - \theta) k \frac{\rho_a
u_\star^2}{g \rho_w}  N\left(k,\theta\right) .
\end{equation}

\noindent
The coefficient $C_{\mathrm{turb}}$ is of order 1 and can be used to adjust for
ocean stratification and wave groupiness.

All relevant source term parameters can be set via the namelists SIN4 and SDS4
to yield parameterizations TEST441b, TEST405, both described by
\cite{art:Aea10} or TEST500 described by \cite{art:FA12}. Please note that the
DIA constant $C$ has been slightly adjusted in TEST441b, $C=2.5\times
10^7$. TEST441f corresponds to a re-tuned wind input formulation when using
NCEP/NCAR winds.
