\vsssub
\subsubsection{~$S_{in} + S_{ds}$: BYDRZ} \label{sec:ST6}
\vsssub

\opthead{ST6}{AUSWEX, Lake George}{S. Zieger}

\noindent
Input and dissipation source terms are based on measurements taken at Lake
George, Australia. This dataset revealed a number of new features, namely: (i)
full air-flow separation with a relative reduction of wind input for
conditions of strong winds/steep waves; (ii) dependence of the wave growth
rate on wave steepness; and (iii) enhancement of input in the presence of wave
breaking \citep{art:Dea06,art:Bea07}. Later modification of the input term
accounts for negative input to attenuate growth for adverse/oblique winds.
The input source term is given as
\begin{eqnarray}
S_{in}(k,\theta) & = & \frac{\rho_a}{\rho_w}\, \sigma\,\gamma(k,\theta)\,N(k,\theta) \:\:\: ,
\label{eq:ST601} \\ \gamma(k,\theta) &=& G\,\sqrt{B_n}\,W \:\:\: ,
\label{eq:ST602} \\
G                &=& 2.8-\Bigl (1+\tanh (10\sqrt{B_n}W-11) \Bigr) \:\:\: ,
\label{eq:ST603} \\
B_n              &=& A(k)\,N(k)\sigma\,k^3 \:\:\: ,
\label{eq:ST604} \\
W                &=& \left (\frac{U_{10}}{c}\,-\,1 \right )^2 \:\:\: .
\label{eq:ST605}
\end{eqnarray}

\noindent
In (\ref{eq:ST601})$-$(\ref{eq:ST605}) $\rho_a$ and $\rho_w$ are the densities
of air and water, respectively, $U_{10}$ is the wind speed at 10\,m height,
$k$ is the wavenumber, $c$ refers to the phase speed $\sigma/k$, and $A(k)$ is the
directional narrowness. The spectral saturation (\ref{eq:ST604}), introduced
by \citet{art:Phi84}, is a convenient measure of steepness $ak$ and was
implemented as a function of omni-directional wave action density and
narrowness $A(k)$ of the directional distribution as a function of wavenumber
\citep{art:BGM02}.  The omni-directional action density is obtained by
integration over all directions: $N(k)=\int N(k,\theta)d\theta$.  \linebreak
The inverse of the directional narrowness $A(k)$ is defined as \linebreak
$A^{-1}(k) =$ $\int_{0}^{2\pi} [{N(k,\theta)}/{N_{\max}(k)}] d\theta$, where
$N_{\max}(k)=\max\bigl \{N(k,\theta)\bigr \}$, for all directions
$\theta\in[0,2\pi]$ \citep{art:BS87}.


Wave models typically scale
growth with friction velocity $u_\star$ rather than $U_{10}$ following scaling
in the form of $C_d=u^2_\star/U^2_{10}$, where $C_d$ is the drag
coefficient. Therefore a flag ({\code \&SIN6 SINU10=F}) was implemented to
allow selection of wind speed scaling $U_{10} = 28\,u_{\star}$ adopted in
equation (\ref{eq:Snyder}) based on \citet{art:Sea81} and \citet{art:KHH84}.
\begin{eqnarray}
W_1 & = & \mathrm{max}^2 \left \{ 0,\frac{U_{10}}{c}\ \cos(\theta-\theta_w)-1
\right \} \:\:\: , \label{eq:ST606-1} \\
W_2 & = & \mathrm{min}^2 \left \{ 0,\frac{U_{10}}{c}\ \cos(\theta-\theta_w)-1
\right \} \:\:\: .\label{eq:ST606-2} 
\end{eqnarray}

\noindent
The directional distribution of $W$ is implemented as the sum of favourable
winds (\ref{eq:ST606-1}) and adverse winds (\ref{eq:ST606-2}):
$W=W_1-a_0\,W_2$, so that they complement one another (i.e. $W=\{W_1\cup
W_2$\}).  The growth rate for adverse winds is negative \citep{art:Dea06} and
is applied after the constraint of the wave-supported stress $\tau_w$ is
met. The value of $a_0$ is a tuning parameter in the parameterization of the
input and is adjustable through namelist parameter {\code \&SIN6 SINA0=0.04}.

One important part of the input is the calculation of the momentum flux
between atmosphere and ocean and vice versa. Close to the surface, the stress
$\vec{\tau}$ can be written as the sum of the viscose and wave-supported
stress: $\vec{\tau} = \vec{\tau}_{v} + \vec{\tau}_{w}$. The wave-supported
stress $ \vec{\tau}_{w}$ is used as a principal constraint for the wind input
and cannot exceed the total stress $\vec{\tau} \le \vec{\tau}_{tot}$.  Here
the total stress is detemined by the flux parameterization:
$\vec{\tau}_{tot}=\rho_a u_\star|u_\star|$. The wave-supported stress $\tau_n$
can be calculated by integration over the wind-momentum-input function:

\begin{equation}\label{eq:ST609}
   \vec{\tau}_w = \rho_w g \int_{0}^{2\pi} \int_{0}^{k_{max}}
   \frac{S_{in}(k^\prime,\theta)}{c} \Bigl (\cos\theta,\sin\theta \Bigr )
   dk^\prime d\theta \:\:\: . 
\end{equation}

\noindent
Computation of the wave-supported stress (\ref{eq:ST609}) includes the
resolved part of the spectrum up to the highest discrete wavenumber $k_{max}$,
as well as the stress supported by short waves. To do so, an $f^{-5}$
diagnostic tail is assumed beyond the highest frequency in the energy density
spectrum. In order to satisfy the constraint and in the case of $\vec{\tau} >
\vec{\tau}_{tot}$, a wavenumber dependent factor $L$, as given in
(\ref{eq:ST610}), is applied to reduce energy from the high frequency part of
the spectrum: $S_{in}(k^\prime)=L(k^\prime)\,S_{in}(k^\prime)$.

\begin{equation}\label{eq:ST610}
L(k^\prime) = \min \Bigl \{ 1, \exp \bigl ( \mu\,[1- U_{10}/c] \bigr ) \Bigr
\}  \:\:\: .
\end{equation}

\noindent
The reduction (\ref{eq:ST610}) is a function of wind speed and phase speed and
follows an exponential form designed to reduce energy from the discrete part
of the spectrum. The strength of reduction is controlled by coefficient $\mu$,
which has a greater impact at high frequencies and only little impact on the
energy-dominant part of the spectrum. The value of $\mu$ is dynamically
calculated by iteration at each integration time step.

For the drag coefficient, parameterization (\ref{eq:ST607}) was selected and
implemented as switch {\code FLX4}. The parameterization was proposed by
\citet{art:Hwa11} and accounts for saturation, and even decline 
for extreme winds, of the sea drag at wind speeds in excess of 30\,m~s$^{-1}$.
To prevent $u_\star$ from dropping to zero at very strong winds
($U_{10}\ge50.33$m~s$^{-1}$) expression (\ref{eq:ST607}) was modified to yield
$u_\star=2.026$m~s $^{-1}$. To allow bulk adjustment to any uniform bias in
the wind input field, the factor in expression $C_d \times 10^4$ on the left
hand side of (\ref{eq:ST607}) was substituted with $C_d \times \mathrm{FAC}$
and added as the namelist parameter {\code \&FLX4 CDFAC=1.0E-4}.

\begin{equation}\label{eq:ST607}
C_d \times 10^4 = 8.058 + 0.967 U_{10} - 0.016 U_{10}^2 \:\:\: .
\end{equation}

\noindent
For the viscous drag coefficient equation (\ref{eq:ST608}) is selected which
was parameterized by \citet{art:Tea10} as a function of wind speed applying
data from \citet{art:BP98}:

\begin{equation}\label{eq:ST608}
  C_v \times 10^3 = 1.1 - 0.05 U_{10} \:\:\: .
\end{equation}

\noindent
For wavebeaking and whitecapping dissipation the Lake George field study
yields three features: (i) the threshold behaviour of wave breaking from
\citep{art:BBY00,art:BGM02}, (ii) the cumulative dissipative effect due to
breaking and dissipation of short waves affected by longer waves
\citep{pro:Don01, pro:BY05, art:YB06, art:Bea10}, and (iii) that waves 
will not break unless they exceed a generic steepness in which case the wave
breaking probability depends on the level of excedence above the threshold.
For waves below the critical threshold, wave-breaking and whitecapping
dissipation is inactive. The wave breaking and dissipation term is implemented
as:
\begin{equation}\label{eq:ST620}
  S_{ds}(k,\theta) = \bigl [ T_1 + T_2 \bigr ]\ N(k,\theta) \:\:\: ,
\end{equation}

\noindent
where $T_1$ is the inherent breaking term and $T_2$ accounts for the
cumulative effect of short-wave breaking due to longer waves at each
frequency. The inherent breaking term $T_1$ is the only breaking-dissipation
term if this frequency is at or below the spectral peak. Once the peak moves
below this particular frequency, $T_2$ becomes active and progressively more
important as the peak downshifts further.

The threshold spectral density $F_{\mathrm{T}}$ is calculated as shown in
(\ref{eq:ST621}), where $k$ is the wavenumber and with
$\varepsilon_{\mathrm{T}}=0.035^2$ being the empirical constant
\citep{art:Bea07}:

\begin{equation}\label{eq:ST621}
  F_{\mathrm{T}}(k)=\frac{\varepsilon_{\mathrm{T}}}{A(k)\,k^3} \:\:\: . 
\end{equation}

\noindent
Let the level of exceedence above the critical threshold spectral density (at
which stage wave breaking is predominant) be defined as
$\Delta(k)=F(k)-F_{\mathrm{T}}(k)$. Furthermore, let $\mathcal{F}(k)$ be a
generic spectral density used for normalization, then the inherent breaking
component can be calculated as

\begin{equation}\label{eq:ST622}
T_1(k)=a_1 A(k)\frac{\sigma}{2\pi} \left [ \frac{\Delta(k)}{\mathcal{F}(k)}
\right ]^L \:\:\: .
\end{equation}

\noindent
The cumulative dissipation term is not local in frequency space and is 
based on an integral that grows towards higher frequencies, dominating at 
smaller scales:

\begin{equation}\label{eq:ST623}
T_2(k)=a_2 \int\limits_0^k A(k) \frac{c_g}{2\pi} \left [
\frac{\Delta(k)}{\mathcal{F}(k)} \right ]^M\!\!dk \:\:\: .
\end{equation}

\noindent
The dissipation terms (\ref{eq:ST622})$-$(\ref{eq:ST623}) depend on five
parameters: a generic spectral density $\mathcal{F}(k)$ used for
normalization, and four coefficients $a_1$, $a_2$, $L$, and $M$.  The
coefficients $L$ and $M$ control the strength of the normalised threshold
spectral density $\Delta(k)/\mathcal{F}(k)$ of the dissipation
terms. \citet{art:BGM02} introduced the directional narrowness $A(k)$ as a
correction for the directional spread to reconcile observed values of the
wavebreaking threshold across different bands. \citet{art:Bea07} and
\citet{art:Bab09} showed that such correction is not necessary and observed
differences in the threshold of \citet{art:BGM02} were due to cumulative
effects not accounted for. Consequently, the directional narrowness parameter
is set to unity $A(k)\approx 1$ in equations
(\ref{eq:ST621})$-$(\ref{eq:ST623}).

% -------------------------------------------------------------------
\begin{table} \begin{center}
\begin{tabular}{|l|c|c|c|c|} \hline \hline
Par.         &  WWATCH var. & namelist & $U_{10}=28u_\star$ & $U_{10}=u_\star C^{-1/2}_{d}$  \\
\hline
  $-$              &  SINU10      & SIN6     &  F      &   T           \\
  $F_{\mathrm{T}}$ &  SDSET       & SDS6     &  T      &   T            \\
  $a_1$            &  SDSA1       & SDS6     & 6.24E-8 &   6.50E-8      \\
  $L$              &  SDSP1       & SDS6     &  4      &   4            \\
  $a_2$            &  SDSA2       & SDS6     & 8.74E-6 &   8.40E-7      \\
  $M$              &  SDSP2       & SDS6     &  4      &   4            \\
\hline
  $a_0$            &  SINA0       & SIN6     & 0.04    &   0.04         \\
  $b_1$            &  SWLB1       & SWL6     & 0.25E-3 &   0.25E-3      \\
  $\mathrm{FAC}$   &  CDFAC       & FLX4     & 1.00E-4 &   1.00E-4      \\
 \hline \hline
\end{tabular} \end{center}
\caption{Summary of calibration parameters for BYDRZ source terms.
         Values tabulated represent default model settings.}
\label{tab:ST601} \botline \end{table}
% -------------------------------------------------------------------

\citet{art:RBW12} recently calibrated the dissipation terms based
on duration-limited academic tests and proposed four sets of coefficients.
These four models result in different shapes for dissipation terms $T_1$ and
$T_2$ from which the linear model ($L$=1 and $M$=1, normalised by the spectral
density, $\mathcal{F}=F$) insufficiently dissipates the strong input at high
frequencies. Dissipation models using the threshold spectral density
$(\mathcal{F}=F_{\mathrm{T}})$ show similar behaviour when compared to
non-dimensional empirical growth curves.  As a result, the highly nonlinear
dissipation model ($L$=4 and $M$=4) is selected \citep{art:RBW12}. The
calibration coefficients listed in Table~\ref{tab:ST601} differ somewhat from
those of \citet{art:RBW12} mainly due to the fact that the wave-supported
stress $\vec{\tau}_w$ is implemented in the form of vector components and the
scaling model for wind speed is customisable in the wind input
parameterization.  Dissipation parameters in the upper part of
Table~\ref{tab:ST601} obtained from initial calibration are considered fixed.

In the absence of wavebreaking and whitecapping other mechanisms of wave
dissipation are present which are referred to as swell dissipation. Here,
swell dissipation is parameterized in terms of the interaction of waves with
oceanic turbulence \citep{bk:Bab11} and is given in (\ref{eq:ST624}). The
value for the non-dimensional proportionality coefficient $b_1$ is
customizable through namelist parameter {\code \&SWL6 SWLB1=0.250E-03}.

\begin{equation}\label{eq:ST624}
  S_{swl}(k,\theta) = -\frac{2}{3}b_1 \sigma\ \sqrt{B_n}\ N(k,\theta) \:\:\:.
\end{equation}

\noindent
The source term {\code ST6} has been calibrated with flux parameterization
{\code FLX4}. Bulk adjustment to the wind filed can be achieved by re-scaling
the drag parameterization {\code FLX4} through namelist parameter {\code
\&FLX4 CDFAC=1.0E-4}.  This has a similar effect to tuning variable
$\beta_{max}$ in TEST451, equations (\ref{eq:SinWAM4}) and
(\ref{eq:tauhfint}), which is customizable through namelist parameter {\code
BETAMAX} (see section \ref{sec:ST3}--\ref{sec:ST4}). \citet{pro:Aea11} and
\citet{art:RA13} listed different sets of values that allows to adjust to
different wind fields. When optimizing the wave model, it is recommended to
only re-tune parameters $a_0$, $b_1$ and $\mathrm{FAC}$. Again, $\mathrm{FAC}$
can potentially eliminate a bias in the wind field, which typically changes
with the selection of the reanalysis product. This reduction was also tested
for extreme wind conditions such as hurricanes \citep{art:ZBRY13}.  In global
hindcast, the coefficient for the negative input $a_0$ can be used to tune
lower wave height in the scatter comparisons, whereas the swell-coefficient
$b_1$ is able to adjust higher waves.

