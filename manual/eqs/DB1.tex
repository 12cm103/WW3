\vsssub
\subsubsection{~$S_{db}$: Battjes and  Janssen 1978} \label{sec:DB1}
\vsssub

\opthead{DB1 / MLIM}{Pre-WAM}{J. H. G. M. Alves}

\noindent
The implementation in \ws\ of depth-induced breaking algorithms is intended to
extend the applicability of the model to within shallow water environments,
where wave breaking, among other depth-induced transformation processes,
becomes important.

For this reason the approach of \citet[][henceforth denoted as
BJ78]{pro:BJ78}, which is based on the assumption that all waves in a random
field exceeding a threshold height, defined as a function of bottom topography
parameters, will break. For a random wave field, the fraction of waves
satisfying this criterion is determined by a statistical description of
surf-zone wave heights (i.e., a Rayleigh-type distribution, truncated at a
depth-dependent wave-height maximum).

The bulk rate $\delta$ of spectral energy density dissipation of the fraction
of breaking waves, as proposed by BJ78, is estimated using an analogy with
dissipation in turbulent bores as

%-------------------------------%
% Battjes Janssen surf breaking %
%-------------------------------%
% eq:BJ78_base

\begin{equation}
\delta = 0.25 \: Q_b \: f_m \: H_{\max}^2  , \label{eq:BJ78_base}
\end{equation}

\noindent
where $Q_b$ is the fraction of breaking waves in the random field, $f_m$ is
the mean frequency and $H_{\max}$ is the maximum individual height a component
in the random wave field can reach without breaking (conversely, above which
all waves would break). In BJ78 the maximum wave height $H_{\max}$ is defined
using a Miche-type criterion \citep{art:Miche44},

% eq:BJ78_Miche

\begin{equation}
\bar{k} H_{\max} = \gamma_M \tanh ( \bar{k} d )
 , \label{eq:BJ78_Miche}
\end{equation}

\noindent
where $\gamma_M$ is a constant factor. This approach also removes energy in
deep-water waves exceeding a limiting steepness. This can potentially result
in double counting of dissipation in deep-water waves. Alternatively,
$H_{\max}$ can be defined using a McCowan-type criterion, which consists of
simple constant ratio

% eq:BJ78_McC

\begin{equation}
H_{\max} = \gamma \: d  , \label{eq:BJ78_McC}
\end{equation}

\noindent
where $d$ is the local water depth and $\gamma$ is a constant derived from
field and laboratory observation of breaking waves. This approach will
exclusively represent depth-induced breaking.  Although more general breaking
criteria for $H_{\max}$ as a simple function of local depth exist
\citep[e.g.,][]{art:TG83}, it should be noted that the coefficient $\gamma$
refers to the maximum height of an individual breaking wave within the random
field. \cite{art:M1894} calculated the limiting wave-height-to-depth ratio for
a solitary wave propagating on a flat bottom to be 0.78, which is still used
presently as a conservative criteria in engineering applications. The average
value found by \cite{pro:BJ78} was $\gamma = 0.73$. More recent analyses of
waves propagating over reefs by \cite{art:Nel94, art:Nel97} suggest a ratio of
0.55.

The fraction of breaking waves $Q_b$ is determined in terms of a Rayleigh-type
distribution truncated at $H_{\max}$ (i.e., all broken waves have a height
equal to $H_{max}$), which results in the following expression:

% eq:BJ78_Qb

\begin{equation}
\frac{1 - Q_b}{-\ln Q_b} = \left ( \frac{H_{rms}}{H_{\max}} \right ) ^{2}
 , \label{eq:BJ78_Qb}
\end{equation}

\noindent
where $H_{rms}$ is the root-mean-square wave height. In the current
implementation, the implicit equation (\ref{eq:BJ78_Qb}) is solved for $Q_b$
iteratively. With the assumption that the total spectral energy dissipation
$\delta$ is distributed over the entire spectrum so that it does not change
the spectral shape \citep{art:EB96} the following depth-induced breaking
dissipation source function is obtained

% eq:BJ78

\begin{equation}
\cS_{db} (k,\theta) = - \alpha \frac{\delta}{E} F(k,\theta)
       = - 0.25 \: \alpha \: Q_b \: f_m \frac{H_{\max}^2}{E} F(k,\theta)
 , \label{eq:BJ78}
\end{equation}

\noindent
where $E$ is the total spectral energy, and $\alpha = 1.0$ is a tunable
parameter. The user can select between Eqs.~(\ref{eq:BJ78_Miche}) and
(\ref{eq:BJ78_McC}), and adjust $\gamma$ and $\alpha$. Defaults are
Eq.~(\ref{eq:BJ78_McC}), $\gamma = 0.73$ and $\alpha = 1.0$.
