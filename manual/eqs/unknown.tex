\vsssub
\subsubsection{~$S_{??}$: Reflection by shoreline and icebergs} \label{sec:unknown}
\vsssub

\opthead{unknown}{\ws}{F. Ardhuin}

\noindent 
Reflections by shorelines and icebergs is activated by setting namelists
parameters REFCOAST, REFSUBGRID or REFBERG (in namelist REF1) to non-zero
values that are the target reflection coefficients $R_0^2$ for the wave
energy.  From these values $R_0^2$ may be varied with wave height and period
following a Miche-type parameter: this is activated by setting REFFREQ to a
non-zero value, and is based on the field measurements of \cite{art:EHG94}.
These coefficients can also be made to vary spatially, by setting REFMAP to a
non-zero value. In that case ww3\_grid will expect to find a extra line after
the reading of the water depths and obstructions in ww3\_grid.inp.

Wave reflection at the shoreline varies from a fraction of a percent to about
40\% of the incoming wave energy, and may have important consequences for the
directional wave spectrum, and the wave climate in otherwise sheltered
locations \citep{pro:ORe99}. Wave reflection is also extremely important for
the generation of seismic noise by ocean waves.

Because reflection involve wave trains with different directions, in a model
like \ws, their interaction can only be represented through a source term in
the right hand side. Nevertheless, this is physically linked to propagation.

In practice, for the regular grids, the reflection source term puts into the
reflected wave directions the proper amount of energy that will be taken away
by propagation at the next time step. When neglecting the cross-shore current,
this is

\begin{equation} 
\cS_{ref}(k,\theta) = 
\int R^2(k,\theta,\theta') \frac{C_g(k)}{\Delta A} \left[\cos \theta \Delta y + \sin \theta  \Delta x \right] N(k,\theta') \mathrm{d} \theta'
\end{equation}
where $R^2$ is an energy reflection coefficient, and $\Delta x$ and $\Delta y$
are the grid spacing along the two axes, and $\Delta A$ is the cell area. The
definition of the shoreline direction from the land/sea mask is explained in
\cite{art:Aea11}. It has not been adapted for curvilinear grids.

In the case of unstructured grids, the spectral density of outgoing directions
on the boundary is directly set to the expected reflected value and the
boundary condition is handled specifically by the the numerical schemes.

The reflection coefficient $R^2$ is taken to be non-zero only for the
directions for which $\cos(\theta-\theta')<0$, and its magnitude is the
product of a reflection coefficient $R_0^2(k)$, integrated over the scattered
directions $\theta$, and a directional distribution $R_2(\theta,\theta')$
around the specular direction $\theta_s$,
\begin{equation} 
R^2(k,\theta,\theta')  =  R_0^2(k) R_2(\theta,\theta').
\end{equation}

This directional distribution takes three forms: 
\begin{itemize}

\item isotropic in all directions opposite to the incoming direction: this is
      for sub-grid islands and icebergs or sharp shoreline angles, 

\item proportional to $\cos(\theta-\theta_s)^2$ for moderate shoreline angles,

\item proportional to $\cos(\theta-\theta_s)^n$ for small shoreline angles
      (nearly straight shoreline). Where $n=4$ by default and can be changed
      to any value using the REFCOSP\_STRAIGHT namelist parameter in the REF1
      namelist.

\end{itemize}

\noindent
That parameterization is described in detail by \cite{art:AR12}.