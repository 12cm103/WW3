\vsssub
\subsubsection{~$S_{nl}$: Generalized Multiple \dia\  (\gmd)} \label{sec:NL3}
\vsssub

\opthead{NL3}{\ws}{H. L. Tolman}

\noindent
The \gmd\ has been developed as an extension to the \dia. Its development is
documented in a set of Technical notes \citep{tol:MMAB03a, tol:MMAB05a, tol:MMAB08b,
tol:MMAB10d}, reports \citep{tol:Oahu04a, tol:Hal09a, tol:Kona11b}, and papers
\citep{tol:OMOD04, tol:OMOD13d}. As part of the development of the \gmd, a
holistic genetic optimization technique was developed \citep{tol:OMOD13e}. A
package to perform this optimization within \ws\ was first provided by
\cite{tol:MMAB10e}. The most recent version of this package is version
\genever\ \citep{\generef}.

The \gmd\ expands on the \dia\ in three ways. First, the definition of the
representative quadruplets is expanded. Second, the equations are developed
for arbitrary depths, including the description of strong interactions in
extremely shallow water \citep[e.g.,][]{art:Web78}. Third, multiple
representative quadruplets are used.

The \gmd\ allows for arbitrary configurations of the representative
quadruplet, by expanding on the resonance conditions (\ref{eq:resonance}) as

\begin{equation} \left .
\begin{array}{ccc}
  \sigma_1    & = & a_1 \: \sigma_r \\
  \sigma_2    & = & a_2 \: \sigma_r \\
  \sigma_3    & = & a_3 \: \sigma_r \\
  \sigma_4    & = & a_4 \: \sigma_r \\
  \theta_{12} & = & \theta_1 \pm \theta_{12}   \
\end{array} \:\:\: \right \rbrace \:\:\: , \label{eq:gmd_res}
\end{equation}

\begin{table}
\caption{One, two, or three parameter definitions of the representative
         quadruplet in the \gmd. $\bk_d$ or ($\sigma_d, \theta_d$) represents
         the discrete spectral grid point for which the discrete interaction
         contributions are evaluated. All quadruplets are aligned with the
         discrete directions by taking $\bk_1 + \bk_2 // \bk_d$.}
  \label{tab:gmd_quad_def}
\vspace{\baselineskip}
\begin{center}
\begin{tabular}{ccccccc} \hline
 parameters  & $a_1$ & $a_2$ & $a_3$  & $a_4$ & $\theta_{12}$ &
 $\sigma_r$  \\ \hline
$(\lambda)$
    &    1    &    1    & $1+\lambda$  & $1-\lambda$ &    0  & $\sigma_d$ \\
$(\lambda, \mu)$
    & $1+\mu$ & $1-\mu$ & $1+\lambda$ & $1-\lambda$  &
                                   implied\textsuperscript{*}& $\sigma_d$ \\
$(\lambda, \mu, \theta_{12})$
    & $1+\mu$ & $1-\mu$ & $1+\lambda$ & $1-\lambda$  & free  &
                                                 $\frac{\sigma_d}{1+\mu}$ \\
\hline
\end{tabular}
\end{center}
\hspace{55mm} \textsuperscript{*}~assuming $\bk_1 + \bk_2 = \bk_3 + \bk_4 =
2\bk_d$ 
\vspace{\baselineskip}
\botline
\end{table}

\noindent 
where $a_1 + a_2 = a_3 + a_4$ to satisfy the general resonance conditions
(\ref{eq:resonance_2}), $\sigma_r$ is a reference frequency, and $\theta_{12}$
is the angular gap between the wavenumbers $\bk_1$ and $\bk_2$. The latter
parameter can either be implicit to the quadruplet definition, or can be an
explicitly tunable parameter. With this, a one- ($\lambda$), two- ($\lambda,
\mu$) or three-parameter ($\lambda, \mu, \theta_{12}$) quadruplet definition
have been constructed as outlined in Table~\ref{tab:gmd_quad_def}. Note that,
unlike in the \dia, all quadruplets are evaluated for the actual water depth
and frequency.

In the \gmd, the discrete interaction are computed for arbitrary depths.
Somewhat surprisingly, interactions computed for the $F(f, \theta)$ spectrum
and converted to the native \ws\ spectrum $N(k, \theta)$ using a Jacobian
transformation proved more easily optimizable than computing the interaction
contributions for the latter spectrum directly. Furthermore, a two-component
scaling function was introduced with a `deep' scaling function for the
traditionally represented weak interactions in intermediate to deep water, and
a `shallow' scaling function representing strong interactions in extremely
shallow water. With these modifications, the discrete interaction
contributions (\ref{eq:snl_dia}) of the \dia\ become
\begin{eqnarray}
\left ( \begin{array}{c}
  \delta S_{nl,1} \\ \delta S_{nl,2} \\ \delta S_{nl,3} \\ \delta S_{nl,4} 
\end{array} \right )  & = &  
\left ( \begin{array}{r} -1 \\  -1 \\ 1 \\ 1 \end{array} \right )
  \left ( \frac{1}{n_{q,d}} C_{\rm{deep}} B_{\rm{deep}} + 
          \frac{1}{n_{q,s}} C_{\rm{shal}} B_{\rm{shal}} \right ) \times \nonumber \\
 &  & \hspace{20mm} \left [ \:\:\:\:
                    \frac{c_{g,1}F_1}{k_1 \sigma_1} 
                    \frac{c_{g,2}F_2}{k_2 \sigma_2} \left ( 
                    \frac{c_{g,3}F_3}{k_3 \sigma_3} +
                    \frac{c_{g,4}F_4}{k_4 \sigma_4} \right )
                                                         \right . \nonumber \\
 &  &  \hspace{20mm} \left . - \:\:
                    \frac{c_{g,3}F_3}{k_3 \sigma_3} 
                    \frac{c_{g,4}F_4}{k_4 \sigma_4} \left ( 
                    \frac{c_{g,1}F_1}{k_1 \sigma_1} + 
                    \frac{c_{g,2}F_2}{k_2 \sigma_2} \right ) \:\:\:\:
                                                    \right ] \:\:\:
, \label{eq:gmd_dsnl} 
\end{eqnarray}

\noindent
where $B_{\rm{deep}}$ and $B_{\rm{shal}}$ are the deep and shallow water
scaling functions

\begin{equation}
B_{\rm{deep}} = \frac{k^{4+m} \sigma^{13-2m} }{(2\pi)^{11} \:
                 g^{4-m} \: c_g^2 } \:\:\: , \label{eq:gmd_B_deep}
\end{equation}

\begin{equation}
B_{\rm{shal}} = \frac{g^2 \: k^{11}}{(2\pi)^{11} \: c_g} (kd)^n
\:\:\: , \label{eq:gmd_B_shal}
\end{equation}

\noindent
with $m$ and $n$ as tunable parameters, $C_{\rm{deep}}$ and $C_{\rm{shal}}$
in Eq.~(\ref{eq:gmd_dsnl}) are the corresponding deep and shallow water tunable proportionality
constants, and $n_{q,d}$ and $n_{q,s}$ are the number of representative
quadruplets with deep and shallow water scaling, respectively, representing
the feature of the \gmd\ that multiple representative quadruplets can be used.

In the namelists {\F snl3} and {\F anl3} the user defines the number of
quadruplets, and per quadruplet $\lambda$, $\mu$, $\theta_{12}$,
$C_{\rm{deep}}$ and $C_{\rm{shal}}$. Values of $m$ and $n$ are defined once,
and used for all quadruplets. Finally relative depth below which deep water
scaling is not used and above which shallow water scaling is not used are
defined. Examples of some of the \gmd\ configurations from \cite{tol:MMAB10d}
are included in the example input file {\file ww3\_grid.inp}
in \para\ref{sub:ww3grid}. The default setting is to reproduce the traditional
\dia.

Note that the \gmd\ is significantly more complex that the \dia\ formulation,
and requires evaluation of the quadruplet layout for every spectral frequency
(compared to a single layout used for the \dia). For effective computation,
quadruplet layouts are pre-computed and stored in memory for a set of
nondimensional depths. Even with these and other optimizations, the \gmd\ is
roughly twice as expensive to compute for a single representative quadruplet
than the \dia\ when using the one-parameter quadruplet layout. Using the two-
or three-parameter quadruplet layout, the \gmd\ has four rather than two
quadruplet realizations, making the \gmd\ per quadruplet four times as
expensive as the traditional \dia. Using multiple representative quadruplets
is linearly additive in computational costs. For more in depth assessment of
computational costs of a model including the \gmd, see \cite{tol:MMAB10d} and
\cite{tol:OMOD13d}.
