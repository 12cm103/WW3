\vsssub
\subsubsection{~$S_{ice}$: Damping by sea ice (simple)} \label{sec:ICE1}
\vsssub

\opthead{IC1}{\ws/NRL}{E. Rogers and S. Zieger}

\noindent
Experimental routines for respresentation of the effect of ice on waves have
been implemented using the switches {\code IC1}, {\code IC2}, and {\code
IC3}. These effects can be presented in terms of a complex wavenumber

\begin{equation}\label{eq:waveno}
     {k} = {k_r} + i{k_i},
\end{equation}

\noindent
with the real part ${k_r}$ representing impact of the sea ice on the physical
wavelength and propagation speeds, producing effects analogous to shoaling and
refraction by bathymetry, whereas the imaginary part of the complex
wavenumber, ${k_i}$, is an exponential decay coefficient
${k_i}({x},{y},{t},\sigma $) (depending on location, time and frequency,
respectively), producing wave attenuation.  The ${k_i}$ is introduced as
${S_{ice}}/{E}=-2{C_g}{k_i}$, where ${S_{ice}}$ is a source term (see also
\cite{bk:WAM94}, pg. 170).

The effect of sea ice on ${k_i}$ is used for all three of the source functions
({\code IC1}, {\code IC2}, {\code IC3}). The effect of sea ice on ${k_r}$ does
not apply to {\code IC1}, has not been implemented for {\code IC2}, and has
been implemented for {\code IC3}.

With the ice source functions, {\code IC1}, {\code IC2}, and {\code IC3}, ice
concentration is not a required input, but if ice concentration has been read
in, the source function will be scaled by ice concentration.

In the case of ice, up to five parameters are allowed. These can be referred
to generically as ${C_{ice,1}}$, ${C_{ice,2}}$,...,${C_{ice,5}}$.  The meaning
of the ice parameters will vary depending on which ${S_{ice}}$ routine is
selected.

In the case where any of the ice and mud source functions are activated with
the switches {\code IC1}, {\code IC2}, {\code IC3}, {\code BT8}, or {\code
BT9}, {\file ww3\_shel} will anticipate intructions for 8 fields (5 for ice,
then 3 for mud). These are given prior to the ``water levels'' information.
The new fields can also be specified as homogeneous field using {\file
ww3\_shel.inp}.

The reader is referred to the regression tests {\file ww3\_tic1.1-3} and
{\file ww3\_tic2.1} for examples of how to use the new ice source functions.

The first implemented method ({\code IC1}) is for the user to specify
${k_i(x,t)}$, which is uniform in frequency space, ${C_{ice,1}}={k_i}$. The
parameters ${C_{ice,2}}$,...,${C_{ice,5}}$ are not used. Descriptions specific
to IC2 and IC3 are given in following sections.

\textrm{\textit{\underline{Limitations of the code:}}} The interface for the
new mud and ice coefficients have only been implemented for {\file
ww3\_shel}. Interface for {\file ww3\_multi} will be available in a future
revision.  

\textrm{\textit{\underline{Limitations of the physics:}}} The
scattering of waves from sea ice is not considered via {\code IC1}, {\code
IC2}, {\code IC3}. This is an important physical process \citep{art:Wad75},
but since it is conservative, it should be treated separately from the source
functions {\code IC1}, {\code IC2}, {\code IC3}, which are intented to
represent non-conservative effects of sea ice. This work is in progress.
