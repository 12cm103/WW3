\vsssub
\subsubsection{~Parameter settings in modules}
\vsssub

Several modules have internally used parameter settings. Here only parameter
settings that are generally usable or impact model behavior are presented.

\vspace{\baselineskip} \noindent
Physical and mathematical constants : \hfill {\file constants.ftn}
\begin{vlist}
\vit{grav  }{rp}{Acceleration of gravity $g$.
                \hfill (m s$^{-2}$)}
\vit{dwat  }{rp}{Density of water. \hfill(kg m$^{-3}$)}
\vit{dair  }{rp}{Density of air. \hfill(kg m$^{-3}$)}
\vit{nu\_air}{rp}{Kinematic viscosity of air \hfill (m$^2$ s$^{-1}$)}
\vit{nu\_water}{rp}{Kinematic viscosity of water \hfill (m$^2$ s$^{-1}$)}
\vit{sed\_sd}{rp}{Specific gravity of sediment \hfill (--)}
\vit{kappa }{rp}{Von Karman's constants \hfill (--)}
\vit{pi    }{rp}{$\pi$.}
\vit{tpi   }{rp}{$2\pi$.}
\vit{hpi   }{rp}{$0.5\pi$.}
\vit{tpiinv}{rp}{$(2\pi)^{-1}$.}
\vit{hpiinv}{rp}{$(0.5\pi)^{-1}$.}
\vit{rade  }{rp}{Conversion factor from radians to degrees.}
\vit{dera  }{rp}{Conversion factor from degrees to radians.}
\vit{radius}{rp}{Radius of the earth. \hfill (m)}
\vit{g2pi3i}{rp}{$g^{-2} (2\pi)^{-3}$.}
\vit{g1pi1i}{rp}{$g^{-1}(2\pi)^{-1}$.}
\end{vlist}

\noindent
Wave model initialization module : \hfill {\file w3initmd.ftn}
\begin{vlist}
\vit{critos}{rp}{Critical fraction of resources used for output only
                     (triggers warning output).}
\vit{wwver }{cp}{Version number of the main program.}
\vit{switches}{cp}{Switches taken from {\file bin/switch}.}
\end{vlist}

\noindent
I/O module ({\file mod\_def.ww3}) : \hfill {\file w3iogrmd.ftn}
\begin{vlist}
\vit{vergrd}{cp\opt}{Version number of file {\file mod\_def.ww3}.}
\vit{idstr }{cp\opt}{ID string for file.}
\end{vlist}

\noindent
I/O module ({\file out\_grd.ww3}) : \hfill {\file w3iogomd.ftn}
\begin{vlist}
\vit{verogr}{cp\opt}{Version number of file {\file out\_grd.ww3}.}
\vit{idstr }{cp\opt}{ID string for file.}
\end{vlist}

\noindent
I/O module ({\file out\_pnt.ww3}) : \hfill {\file w3iopomd.ftn}
\begin{vlist}
\vit{veropt}{cp\opt}{Version number of file {\file out\_pnt.ww3}.}
\vit{idstr }{cp\opt}{ID string for file.}
\vit{acc   }{cp}{Relative offset below which output point is moved to grid
                 point.}
\end{vlist}

\noindent
I/O module ({\file track\_o.ww3}) : \hfill {\file w3iotrmd.ftn}
\begin{vlist}
\vit{vertrk}{cp\opt}{Version number of file {\file track\_o.ww3}.}
\vit{idstri}{cp\opt}{ID string for file {\file track\_i.ww3}.}
\vit{otype }{cp}{Array dimension.}
\end{vlist}

\noindent
I/O module ({\file restart.ww3}) : \hfill {\file w3iorsmd.ftn}
\begin{vlist}
\vit{verini}{cp\opt}{Version number of file {\file restart.ww3}.}
\vit{idstr }{cp\opt}{ID string for file.}
\end{vlist}

\noindent
I/O module ({\file nest.ww3}) : \hfill {\file w3iobcmd.ftn}
\begin{vlist}
\vit{verbpt}{cp\opt}{Version number of file {\file nest.ww3}.}
\vit{idstr }{cp\opt}{ID string for file.}
\end{vlist}

\noindent
I/O module ({\file partition.ww3}) : \hfill {\file w3iosfmd.ftn}
\begin{vlist}
\vit{vertrt}{cp\opt}{Version number of file {\file partition.ww3}.}
\vit{idstr }{cp\opt}{ID string for file.}
\end{vlist}

\noindent
Multi-grid model input update : \hfill {\file wmupdtmd.ftn}
\begin{vlist}
\vit{swpmax}{ip}{Maximum number of extrapolation sweeps allowed to make maps
                 match in conversion from input from input grid to wave model
                 grid.}
\end{vlist}

\noindent
Several routines contain interpolation tables that are set up with parameter
statements, including

\vspace{\baselineskip} \noindent
Solving the dispersion relation : \hfill {\file w3dispmd.ftn}
\begin{vlist}
\vit{nar1d }{ip}{Dimension of interpolation tables.}
\vit{dfac  }{rp}{Maximum nondimensional water depth $kd$.}
\vit{ecg1  }{ra}{Table for calculating  group velocities from
                 the frequency and the depth.}
\vit{ewn1  }{ra}{Id. wavenumbers.}
\vit{n1max }{i }{Largest index in tables.}
\vit{dsie  }{r }{Nondimensional frequency increment.}
\end{vlist}

\noindent
Shallow water quadruplet lookup table for \gmd\ : \hfill {\file w3snl3md.ftn}
\begin{vlist}
\vit{nkd   }{ip}{Number of nondimensional depths in storage array.}
\vit{kdmin }{rp}{Minimum relative depth in table.}
\vit{kdmax }{rp}{Maximum relative depth in table.}
\vit{lammax}{rp}{Maximum value for $\lambda$ or $\mu$.}
\vit{delthm}{rp}{Maximum angle gap $\theta_{12}$ ($\degree$).}
\end{vlist}

\noindent
Shallow water lookup table for nonlinear filter : \hfill {\file w3snlsmd.ftn}
\begin{vlist}
\vit{nkd   }{ip}{Number of nondimensional depths in storage array.}
\vit{kdmin }{rp}{Minimum relative depth in table.}
\vit{kdmax }{rp}{Maximum relative depth in table.}
\vit{abmax }{rp}{Maximum value for $a_{34}$.}
\end{vlist}

\noindent
Lookup table for $\beta$ in Tolman and Chalikov 1996 : \hfill {\file w3src2md.ftn}
\begin{vlist}
\vit{nrsiga}{ip}{Array dimension ($\sigma_a$).}
\vit{nrdrag}{ip}{Array dimension ($C_d$).}
\vit{sigamx}{rp}{Maximum nondimensional frequency $\tilde{\sigma}_a$.}
\vit{dragmx}{rp}{Maximum drag coefficient $C_d$}
\end{vlist}

\noindent
Lookup table for \ldots in WAM-4 / ECWAM : \hfill {\file w3src3md.ftn}
\begin{vlist}
\vit{kappa  }{rp}{von K{\'a}rm{\'a}n's constant.}
\vit{nu\_air}{rp}{air viscosity.}
\vit{itaumax}{ip}{size of stress dimension.}
\vit{jumax  }{ip}{size of wind dimension.}
\vit{iustar }{ip}{size of ustar dimension.}
\vit{ialpha }{ip}{size of Charnock dimension.}
\vit{ilevtail}{ip}{size of tail level dimension.}
\vit{umax   }{rp}{Maximum wind speed in table.}
\vit{tauwmax}{rp}{Maximum ustar in table.}
\vit{eps1   }{rp}{Small number for stress convergence.}
\vit{eps2   }{rp}{Small number for stress convergence.}
\vit{niter  }{ip}{Number of iterations in stress table.}
\vit{xm     }{ip}{power of TAUW/TAU in roughness parameterization.}
\vit{jtot   }{ip}{Number of points in discretization of tail.}
\end{vlist}

\noindent
Lookup tables Ardhuin et al. 2010 : \hfill {\file w3src3md.ftn}

Combination of previous two sets of parameters. \\

\noindent
Table of error functions in bottom friction : \hfill {\file w3sbt4md.ftn}
\begin{vlist}
\vit{sizeerftable}{ip}{Size of table for erf function.}
\vit{xerfmax}{rp }{Maximum value of x in table of erf(x).}
\vit{wsub   }{rpa}{Weights for 3-point Gauss-Hermitte quadrature.}
\vit{xsub   }{rpa}{x values for 3-point Gauss-Hermitte quadrature.}
\end{vlist}

\noindent
Some model parameters are set using parameter statements.

\vspace{\baselineskip}
\noindent
Source term computation and integration : \hfill {\file w3srcemd.ftn}
\begin{vlist}
\vit{offset}{rp\opt}{Offset $\epsilon$ in Eq.~(\ref{eq:implicit_st}).}
\end{vlist}

\noindent
Auxiliary data storage : \hfill {\file w3adatmd.ftn}
\begin{vlist}
\vit{mpibuf}{ip}{Number of buffers used in \mpi\ data transpose.}
\end{vlist}

\noindent
Some service routines contain parameters that can be used to influence, for
instance, the model output.

\vspace{\baselineskip}
\noindent
Array I/O including text outputs : \hfill {\file w3arrymd.ftn}
\begin{vlist}
\vit{icol  }{ip\opt}{Set maximum columns on output (now set to 80).}
\vit{nfrmax}{ip\opt}{Set maximum number of frequency in spectral print plots
                     (now set to 50).}
\end{vlist}

\noindent
Automatic unit number assignment : \hfill {\file wmunitmd.ftn}
\begin{vlist}
\vit{unitlw}{ip}{Lowest unit number to be considered.}
\vit{unithg}{ip}{Highest unit number to be considered.}
\vit{inplow, inphgh}{}{}
\vit{      }{ip}{Range of input file unit numbers.}
\vit{outlow, outhgh}{}{}
\vit{      }{ip}{Range of output file unit numbers.}
\vit{scrlow, scrhgh}{}{}
\vit{      }{ip}{Range of scratch file unit numbers.}
\end{vlist}

\noindent
Creating spectral bulletins : \hfill {\file w3bullmd.ftn}
\begin{vlist}
\vit{nptab, nfld, npmax, bhsmin, bhsdrop, dhsmax,}{}{}
\vit{dptmx, ddmmax, ddwmax, agemin}{}{}
\vit{}{i/rp}{Setting of size of bulletin as well as various filter values.}
\end{vlist}


