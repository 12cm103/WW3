\subsubsection{Data in {\F w3gdatmd}.} \label{sub:gdatmd}

\noindent
The module {\F w3gdatmd} in the file {\file w3gdatmd.ftn} contains the data
describing a set of models. First several variables and parameters embedded in
the module are defined:

\begin{vlist}
\vit{ngrids}{i }{Number of models for which space is available.}
\vit{nauxgr}{i }{Auxiliary grids (model input, unified point output).}
\vit{igrid }{i }{Number of model/grid presently selected.}
\vit{isgrd }{i }{Number of spectral model grid presently selected.}
\vit{ipars }{i }{Number of physical model parameter settings presently 
                 selected.}
\vit{flagll}{lp}{Flag for spherical grid (otherwise Cartesian).}
\vit{grid  }{t }{Data structure containing information of spatial grids.}
\vit{grids }{ta}{Array if type {\F grid} and size {\F ngrids}.}
\vit{sgrd  }{t }{Data structure containing information of spectral grids.}
\vit{sgrds }{ta}{Array if type {\F sgrd} and size {\F ngrids}.}
\vit{mpar  }{t }{Data structure containing all other model parameters in
                 sub structures (see below).}
\vit{mpars }{ta}{Array if type {\F mpar} and size {\F ngrids}.}
\end{vlist}

\noindent
For each element of the type {\F grid}, direct aliases are defined with the
same name, as illustrated in Figs.~\ref{fig:struc_1} and \ref{fig:struc_2}.
These variables are:

\begin{vlist}
\vit{nx,ny }{i }{discrete grid dimensions, {\F nx, ny} $\geq 3$.}
\vit{nsea  }{i }{Number of sea points in grid.}
\vit{nseal }{i }{Id. locally stored on present process.}
\vit{trflag}{i }{Type of transparencies used.}
\vit{mapsta}{ia}{Grid status map.}
\vit{mapst2}{ia}{Secondary grid status map.}
\vit{mapsf }{ia}{Storage grid map.}
\vit{mapfs }{ia}{Id.}
\vit{sx,sy }{r }{Spatial grid increments. \hfill ($\degree$ or m)}
\vit{x0,y0 }{r }{Lower left point of spatial grid. \hfill ($\degree$ or m)}
\vit{dtcfl }{r }{Maximum \cfl\ number for spatial propagation.}
\vit{dtcfli}{r }{Id. for intra-spectral propagation.}
\vit{dtmax }{r }{Maximum overall time step.}
\vit{dtmin }{r }{Minimum source term integration time step.}
\vit{dmin  }{r }{Minimum water depth. \hfill (m)}
\vit{ctmax }{r }{Maximum \cfl\ number for depth refraction.}
\vit{fice0/n}{r }{Ice concentration cut-off. \hfill (-)}
\vit{pfmove}{r }{Power factor in GSE alleviation correction for moving grid.}
\vit{zb    }{ra}{Bottom depths on storage grid. \hfill (m)}
\vit{clat(i)}{ra}{(Inverse) cosine of latitude.}
\vit{clats }{ra}{Id.}
\vit{cthg0 }{ra}{Latitude factor in great circle propagation speed.}
\vit{trnx/y}{ra}{Grid box transparencies.\hfill (--)}
\vit{ginit }{l }{Flag for initialization of model/grid.}
\vit{global}{l }{Flag for global grid.}
\vit{fldry }{l }{Flag for 'dry run' (no calculations).}
\vit{flc{\it xx}}{l }{Flags for propagation in all spaces.}
\vit{flsou }{l }{Flag for source term integration.}
\vit{flast }{la}{Flags for source term computation per grid point.}
\vit{gname }{c }{Grid name.}
\vit{filext}{c }{File extension for raw files for this model/grid.}
\end{vlist}

\noindent
Similarly, the structure {\F sgrd} contains the following variables and/or
aliases: 

\begin{vlist}
\vit{nk    }{i }{Number of discrete spectral wavenumbers, {\F nk} $\geq 3$.}
\vit{nk2   }{i }{Extended wavenumber range.} 
\vit{nth   }{i }{Number of discrete spectral directions, {\F nth} $\geq 4$.}
\vit{nspec }{i }{Number of spectral bins.}
\vit{mapwn }{ia}{Map with discrete wavenumber for the one dimensional
                 description of the spectrum.}
\vit{mapth }{ia}{Id. for discrete directions.}
\vit{dth   }{r }{Spectral directional increment. \hfill (rad)} 
\vit{xfr   }{r }{Factor defining discrete frequency increment.}
\vit{fr1   }{r }{Lowest discrete frequency. \hfill (Hz)}
\vit{fte   }{r }{Factor in tail integration of total energy.}
\vit{ftf   }{r }{Id. mean frequency.}
\vit{ftwn  }{r }{Id. mean wavenumber.}
\vit{fttr  }{r }{Id. mean period.}
\vit{ftwl  }{R }{Id. mean wave length.}
\vit{facti{\it n}}{r }{Auxiliary to calculate discrete frequency from
                 continuous frequency.}
\vit{fachfa}{r }{Factor defining parametric tail for the action
                 spectrum $N(k,\theta)$.}
\vit{fachfe}{r }{Id. for the energy spectrum $F(f)$.}
\vit{th    }{ra}{Spectral directions. \hfill (rad)}
\vit{esin  }{ra}{$\sin(\theta)$ for discrete spectral directions.}
\vit{ecos  }{ra}{Id. $\cos(\theta)$.}
\vit{es2   }{ra}{$\sin^2(\theta)$ for entire spectrum.}
\vit{esc   }{ra}{Id. $\sin(\theta)\: \cos(\theta)$.}
\vit{ec2   }{ra}{Id. $\cos^2(\theta)$.}
\vit{sig   }{ra}{Frequencies for discrete wavenumbers. \hfill (rad s$^{-1}$)}
\vit{sig2  }{ra}{Id. for entire discrete spectrum. }
\vit{dsip  }{ra}{Frequency band widths for each wavenumber as used in
                 propagation. \hfill (rad s$^{-1}$)}
\vit{dsii  }{ra}{Id. for spectral integration.}
\vit{dden  }{ra}{Composite band with and conversion to energy for each
                 wavenumber ({\F dden = dth * dsii * sig}). \\
                 \strut \hfill (rad s$^{-1}$)}
\vit{dden2 }{ra}{Id. for entire spectrum.}
\vit{sinit }{l }{Flag for initialization of spectral grid.}
\end{vlist}

\noindent
The structure {\F mpar} contains addition structures and a single aliased
variable:

\begin{vlist}
\vit{pinit }{l }{Flag for initialization of this structure.}
\vit{npars }{t }{Structure with numerical parameters for source term
                 integration (structure {\F npar}).}
\vit{props }{t }{Structure with parameters for propagation schemes
                 (structure {\F prop}).}
\vit{sflps }{t }{Structure with parameters for flux computation (structure
                 {\F sflp}).}
\vit{slnps }{t }{Structure with parameters linear wind input source
                 term (structure {\F slnp}).}
\vit{srcps }{t }{Structure with parameters for input and dissipation source
                 term (structure {\F scrp}).}
\vit{snlps }{t }{Structure with parameters for nonlinear interaction source
                 term (structure {\F snlp}).}
\vit{sbtps }{t }{Structure with parameters for bottom friction source
                 term (structure {\F sbtp}).}
\vit{sdbps }{t }{Structure with parameters for depth-induced breaking
                 term (structure {\F sdbp}).}
\vit{strps }{t }{Structure with parameters for triad interaction source
                 term (structure {\F strp}).}
\vit{sbsps }{t }{Structure with parameters for bottom scattering source
                 term (structure {\F sbsp}).}
\vit{sxxps }{t }{Structure with parameters for arbitrary additional source
                 term (structure {\F sxxp}).}
\end{vlist}

\noindent
The structure {\F npar} contains the following alias pointers:

\begin{vlist}
\vit{facp  }{r }{Composite constant for parametric cut-off.} % in
				% Eq.~(\ref{eq:st_d_5}).}
\vit{xrel  }{r }{$X_r$ in dynamic integration.} %Eq.~(\ref{eq:st_d_6}).}
\vit{xflt  }{r }{$X_f$ in dynamic integration.} % Eq.~(\ref{eq:st_d_7}).}
\vit{fxfm  }{r }{First constant in tail.} % Eq.~(\ref{eq:tail_WAM3}).}
\vit{fxpm  }{r }{Second constant in tail.} % Eq.~(\ref{eq:tail_WAM3}).}
\vit{xft   }{r }{Constant for $f_{2}$ in tail.} % Eq.~(\ref{eq:TC_f}).}
\vit{xfc   }{r }{Constant for $f_{hf}$ in tail.} %Eq.~(\ref{eq:TC_f}).}
\vit{facsd }{r }{Seeding constant $X_{\rm seed}$.} %  in Eq.~(\ref{eq:seed}).}
\vit{fhmax }{r }{Maximum $H_s/d$ ratio in shallow water limiter.}
\end{vlist}

\noindent
The structure {\F prop} contains the following alias pointers. All pointer
are activated by the switches on the right.

\begin{vlist}
\vit{dtme  }{r }{Swell age in disp. corr.      \hfill ({\F !/pr2})}
\vit{clatmn}{r }{Id. minimum cosine of lat.    \hfill ({\F !/pr2})}
\vit{wdcg  }{r }{Factors in width of av. Cg.   \hfill ({\F !/pr3})}
\vit{wdth  }{r }{Factors in width of av. Th.   \hfill ({\F !/pr3})}
%\vit{qtfact}{r }{Factors for size of AOI.      \hfill ({\F !/pr4})}
%\vit{rnfact}{r }{Factor normal dispersion.     \hfill ({\F !/pr4})}
%\vit{rsfact}{R }{Factor tangential dispersion. \hfill ({\F !/pr4})}
\end{vlist}

\noindent
The structure {\F sflp} contains the following alias pointer, presented as
above.

\begin{vlist}
\vit{nittin}{i }{Number of iterations for drag calculation. \\ \strut
                                  \hfill ({\F !/flx2}, {\F !/flx3})}
\vit{cinxsi}{r }{Constant in parametric description of drag. \\ \strut
                                  \hfill ({\F !/flx2}, {\F !/flx3})}
\vit{cap\_id}{i}{Type of cap applied to $C_d$.    \hfill ({\F !/flx3})}
\vit{cd\_max}{r}{Maximum value for $C_d$.         \hfill ({\F !/flx3})}
\end{vlist}

\noindent
The structure {\F slnp} contains the following alias pointer, presented as
above.

\begin{vlist}
\vit{snlc1 }{r }{Proportionality and other constants. \hfill ({\F !/ln1})}
\vit{fspm  }{r }{Factor for $f_{PM}$ in filter.       \hfill ({\F !/ln1})}
\vit{fshf  }{r }{Factor for $f_h$ in filter.          \hfill ({\F !/ln1})}
\end{vlist}

\noindent
The structure {\F srcp} contains the following alias pointers, with the
associated switches again displayed on the right.

\begin{vlist}
\vit{sinc1 }{r }{Combined constant for input.      \hfill ({\F !/st1})}
\vit{sdsc1 }{r }{Combined constant for dissipation.\hfill ({\F !/st1})}
\vit{zwind }{r }{Height at which the wind is defined.
                                               \hfill (m) ({\F !/st2})}
\vit{fswell}{r }{Reduction factor of negative input for swell.
                                                   \hfill ({\F !/st2})}
\vit{shstab, ofstab, ccng, ccps, ffng, ffps}{}{}
\vit{      }{r }{Factors in effective wind speed.  \hfill ({\F !/st2})}
\vit{cdsa{\it n}}{r }{Constants in high-freq. dissipation.
                                                   \hfill ({\F !/st2})}
\vit{sdsaln}{r }{Factor for nondimensional 1-D spectrum.
                                                   \hfill ({\F !/st2})}
\vit{cdsbn }{r }{Constants in parameterization of $\phi$.
                                                   \hfill ({\F !/st2})}
\vit{fpimin}{r }{Minimum value for $f_{p,i}$.      \hfill ({\F !/st2})}
\vit{xfh   }{r }{Constant for turbulent length scale.
                                                   \hfill ({\F !/st2})}
\vit{xfn   }{r }{Constants in combining low and high \\ frequency dissipation.
                                                   \hfill ({\F !/st2})}
\vit{zwnd }{r }{Height at which the wind is defined.
                                               \hfill (m) ({\F !/st3})}
\vit{aalpha}{r }{Minimum value of Charnock coefficient.
                                                   \hfill ({\F !/st3})}
\vit{bbeta}{r }{Wind-wave growth $\beta_{\mathrm{max}}$ parameter.
                                                   \hfill ({\F !/st3})}
\vit{zzalp}{r }{Wave age correction $z_\alpha$ in wind input.
                                                   \hfill ({\F !/st3})}
\vit{ttauwshelter }{}{}
\vit{    }{r }{Sheltering coefficient for input to short waves.
                                                   \hfill ({\F !/st3})}
\vit{sswellfpar }{}{}
\vit{    }{i }{Choice of negative input parameterization.
                                                   \hfill ({\F !/st3})}
\vit{ssdsc1}{r }{Constant for WAM4-part of dissipation.
                                                   \hfill ({\F !/st3})}
\vit{ddelta1}{r }{Weight of $k$ part in WAM4 dissipation.
                                                   \hfill ({\F !/st3})}
\vit{ddelta2}{r }{Weight of $k^2$ part in WAM4 dissipation.
                                                   \hfill ({\F !/st3})}
\vit{wwnmeanp}{r }{Mean wavenumber power for $S_{\mathrm{ds}}$.
                                                   \hfill ({\F !/st3})}
\vit{wwnmeanptail}{r }{Mean wavenumber power for the tail.
                                                   \hfill ({\F !/st3})}
\vit{sstxftwn,sstxftf,sstxftftail }{}{}
\vit{    }{r }{Tail coefficients for mean wavenumber.
                                                   \hfill ({\F !/st3})}
\vit{sswellf}{r }{Reduction factor of negative input for swell.
                                                   \hfill ({\F !/st4})}
\vit{sswellf2}{r }{Extra parameter for negative input.
                                                   \hfill ({\F !/st4})}
\vit{ssdsc2}{r }{Constant for dissipation.
                                                   \hfill ({\F !/st4})}
\vit{ssdsc3}{r }{Modification parameter for saturation power.
                                                   \hfill ({\F !/st4})}
\vit{ssdsc4}{r }{Hard relative threshold for saturation.
                                                   \hfill ({\F !/st4})}
\vit{ssdsc5}{r }{Constant for wave-turbulence term.
                                                   \hfill ({\F !/st4})}
\vit{ssdsbr}{r }{Saturation threshold for wave breaking.
                                                   \hfill ({\F !/st4})}
\vit{ssdsp}{r }{Base value of power of saturation.
                                                   \hfill ({\F !/st4})}
\vit{ssdsdth}{r }{Directional window for saturation.
                                                   \hfill ({\F !/st4})}
\end{vlist}

\noindent
The structure {\F snlp} contains the following alias pointers, presented as
above.

\begin{vlist}
\vit{snlc1 }{r }{Scaled proportionality constant.       \hfill ({\F !/nl1})}
\vit{lam   }{r }{Factor defining quadruplet.            \hfill ({\F !/nl1})}
\vit{kdcon }{r }{Conversion factor for relative depth.  \hfill ({\F !/nl1})}
\vit{kdmn  }{r }{Minimum relative depth.                \hfill ({\F !/nl1})}
\vit{snlsn }{r }{Constants in shallow water factor.     \hfill ({\F !/nl1})}
\vit{iqtpe }{i }{Type of depth treatment.               \hfill ({\F !/nl2})}
\vit{ndpths}{i }{Number of depth for which integration space \\
                 needs to be computed.                  \hfill ({\F !/nl2})}
\vit{nltail}{r }{Tail factor for parametric tail.       \hfill ({\F !/nl2})}
\vit{dpthnl}{ra}{Depths corresponding to {\F ndpths}.   \hfill ({\F !/nl2})}
\end{vlist}

\noindent
The structure {\F sbtp} contains the following alias pointer, presented as
above.

\begin{vlist}
\vit{sbtc1 }{r }{Proportionality constant.        \hfill ({\F !/bt1})}
\end{vlist}

\noindent
The structure {\F sdbp} contains the following alias pointer, presented as
above.

\begin{vlist}
\vit{sdbc1 }{r }{Proportionality constant.        \hfill ({\F !/db1})}
\vit{sdbc2 }{r }{$H_{\max}/d$ ratio.              \hfill ({\F !/db1})}
\vit{fdonly}{l }{Flag for chocking depth only, otherwise use \\
                 Miche criterion.                 \hfill ({\F !/db1})}
\end{vlist}

\noindent
The structure {\F strp} contains the following alias pointer, presented as
above.

\begin{vlist}
\vit{dummy }{r }{Placeholder only.             \hfill ({\F !/tr0-x})}
\end{vlist}

\noindent
The structure {\F sbsp} contains the following alias pointer, presented as
above.

\begin{vlist}
\vit{dummy }{r }{Placeholder only.             \hfill ({\F !/bs0-x})}
\end{vlist}

\noindent
The structure {\F sxxp} contains the following alias pointer, presented as
above.

\begin{vlist}
\vit{dummy }{r }{Placeholder only.             \hfill ({\F !/xx0-x})}
\end{vlist}
