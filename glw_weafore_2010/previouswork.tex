\section{Previous Great Lakes Wave Modeling Efforts}

Forecasts and hindcasts of wind waves at the Great Lakes, made with numerical
prediction schemes or models, have been generated since the early 1970's. Since
then, the techniques applied to wave hindcasting have followed more closely the
latest trends and techniques made available throughout time. Wave forecasting,
however, until recently has favoured older modelling approaches based on
parametric models, that are still applicable in the region due to the
predominance of wave climates associated with short fetches and local wind
generation. 

The need for more detailed description of complex wave generation scenarios, and
shallow-water wave propagation, has provided a push towards upgrading wave
forecasting models in the region. In this section we provide a brief, general
history of both wind-wave hindcasting and forecasting in the Great Lakes. 


\subsection{Hindcasts}

The first long-term Great Lakes wave hindcast product generated using a
numerical scheme for computing waves from wind data is presented in
\cite{resvin76}. The technique was based on overlake wind data estimated from
overland measurements and ship anemometers. These early hindcasts covered a
69-year period (1907-1975), and were a useful source of wave data for
engineering applications in the Great Lakes during the late 1970's and 1980's.

With the advent of a two-dimensional model for wave simulation within the Great
Lakes, the Ontario Ministry of Natural Resources funded in the mid-1980's the
development of a new wave climate database, with the latest available
technology, as part of its shoreline management plan. The approach consisted of
using the \cite{schwet84a} wave model, forced with gridded overlake winds
derived from overland measurements via an estimation technique described in
\cite{philirb78}. Results were assembled onto wave hindcast databases for
individual basins within the Great Lakes system.

In a series of Wave Information Studies (WIS), the USACE has been generating and
constantly upgrading more recent wave hindcast databases in several oceanic
basis, and in major inland water bodies such as the Great Lakes. Fot the latter,
the WIS program has provided two distinct data streams. In the first, the wave
model WISWAVE was run for a 32-year period (1956-1987) over a 10-mile-resolution
grid, using gridded winds derived from land stations via adjustments for the
transition between land-water boundary layers, stability and measurement height.
This first WIS database was later extended for the period 1988-1997, using
improved wind estimates developed in partnership with GLERL that included buoy
measurements available in the Great Lakes since early 1980's. A general
description is provided in \cite{linres01}.

USACE's WIS has recently upgraded the Great Lakes wind-wave hindcast database
with higher-resolution hindcasts produced for Lake Ontario. Wave simulations
were made for a 40-year period (1961-2000), using an upgraded version of WISWAVE
model baptized WAVAD. The hindcasts were generated using a 3km grid covering
Lake Ontario, with wind fields derived from land-based meteorological stations
and buoys and ice concentrations assembled from databases developed by
\cite{asset83}, for the first 14 years, and at GLERL for the remaining period. A
full description of the Lake Ontario hindcasts, which have been made available
since 2003, is provided in USACE's Field Research Facility website
\citep{wisweb}.

\subsection{Forecasts}

A first wave guidance product for the Great Lakes based on numerical schemes was
implemented in 1974 by the US National Weather Service (henceforth NWS), which
consisted of forecasts of sea state extending out to 36 hours, at 12-h
intervals. These early wave forecasts were computed using an automated numerical
scheme developed by \cite{pore74}, based on an adaptation of the method by
\cite{bret70}, both cited in \cite{burrdal97}. 

In response to requests made by the forecasting community to expand the wave
forecasts generated by the NWS in the 1970's, which were limited to 64 fixed
locations, the Great Lakes Environmental Research Laboratory (GLERL) developed a
two-dimensional wind-wave model, which was later implemented for operational
forecasting in the region. The deployed model was a first-generation wind-wave
model, solving a local momentum balance equation over a grid with resolution at
5 km or 10 km for different lakes. Ice coverage is ignored. Detailed history and
description of the GLERL wave model are provided in \cite{schwet84a,schwet84b},
whereas the initial validation that led to its operational implementation are
reported in \cite{liuet84}. 

As pointed out by \cite{liuet84}, the implemented model at GLERL was ``not
without drawbacks. It is purely a wind-wave prediction model and has no
provision for swell propagation at present. [...] In addition, the model is for
deep-water waves and the results may not be accurate for shallow-water waves''.
The concerns expressed by \cite{liuet84} were the central motivation pushing
NOAA/NCEP efforts towards implementing a state-of-the-art, third-generation
spectral wind-wave model, where not only swell propagation but also
shallow-water and nonlinear processes may be represented for a full
two-dimensional wave energy-density spectrum. 

A first implementation of the WAVEWATCH III model at the Great Lakes region was
made in an independent effort by the NWS Weather Forecast Office at Marquette,
Illinois (Thomas Hultquist, personal communication, 2004). Several case studies
were performed, using winds generated with the RAMS model. Results were
promising, and established loosely the feasibility of running WAVEWATCH III
operationally for wind-wave forecasting at the Great Lakes.
